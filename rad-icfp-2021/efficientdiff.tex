\section{Efficient reverse-mode}
\label{sec:efficientrad}

\subsection{Target Language} 

The last language we present, which we call the target language, is an extension to the source language. 
The types and terms are presented in Figure~\ref{fig:target_grammar}.
The language contains lambda abstractions for dealing with continuations. 
We are not interested here with partial applications and our language is pure, so functions take $n$ arguments.
Lambda abstractions and applications are going to be removed during partial evaluation and this notation makes reading slightly easier.

\begin{figure*}[t]
    \setlength{\tabcolsep}{0.3em}
    \centering
    \begin{tabular}{|l c l|l|}
    \hline
    \multicolumn{3}{|c|}{\textbf{Core Grammar}} & \multicolumn{1}{c|}{\textbf{Description}}\\\hline
    !T! & \mbox{::=} & ... & \grammarcomment{Same as Source} \\
    & $\mid$ & !T->T! & \grammarcomment{Function Type}\\ 
    \hline
    !e! & \mbox{::=} & ... & \grammarcomment{Same as Source}\\
    & $\mid$ & !fun (x$_1$,...,x$_n$) -> e! & \grammarcomment{Lambda Abstraction}\\
    & $\mid$ & !e(e$_1$...e$_n$)! & \grammarcomment{Function Application}\\
    & $\mid$ & !scanl (x,y.e) e e! $\mid$ !scanr (x,y.e) e e! & \grammarcomment{Array scan left and right}\\
    & $\mid$ & !shift1L e! & \grammarcomment{Array left shifting}\\
    & $\mid$ & !shift1R e! & \grammarcomment{Array right shifting}\\
    \hline
    \end{tabular}
    \vspace{-0.2cm}
    \caption{Grammar of the target language.}
    \label{fig:target_grammar}
    \end{figure*}
    

In addition, we require the set of operations to be closed under partial differentiation. 
In more detail, for every unary scalar operation !op1!, 
we assumed given an operator $\partial_1$!op1! whose semantics should be the derivative of !op1!.
For every binary operator !op2!, we assume given operators $\partial_1$!op2!, $\partial_2$!op2! 
representing respectively the first and second partial derivative of !op2!.
Similarly, we need to add some functions on arrays. 
!scanl! is a !foldl! that stores all intermediate results. 
!scanr! is similar, but reads the array from right to left, and also stores from right to left.
!reduce! is a !foldl! for which the function is associative. 
This means in practice that it can be computed faster.
We also include a !map3! operator. 
Though not necessary as it can be simulated with two !map2!, it is a convenient notation and can somtimes run faster in practice.
Finally, we add a somewhat strange operator which we call !shift1!. 
It takes an array of size $n$,  shifts all the element of the array by one, 
forgetting the last element, and puts a one in the first place. 
This operator naturally shows up when differentiating !foldl!.

The type system for the extended grammar of target is in Figure~\ref{fig:target_typesystem}. 

\begin{figure*}[tb]
    \centering

    \begin{tabular}{c}
        $\Gamma$, !x!: $\reals$, !y!: $\reals$ $\vdash$ !e1!: $\reals$ 
        $\quad$ $\Gamma$ $\vdash$ !e2!: $\reals$
        $\quad$ $\Gamma$ $\vdash$ !e3!: $\Array{\reals}$
        \\\hline  
        $\Gamma \vdash$ !reduce (x,y.e1) e2 e3!: $\reals$
    \end{tabular}

    \begin{tabular}{c}
        $\Gamma$, !x!: $\reals$, !y!: $\reals$ $\vdash$ !e1!: $\reals$ 
        $\quad$ $\Gamma$ $\vdash$ !e2!: $\reals$
        $\quad$ $\Gamma$ $\vdash$ !e3!: $\Array{\reals}$
        \\\hline  
        $\Gamma \vdash$ !scanl (x,y.e1) e2 e3!: $\Array{\reals}$
    \end{tabular}

    \begin{tabular}{c}
        $\Gamma$, !x!: $\reals$, !y!: $\reals$ $\vdash$ !e1!: $\reals$ 
        $\quad$ $\Gamma$ $\vdash$ !e2!: $\reals$
        $\quad$ $\Gamma$ $\vdash$ !e3!: $\Array{\reals}$
        \\\hline  
        $\Gamma \vdash$ !scanr (x,y.e1) e2 e3!: $\Array{\reals}$
    \end{tabular}

    \begin{tabular}{c}
        $\Gamma$, !x!: $\reals$, !y!: $\reals$ $\vdash$ !e1!: $\reals$ 
        $\quad$ $\Gamma$ $\vdash$ !e2!: $\Array{\reals}$
        $\quad$ $\Gamma$ $\vdash$ !e3!: $\Array{\reals}$
        $\quad$ $\Gamma$ $\vdash$ !e4!: $\Array{\reals}$
        \\\hline  
        $\Gamma \vdash$ !map3 (x,y.e1) e2 e3 e4!: $\Array{\reals}$
    \end{tabular}

    \begin{tabular}{c}
        $\Gamma$, !x1!: !G1!, $\ldots$, !xn!: !Gn! $\vdash$ !e!: !T! 
        \\\hline  
        $\Gamma \vdash$ !fun (x1,...,xn) -> e!: !G1x...xGn->T!
    \end{tabular}

    \begin{tabular}{c}
        $\Gamma$ $\vdash$ !e!: !G1x...xGn -> T!
        $\quad$ $\Gamma$ $\vdash$ !ei!: !Gi! for all $1\leq i\leq n$
        \\\hline  
        $\Gamma \vdash$ !ee1...en!: !T!
    \end{tabular}

    \vspace{-0.2cm}
    \caption{Type system of the target language}
    \vspace{-0.4cm}
    \label{fig:target_typesystem}
    \end{figure*}    

A standard semantics of this language is given in Section~\ref{sec:correctness}.

We now present our transformation from target UNF to target. 
The transformation is presented in Figure~\ref{fig:unf_to_target}.

\begin{figure*}[t]
    \begin{tabular}{r c l}
    $\invUNFSymbol$(!var!$_{\Gamma;i}$) &=& !x$_i$: Ti! \\
    && where !x$_i$! is the $i$-th element in the context $\Gamma$ \\
    $\invUNFSymbol$(!op!$_{\Gamma;n}$) &=& !op$_n$(x$_k$,$\ldots$,x$_{k+n}$)! \\
    && where !(x$_k$,$\ldots$,x$_{k+n}$)! are the last $n$ variables of $\Gamma$ \\ 
    $\invUNFSymbol$(!e$_1$ $\comp$ e$_2$!) &=& !let x=$\invUNFSymbol$(e$_1$) in $\invUNFSymbol$(e$_2$)! where !$\Gamma$,x:A $\vdash$ e$_2$:$\Gamma$,A,B! \\ 
    $\invUNFSymbol$(!map2!$_{\Gamma;x,y.e}$) &=& !map2 (x,y.e) X$_1$ X$_2$! where !$\Gamma$,X$_1$,X$_2$ $\vdash$ map2$_{\Gamma;x,y.e}$! \\
    $\invUNFSymbol$(!reduce!$_{\Gamma;x,y.e_1;e_2}$) &=& !reduce (x,y.e$_1$) $e_2$ X! where !$\Gamma$,X $\vdash$ reduce$_{\Gamma;x,y.e_1;e_2}$! \\ 
    $\invUNFSymbol$(!< e$_1$, e$_2$>!) &=& !<$\invUNFSymbol$(e$_1$), $\invUNFSymbol$(e$_2$)>! \\
    $\invUNFSymbol$(!e$_1$!$\pcomp$!e$_2$!) &=& !let x=$\invUNFSymbol$(e$_1$) in $\invUNFSymbol$(e$_2$)! TODO: not sure \\
    $\invUNFSymbol$(!e1!$\icomp$!e2!) &=& TODO: \\
    $\invUNFSymbol$(!proj!) &=& TODO \\
    $\invUNFSymbol$(!pair!) &=& TODO \\
    $\invUNFSymbol$(!J!$^T$!var!$_{\Gamma;i}$) &=& !(x1,$\ldots$,x(i-i),xi+z,x(i+1),$\ldots$,xm)! \\
    $\invUNFSymbol$(!J!$^T$!op!$_{\Gamma;n}$) &=& $(x_1',\ldots,'x_{m-n},x_{m-n+1}+\partial_1op_n*z,\ldots,xm+\partial_nop_n*z)$ \\
    && where $\partial_iop_n$ is the $i$-th partial derivative of $op_n$ \\
    $\invUNFSymbol$(!J!$^T$!proj!) &=& TODO \\
    $\invUNFSymbol$(!J!$^T$!pair!) &=& TODO \\
    $\invUNFSymbol$(!J!$^T$!map2!$_{\Gamma;x,y.e}$) &=&  Y((x1,$\ldots$,xm)+$\nabla_{\Gamma}e *$(!reduce! + 0 Z),\\
    && !map2 (fun (a,b) -> a+!$\nabla_{\{x1\}}e *$!b) X1' Z'!, \\
    && !map2 (fun (a,b) -> a+!$\nabla_{\{x2\}}e *$!b) X2' Z'!) \\
    && where $X1'$, $X2'$ are the tangent arrays for the two arrays arguments \\
    && of !map2! and $Z$ the tangent part for the return array of !map2! \\
    $\invUNFSymbol$(!J!$^T$!reduce!$_{\Gamma;x,y.e1;e2}$) &=& !let A = scanl (fun (x,y) -> e1) e2 e3 in! \\
    && !let B = map2 (fun (x,y) ->! $\nabla_{\{x1\}}$!e1(x,y)) A e3 in!\\
    && !let C = map2 (fun (x,y) ->! $\nabla_{\{x2\}}$!e1(x,y)) A e3) in!\\
    && !let D = scanr (fun (x,y) -> x*y) 1 B in!\\
    && Y(x1,$\ldots$,xm, !map3 (fun (a,b,c) -> a+b*c*z) X' D C!)
    \end{tabular}
    \caption{UNF transformation from Target UNF to Target}
    \label{fig:unf_to_target}
    \end{figure*}

\begin{remark}
    One could rightfully wonder: where did reverse-mode really happened here?
    In a way, source UNF does not seem to do much. 
    Then $\Dsynrevsymbol$ is very simple on UNF.
    So perhaps everything happens when going from target UNF to target?
    \MH{finish or remove.}
\end{remark}



\subsection{Efficient representation}

\subsection{Optimisations via partial evaluation}