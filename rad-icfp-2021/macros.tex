\usepackage{amsmath}
\usepackage{mathtools}
\usepackage{suffix}
\usepackage{tikz-cd}
\usepackage{mathpartir}
\usepackage{enumitem}
\usepackage{stmaryrd}
\usepackage[all]{xy}
\usepackage{twoopt}
\usepackage{array}
\usepackage{listings}

\lstdefinelanguage{llql}%
{morekeywords={
  if,then,else,let,in,
  op2,op1,map,map2,foldl,
  scanl,scanr,map3,fun,reduce,shift1,
  op,var,J
  },%
  sensitive,%
  morecomment=[l]//,%
  morecomment=[s]{/*}{*/},%
  morestring=[b]",%
  morestring=[b]',%
  showstringspaces=false,%
  morecomment=[s][\color{gray}]{@}{\ },%
    breaklines=true,%
  mathescape=true,%
showspaces=false,
showtabs=false,
showstringspaces=false,
breakatwhitespace=true,
  aboveskip=1pt,
  belowskip=1pt,
  lineskip=-0.2pt,
   numbersep=5pt,
   numberstyle=\tiny\ttfamily,
   basicstyle=\small\ttfamily,
   keywordstyle=\bfseries\color{blue!70!black},%
   columns=fullflexible,
  frame=single,
  escapeinside={(*@}{@*)}
}[keywords,comments,strings]%

\lstset{language=llql}
\lstMakeShortInline[columns=fixed]!


\newcommand{\system}{RAD\xspace}
\newcommand{\dfsmooth}{$\text{d}\widetilde{\textsc{f}}$\xspace}
\newcommand{\supfull}{\CIRCLE}
\newcommand{\suphalf}{\LEFTcircle}
\newcommand{\supnone}{\Circle}
\newcommand{\supfullstar}{\hspace{1ex}\supfull*}
\newcommand{\notexists}{-}
\newcommand{\RR}{\mathbb{R}}

\usepackage{todonotes}
\newcommand{\TODO}[1]{\todo[inline,author=TODO]{#1}}
\newcommand{\AS}[1]{\todo[inline,author=AS]{#1}}
\newcommand{\as}[1]{\todo[size=\tiny]{as: #1}{}}
\newcommand{\SSS}[1]{\todo[inline,author=SS]{#1}}
\newcommand{\sss}[1]{\todo[size=\tiny]{ss: #1}{}}
\newcommand{\EA}[1]{\todo[inline,author=EA]{#1}}
\newcommand{\ea}[1]{\todo[size=\tiny]{ea: #1}{}}
\newcommand{\MH}[1]{\todo[inline,author=MH]{#1}}
\newcommand{\mh}[1]{\todo[size=\tiny]{mh: #1}{}}

\newcommand{\code}[1]{\texttt{#1}}

\newcommand{\dif}{\mathop{}\!\mathrm{d}}
\newcommand{\Diff}{\mathbf{Diff}}
\newcommand{\sem}[1]{\llbracket #1\rrbracket}
\newcommand{\semgl}[1]{\llparenthesis #1\rrparenthesis}
\newcommand{\defeq}{\stackrel {\mathrm{def}}=}

%-- Syntax --
\makeatletter
\newcommand\@TyAlph[1]{%
\ifcase #1\or \tau\or \sigma\or \rho\else \@ctrerr \fi%
}
\newcommand\ty[1][1]{{\@TyAlph{#1}}}

\newcommand\tvar[1][1]{{\@TyVarAlph{#1}}}
\newcommand\@TyVarAlph[1]{%
\ifcase #1\or \alpha\or \beta\or \gamma\else \@ctrerr \fi%
}

\newcommand\var[1][1]{{\@VarAlph{#1}}}
\newcommand\@VarAlph[1]{%
\ifcase #1\or x\or y\or z\or u\or v\or w\else \@ctrerr \fi%
}

\newcommand\trm[1][1]{{\@TermAlph{#1}}}
\newcommand\@TermAlph[1]{%
\ifcase #1\or t\or s\or r\else \@ctrerr \fi%
}

\newcommand\val[1][1]{%
\ifcase #1\or v\or w\or u\else \@ctrerr \fi%
}

\newcommand\op[1][1]{%
\ifcase #1\or \mathsf{op}\or \mathsf{op}'\or \mathsf{op}''\else \@ctrerr \fi%
}
\makeatother

\newcommand\Dsynsymbol[1][]{\scalebox{0.8}{$\overrightarrow{\mathcal{D}}$}_{#1}}
\newcommand\Dsyn[2][]{\Dsynsymbol[#1](#2)}

\newcommand\Dsynrevsymbol[1][]{\scalebox{0.8}{$\overleftarrow{\mathcal{D}}$}_{#1}}
\newcommand\Dsynrev[2][]{\Dsynrevsymbol[#1](#2)}

\newcommand\letin[3]{\mathbf{let}\,#1=\,#2\,\mathbf{in}\,#3}
\newcommand\SSSynHO{\mathbf{SSSynHO}}
\newcommand\SSSyn{\mathbf{SSSyn}}
\newcommand\TSSyn{\mathbf{TSSyn}}
\newcommand\TTSynHO{\mathbf{TTSynHO}}
\newcommand\TTSyn{\mathbf{TTSyn}}

\newcommand\Syn{\mathbf{Syn}}

\newcommand\IH{\stackrel{I.H.}{=}}

\newtheorem{notation}{Notation}
\newcommand\UNFSymbol{\mathbf{UNF}}
\newcommand\UNF[1]{\UNFSymbol(#1)}
\newcommand\invUNFSymbol{\mathbf{UNF}^{-1}}
\newcommand\invUNF[1]{\UNFSymbol^{-1}(#1)}
\newcommand\tPair[2]{\langle #1, #2\rangle}
\newcommand\inllambda{\widetilde{\lambda}}

\newcommand\bp[1]{\boldsymbol{(}#1\boldsymbol{)}}
\newcommand\ctx{\Gamma}
\newcommand\tinf{\vdash}
\newcommand\Ginf[3][]{\ctx #1\tinf #2 : #3}
\newcommand\subst[2]{#1{}[#2]}
\newcommand\Op{\mathsf{Op}}
\newcommand\cnst{\underline{c}}
\newcommand\sigmoid{\varsigma}
\newcommand\nat{\mathbf{nat}}
\newcommand\reals{\mathbf{R}}
\newcommand\bool{\mathbf{bool}}
\newcommand{\sPair}[2]{( #1, #2 )}
\newcommand{\sTriple}[3]{(#1, #2, #3)}
\newcommand{\sTuple}[1]{(#1)}
\newcommand\tUnit{\bp{}}
\newcommand\tTriple[3]{\langle #1, #2, #3\rangle}
\newcommand\tTuple[1]{\langle #1\rangle}
\newcommand\bProd[2]{\bp{#1 \t* #2}}
\newcommand\tProd[3]{\bp{#1 \t* #2 \t* #3}}

\newcommand\syncat[1]{\mspace{-25mu}\synname{#1}}
\newcommand\synname[1]{\qquad\text{#1}}
\newenvironment{syntax}[1][]{%
\(
  \begin{array}[t]{#1l@{\quad\!\!}*3{l@{}}@{\,}l}
}{
\end{array}
\)%
}

\newcommand\gdefinedby{::=}
\newcommand\gor{\mathrel{\lvert}}

\newcommand\TyDinter[2]{\mathsf{TyD}_{#1}(#2)}
\newcommand\TyD[1]{\mathsf{TyD}(#1)}

\newcommand{\NN}{\mathbb{N}}
\newcommand{\BB}{\mathbb{B}}
\newcommand\ifelse[3]{\mathbf{if}\,#1\,\mathbf{then}\,#2\,\mathbf{else}\,#3\,}
\newcommand\ArrayMake[2]{\vbuildk\,#1\,#2}
\newcommand\ArrayGet[2]{\vgetk\,#1\,#2}
\newcommand\ArrayFold[3]{\vifoldk\,#1\,#2\,#3}
\newcommand\ArrayLength[1]{\vlengthk\,#1}
\newcommand\TensorMake[2]{\boxplus_{i=0}^{#1}#2}
\newcommand\TensorSum[2]{\sum_{i=0}^{#1}#2}
\newcommand\TensorAccess[2]{#1{[}#2{]}}
\newcommand\TensorPred[2]{{[}#1{]}#2}

\newcommand\viteratek{\mathbf{ifold}}
\newcommand\vifoldk{\viteratek}
\newcommand\vbuildk{\mathbf{build}}
\newcommand\vlengthk{\mathbf{length}}
\newcommand\vgetk{\mathbf{get}}
\newcommand\ArraySym{\mathbf{A}}
\newcommand\Array[1]{\ArraySym[#1]}
\newcommand\To{\to}
\newcommand\tMatch[4][\,]{\mathbf{case}\,#2\,\mathbf{of}#1\tTuple{#3}\To#4}
\newcommand{\semu}[1]{\llbracket #1\rrbracket_\mathbf{U}}
\newcommand{\grammarcomment}[1]{\textit{\small #1}}

\newcommand{\Source}{\mathbf{Source}}
\newcommand{\Target}{\mathbf{Target}}