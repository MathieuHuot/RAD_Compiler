\section{Conclusion}

\subsection{Summary} % (fold)
\label{sub:summary}

\subsection{Related Work} % (fold)
\label{sub:related_work}

\noindent \textbf{Automatic Differentiation.} 

\noindent \textbf{Array Languages and Fusion.}

\noindent \textbf{Numerical DSLs.} 

\noindent \textbf{Correctness of AD in functional languages.}

\noindent \textbf{Comparison to other recent papers.}

\subsection{Discussion and future work} % (fold)
\label{sub:discussion_and_future_work}

\noindent \textbf{Other Language features.}
%more primitives, proved correct and more efficient. extend library

\noindent \textbf{Probabilistic Programming.}
%find something simple and precise, o.w. useless section

\noindent \textbf{More dynamic language}
%TF is losing ground to Pytorch, Swift tries to renew with Pytorch-like dynamic features. 

\noindent \textbf{More optimisations.}
%maybe?

\noindent \textbf{Limitations.}
%maybe?

\MH{useful links (Swift is close to our project, and the doc is very readable):}

\begin{verbatim}https://docs.google.com/document/d/1_BirmTqdotglwNTOcYAW-ib6mx_jl-gH9Dbg4WmHZh0/edit# \end{verbatim}
\begin{verbatim}

	https://github.com/apple/swift/blob/master/docs/DifferentiableProgramming.md#approaches-to-automatic-differentiation
\end{verbatim}

\MH{understand related work sentence from Diff. Curry paper: The idea of using closures as back-propagators is receiving recent attention. For example Julia Zy- gote [15] and Swift AD adopts this design. Other recent work follows similar ideas [27, 26] but is using meta-programming as an implementation technique.}

\clearpage