\documentclass[acmsmall,anonymous]{acmart}\settopmatter{printfolios=true,printccs=false,printacmref=false}

%%
%% \BibTeX command to typeset BibTeX logo in the docs
\AtBeginDocument{%
  \providecommand\BibTeX{{%
    \normalfont B\kern-0.5em{\scshape i\kern-0.25em b}\kern-0.8em\TeX}}}


\setcopyright{acmcopyright}
\copyrightyear{2021}
\acmYear{2021}
\bibliographystyle{ACM-Reference-Format}
\citestyle{acmauthoryear} 

\usepackage{booktabs}   %% For formal tables:
                        %% http://ctan.org/pkg/booktabs
\usepackage{subcaption} %% For complex figures with subfigures/subcaptions
                        %% http://ctan.org/pkg/subcaption

\usepackage{wasysym}
\usepackage{xspace}
\usepackage{array,multirow}
\usepackage{adjustbox}

\usepackage{todonotes}
\newcommand{\TODO}[1]{\todo[inline,author=TODO]{#1}}
\newcommand{\AS}[1]{\todo[inline,author=AS]{#1}}
\newcommand{\as}[1]{\todo[size=\tiny]{as: #1}{}}
\newcommand{\SSS}[1]{\todo[inline,author=SS]{#1}}
\newcommand{\sss}[1]{\todo[size=\tiny]{ss: #1}{}}
\newcommand{\EA}[1]{\todo[inline,author=EA]{#1}}
\newcommand{\ea}[1]{\todo[size=\tiny]{ea: #1}{}}
\newcommand{\MH}[1]{\todo[inline,author=MH]{#1}}
\newcommand{\mh}[1]{\todo[size=\tiny]{mh: #1}{}}
\newcommand{\YK}[1]{\todo[inline,author=YK]{#1}}
\newcommand{\yk}[1]{\todo[size=\tiny]{yk: #1}{}}

\usepackage{amsmath}
\usepackage{mathtools}
\usepackage{suffix}
\usepackage{tikz-cd}
\usepackage{mathpartir}
\usepackage{enumitem}
\usepackage{stmaryrd}
\usepackage[all]{xy}
\usepackage{twoopt}
\usepackage{array}
\usepackage{listings}
\usepackage{multirow, bigdelim}

\lstdefinelanguage{llql}%
{morekeywords={
  if,then,else,let,in,
  op2,op1,map,map2,foldl,
  scanl,scanr,map3,fun,reduce,
  shift1L, shift1R,
  op,var,J,
  sum,prod,dot,fst,
  zip,
  true,false,
  OnesLike,
  ZerosLike,
  pair,proj
  },%
  sensitive,%
  morecomment=[l]//,%
  morecomment=[s]{/*}{*/},%
  morestring=[b]",%
  morestring=[b]',%
  showstringspaces=false,%
  morecomment=[s][\color{gray}]{@}{\ },%
    breaklines=true,%
  mathescape=true,%
showspaces=false,
showtabs=false,
showstringspaces=false,
breakatwhitespace=true,
  aboveskip=1pt,
  belowskip=1pt,
  lineskip=-0.2pt,
  numbersep=5pt,
  numberstyle=\tiny\ttfamily,
  basicstyle=\small\ttfamily,
  keywordstyle=\bfseries\color{blue!70!black},%
  columns=fullflexible,
  frame=single,
  escapeinside={(*@}{@*)},
  literate={->}{$\rightarrow\;$}{2}
           {<}{$\langle$}{1}
           {>}{$\rangle$}{1}
}[keywords,comments,strings]%

\lstset{language=llql}
\lstMakeShortInline[columns=fixed]!

% Jacobian
\DeclareMathOperator{\J}{J}
% \newcommand{\J}{\mathop{\mathbf J}}
\newcommand{\system}{RAD\xspace}
\newcommand{\dfsmooth}{$\text{d}\widetilde{\textsc{f}}$\xspace}
\newcommand{\supfull}{\CIRCLE}
\newcommand{\suphalf}{\LEFTcircle}
\newcommand{\supnone}{\Circle}
\newcommand{\supfullstar}{\hspace{1ex}\supfull*}
\newcommand{\notexists}{-}
\newcommand{\RR}{\mathbb{R}}

\newtheorem{notation}{Notation}
\newtheorem{remark}{Remark}

\newcommand{\code}[1]{\texttt{#1}}

\newcommand{\dif}{\mathop{}\!\mathrm{d}}
\newcommand{\Diff}{\mathbf{Diff}}
\newcommand{\sem}[1]{\llbracket #1\rrbracket}
\newcommand{\semgl}[1]{\llparenthesis #1\rrparenthesis}
\newcommand{\defeq}{\stackrel {\mathrm{def}}=}

%-- Syntax --
\makeatletter
\newcommand\@TyAlph[1]{%
\ifcase #1\or \tau\or \sigma\or \rho\else \@ctrerr \fi%
}
\newcommand\ty[1][1]{{\@TyAlph{#1}}}

\newcommand\tvar[1][1]{{\@TyVarAlph{#1}}}
\newcommand\@TyVarAlph[1]{%
\ifcase #1\or \alpha\or \beta\or \gamma\else \@ctrerr \fi%
}

\newcommand\var[1][1]{{\@VarAlph{#1}}}
\newcommand\@VarAlph[1]{%
\ifcase #1\or x\or y\or z\or u\or v\or w\else \@ctrerr \fi%
}

\newcommand\trm[1][1]{{\@TermAlph{#1}}}
\newcommand\@TermAlph[1]{%
\ifcase #1\or t\or s\or r\else \@ctrerr \fi%
}

\newcommand\val[1][1]{%
\ifcase #1\or v\or w\or u\else \@ctrerr \fi%
}

\newcommand\op[1][1]{%
\ifcase #1\or \mathsf{op}\or \mathsf{op}'\or \mathsf{op}''\else \@ctrerr \fi%
}
\makeatother

\newcommand\Dsynsymbol[1][]{\scalebox{0.8}{$\overrightarrow{\mathcal{D}}$}_{#1}}
\newcommand\Dsyn[2][]{\Dsynsymbol[#1](#2)}

\newcommand\Dsynrevsymbol[1][]{{\scalebox{0.8}{$\overleftarrow{\mathcal{D}}$}_{#1}}}
\newcommand\Dsynrev[2][]{\Dsynrevsymbol[#1](#2)}

\newcommand\letin[3]{\mathbf{let}\,#1=\,#2\,\mathbf{in}\,#3}
\newcommand\SSSynHO{\mathbf{SSSynHO}}
\newcommand\SSSyn{\mathbf{SSSyn}}
\newcommand\TSSyn{\mathbf{TSSyn}}
\newcommand\TTSynHO{\mathbf{TTSynHO}}
\newcommand\TTSyn{\mathbf{TTSyn}}

\newcommand\Syn{\mathbf{Syn}}

\newcommand\IH{\stackrel{I.H.}{=}}

\newcommand\UNFSymbol{\mathbf{UNF}}
\newcommand\UNF[1]{\UNFSymbol(#1)}
\newcommand\invUNFSymbol{\mathbf{UNF}^{-1}}
\newcommand\invUNF[1]{\UNFSymbol^{-1}(#1)}
\newcommand\tPair[2]{\langle #1, #2\rangle}
\newcommand\inllambda{\widetilde{\lambda}}

\newcommand\bp[1]{\boldsymbol{(}#1\boldsymbol{)}}
\newcommand\ctx{\Gamma}
\newcommand\tinf{\vdash}
\newcommand\Ginf[3][]{\ctx #1\tinf #2 : #3}
\newcommand\subst[2]{#1{}[#2]}
\newcommand\Op{\mathsf{Op}}
\newcommand\cnst{\underline{c}}
\newcommand\sigmoid{\varsigma}
\newcommand\nat{\mathbf{nat}}
\newcommand\reals{\mathbf{R}}
\newcommand\bool{\mathbf{bool}}
\newcommand{\sPair}[2]{( #1, #2 )}
\newcommand{\sTriple}[3]{(#1, #2, #3)}
\newcommand{\sTuple}[1]{(#1)}
\newcommand\tUnit{\bp{}}
\newcommand\tTriple[3]{\langle #1, #2, #3\rangle}
\newcommand\tTuple[1]{\langle #1\rangle}
\newcommand\bProd[2]{\bp{#1 \t* #2}}
\newcommand\tProd[3]{\bp{#1 \t* #2 \t* #3}}

\newcommand\syncat[1]{\mspace{-25mu}\synname{#1}}
\newcommand\synname[1]{\qquad\text{#1}}
\newenvironment{syntax}[1][]{%
\(
  \begin{array}[t]{#1l@{\quad\!\!}*3{l@{}}@{\,}l}
}{
\end{array}
\)%
}

\newcommand\gdefinedby{::=}
\newcommand\gor{\mathrel{\lvert}}

\newcommand\TyDinter[2]{\mathsf{TyD}_{#1}(#2)}
\newcommand\TyD[1]{\mathsf{TyD}(#1)}

\newcommand{\NN}{\mathbb{N}}
\newcommand{\BB}{\mathbb{B}}
\newcommand\ifelse[3]{\mathbf{if}\,#1\,\mathbf{then}\,#2\,\mathbf{else}\,#3\,}
\newcommand\ArrayMake[2]{\vbuildk\,#1\,#2}
\newcommand\ArrayGet[2]{\vgetk\,#1\,#2}
\newcommand\ArrayFold[3]{\vifoldk\,#1\,#2\,#3}
\newcommand\ArrayLength[1]{\vlengthk\,#1}
\newcommand\TensorMake[2]{\boxplus_{i=0}^{#1}#2}
\newcommand\TensorSum[2]{\sum_{i=0}^{#1}#2}
\newcommand\TensorAccess[2]{#1{[}#2{]}}
\newcommand\TensorPred[2]{{[}#1{]}#2}

\newcommand\viteratek{\mathbf{ifold}}
\newcommand\vifoldk{\viteratek}
\newcommand\vbuildk{\mathbf{build}}
\newcommand\vlengthk{\mathbf{length}}
\newcommand\vgetk{\mathbf{get}}
\newcommand\ArraySym{\mathbf{A}}
\newcommand\Array[2]{\ArraySym[#1]^{#2}}
\newcommand\To{\to}
\newcommand\tMatch[4][\,]{\mathbf{case}\,#2\,\mathbf{of}#1\tTuple{#3}\To#4}
\newcommand{\semu}[1]{\llbracket #1\rrbracket_\mathbf{U}}
\newcommand{\grammarcomment}[1]{\textit{\small #1}}

\newcommand{\Source}{\mathbf{Source}}
\newcommand{\Target}{\mathbf{Target}}

\newcommand\Dtype{\mathbf{D}}

\newcommand{\cost}{Cost}
\newcommand{\transto}{\text{ $\leadsto$ }}
\newcommand{\directD}[3]{\Dsynrevsymbol^{#1}_{#2;#3}}
\newcommand{\grad}{\nabla}
\newcommand{\hcomp}{\widehat{;}}
\newcommand{\pcomp}{\widetilde{;}}
\newcommand{\comp}{;}
\newcommand{\ppcomp}{\times}
\newcommand{\icomp}{\circ}

\newcommand{\SynSource}{SynSource}
\newcommand{\SynTarget}{SynTarget}
\newcommand{\catC}{\mathcal{C}}
\newcommand{\ConcatS}{Concat1}
\newcommand{\ConcatT}{Concat2}
\newcommand{\seml}{\llbracket}
\newcommand{\semr}{\rrbracket}

\newcommand{\concatcomp}{+\kern-1.3ex+\kern0.8ex}

\begin{document}

% \title{Differentiable Programming: correct, efficient and pure}
% \subtitle{Lambda, the ultimate backpropagator, revisited}
\title{Denotationally Correct, Purely Functional, Efficient Reverse-mode Automatic Differentiation}

\author{Mathieu Huot}
\email{mathieu.huot@cs.ox.ac.uk}   
\affiliation{
  \institution{University of Oxford}
  \city{Oxford}
  \country{UK}}

\author{Amir Shaikhha}
\affiliation{
  \institution{University of Edinburgh}
  \city{Edinburgh}
  \country{UK}}

\renewcommand{\shortauthors}{Huot, et al.}

\begin{abstract}
    Reverse-mode differentiation is used for optimization, but it introduces references which break the purity of the underlying program, making them notoriously harder to optimize.
    We present a reverse-mode differentiation on a purely functional language with array operations.  
    It is the first one to deliver a provably efficient, purely functional, and denotationally correct reverse-mode differentiation.
    We show that our transformation is semantically correct and verifies the cheap gradient principle. 
    Inspired by PROPs and compilation to categories, we introduce a novel intermediate representation that we call `unary form'.
    Our reverse-mode transformation is factored as a compilation scheme through this intermediate representation.
    We obtain provably efficient gradients by performing general partial evaluation optimizations after our reverse-mode transformation, as opposed to manually derived ones.
    For simple first-order programs, the obtained output programs resemble static-single-assignment (SSA) code. 
    We emphasize the modularity of our approach and show how our language can easily be enriched with more optimized primitives, as required for some speedups in practice.
\end{abstract}

\keywords{denotational semantics, automatic differentiation, functional programming, optimization}

\maketitle

\section{Introduction}
\label{sec:intro}

% % (Source-to-source) Automatic Differentiation (AD) is a technique for transforming code that implements 
% % a function $f$ into code that computes $f$'s derivative, essentially by using the chain rule for derivatives. 
% % Due to its efficiency and numerical stability, AD is the technique of choice whenever derivatives need to be computed 
% % of functions that are implemented as programs, particularly in high dimensional settings. 
% % Optimizations and Monte Carlo integration algorithms, such as Gradient Descent and Hamiltonian Monte-Carlo methods, 
% % rely crucially on the calculation of derivatives. These algorithms are used in virtually everywhere in machine learning 
% % and computational statistics, and the calculation of derivatives is usually the computational bottle-neck. 
% % AD, roughly speaking, comes in two modes; forward-mode and reverse-mode. 
% % When differentiating a function $\RR^n\to\RR^m$, forward-mode tends to be more efficient if $m \gg n$, while reverse-mode 
% % is generally more performant if $n \gg m$. As most applications reduce to optimization of Monte-Carlo 
% % integration of an objective function $\RR^n\to\RR$ with $n$ very large ($10^4-10^7$), reverse-mode AD is 
% % in many ways the more interesting algorithm.

% % %define diff prog beyond AD, be specific with references/different define-then-run vs Pytorch/Swift etc.
% % %Diff Curry has OK intro

% \subsection{Motivation}
Deep learning is moving towards increasingly sophisticated optimization objectives that employ tensors and operations on tensors.
Reverse-mode Automatic Differentiation (AD) is a technique to automatically compute the gradient of objective functions of form $\RR^n\to\RR$.
Such functions appear a lot in practice: for instance, as loss functions in machine learning.

In order to reach the efficiency of the usual imperative version of reverse-mode, the transformations even in functional languages~\cite{pearlmutter2008reverse} resort to references.
The lack of purity in reverse-mode is known to make it significantly harder to optimize and parallelize. 
None of the current implementations of reverse-mode in a functional setting~\cite{lantern_icfp,pearlmutter2008reverse,baydin2016diffsharp} are pure, and often complicated heuristics with no guarantees are used, e.g.~\cite{xla}.
As a result, to optimize for efficiency, a hand-crafted reverse-derivative must be given, for every important non-trivial operation.
In other words, abstracting away from imperative code is still a hurdle that functional implementations need to overcome.

% An efficient purely functional reverse-mode transformation would be more easily explainable, optimizable, and reusable.

% \subsection{Problem}

% Following \cite{pearlmutter2008reverse}, it is relatively easy to define a purely functional reverse-mode on a first-order language using continuations. 
% Yet, as remarked by the authors, this is highly inefficient. 
% The problem we tackle in this paper is how to go from a simple and general setting, where we can relatively easily prove correctness of reverse-mode, 
% to a representation that leads to more efficient implementations.

In this paper, we define a purely functional (without references or control mechanisms such as state monads), denotationally correct, and provably efficient reverse-mode AD. 
To do so, we define the Unary Normal Form (UNF) representation inspired by PROPs \cite{hackney2015category} and compilation to categories \cite{elliott2017compiling}.
We can easily define and prove correctness of reverse-mode on this representation. 
The whole reverse-mode transformation is obtained by compiling the language to this intermediate representation, applying the simpler reverse-mode transformation, and compiling again to the original language.
After standard optimizations, the output program looks like SSA~\cite{cytron1989efficient} or ANF~\cite{sabry1993reasoning}, which leads to more efficient implementations.

\begin{table}
 \label{fig:comparison-table}
 % \begin{tabular}{|l|c|c|c|c|c|c|c|c|}
 % \hline
 %  & \rot{Reverse Mode}  & \rot{Complexity} & \rot{Pure Derivatives} & \rot{Correctness Proofs} & \rot{Tensor Support}  & \rot{Recursion} & \rot{Conditional} \\
 % \hline
 % \system{} (This Paper) &
 % \supfull & \supfull & \supfull & \supfull  & \supfull  & \suphalf & \supfull \\ 
 % \hline
 % Lantern~\cite{lantern_icfp} & 
 % \supfull & \suphalf & \supnone & \supnone & \supfull & \supfull & \supfull \\ 
 % \hline
 % \dfsmooth{}~\cite{shaikhha2019efficient} 
 % & 
 % \supnone & \supnone & \supfull & \supnone & \supfull  & \suphalf  & \supfull \\ 
 % \hline
 % \cite{huot2020correctness} &
 % \supfull & \supnone & \supfull & \supfull & \supnone & \suphalf & \supfull \\ 
 % \hline
 % \cite{brunel2019backpropagation} &
 % \supfull & \suphalf & \supfull & \supfull & \supnone & \supnone & \supnone \\ 
 % \hline
 % \cite{abadi2019simple} &
 % \supfull & \suphalf & \supfull & \supfull & \supnone & \supfull & \supfull \\ 
 % \hline
 % \cite{barthe2020versatility} &
 % \supnone & \supnone & \supfull & \supfull  & \supnone & \supnone & \supfull \\ 
 % \hline
 % \cite{pearlmutter2008reverse} &
 % \supfull & \supfull  & \supnone & \supnone & \supnone & \supfull & \supfull \\ 
 % \hline
 % \cite{Elliott:2018:SEA:3243631.3236765} &
 % \supfull & \supnone & \supfull & \supfull & \supnone & \supnone & \supnone \\ 
 % \hline
 % \cite{sherman2021} & 
 % \supnone & \supnone & \supfull & \supfull & \supnone & \suphalf & \supfull \\ 
 % \hline
 % \cite{vytiniotis2019differentiable} &
 % \supfull & \suphalf & \supfull & \supnone & \supfull & \supnone & \supnone \\ 
 % \hline
 % \cite{mak2020differential} & 
 % \supfull & \supnone & \supfull & \supfull & \supnone & \supnone & \supnone \\ 
 % \hline
 % \cite{vakar2020reverse} & 
 % \supfull & \supnone & \supfull & \supfull & \supfull & \supnone & \supnone \\ 
 % \hline
 % \cite{Manzyuk2012} & 
 % \supnone & \supnone & \supfull & \supfull & \supnone & \supnone & \supnone \\ 
 % \hline 
 % \cite{cockett2019reverse} &
 % \supfull & \supnone & \supfull & \supfull & \supnone & \supnone & \supnone  \\ 
 % \hline
 % \cite{gallagher-sdg}  & 
 % \supnone & \supnone & \supfull & \supfull & \supnone & \supnone & \supnone  \\ 
 % \hline
 \begin{tabular}{|l|c|c|c|c|c|c|c|c|c|c|c|c|c|c|c|c|}
 \hline
  & \rot{This Paper} & \rot{\cite{lantern_icfp}}  & \rot{\cite{shaikhha2019efficient}}  & \rot{\cite{huot2020correctness}}  & \rot{\cite{brunel2019backpropagation}}  & \rot{\cite{abadi2019simple}}  & \rot{\cite{barthe2020versatility}}  & \rot{\cite{pearlmutter2008reverse}}  & \rot{\cite{Elliott:2018:SEA:3243631.3236765}} & \rot{\cite{sherman2021}}  & \rot{\cite{vytiniotis2019differentiable}}  & \rot{\cite{mak2020differential}}  & \rot{\cite{vakar2020reverse}}  & \rot{\cite{Manzyuk2012}}  & \rot{\cite{cockett2019reverse}}  & \rot{\cite{gallagher-sdg}} \\ \hline
Reverse Mode &
\supfull & \supfull & \supnone & \supfull & \supfull & \supfull & \supnone & \supfull & \supfull & \supnone & \supfull & \supfull & \supfull & \supnone & \supfull & \supnone \\ \hline
Complexity &
\supfull & \suphalf & \supnone & \supnone & \suphalf & \suphalf & \supnone & \supfull & \supnone & \supnone & \suphalf & \supnone & \supnone & \supnone & \supnone & \supnone \\ \hline
Pure Derivatives &
\supfull & \supnone & \supfull & \supfull & \supfull & \supfull & \supfull & \supnone & \supfull & \supfull & \supfull & \supfull & \supfull & \supfull & \supfull & \supfull \\ \hline
Correctness Proofs &
\supfull & \supnone & \supnone & \supfull & \supfull & \supfull & \supfull & \supnone & \supfull & \supfull & \supnone & \supfull & \supfull & \supfull & \supfull & \supfull \\ \hline
Tensor Support &
\supfull & \supfull & \supfull & \supnone & \supnone & \supnone & \supnone & \supnone & \supnone & \supnone & \supfull & \supnone & \supfull & \supnone & \supnone & \supnone \\ \hline
Higher-order Functions &
\supnone & \supfull & \supfull & \supfull & \supfull & \supnone & \supfull & \supfull & \supnone & \supfull & \supfull & \supfull & \supfull & \supfull & \supnone & \supfull \\ \hline
Recursion &
\suphalf & \supfull & \suphalf & \suphalf & \supnone & \supfull & \supnone & \supfull & \supnone & \suphalf & \supnone & \supnone & \supnone & \supnone & \supnone & \supnone \\ \hline
Conditional &
\supfull & \supfull & \supfull & \supfull & \supnone & \supfull & \supfull & \supfull & \supnone & \supfull & \supnone & \supnone & \supnone & \supnone & \supnone & \supnone \\ \hline
 \end{tabular}
 \caption{Comparison of different functional differentiable programming frameworks.
 $\supfull$ means that the property is verified, and $\supnone$ means that it is absent in the work.
 $\suphalf$ for complexity means that the proof is not fully covered, and for recursion, that it does not support general recursion.}
 \label{tbl:relwork}
 \end{table}

\subsection{Examples}

We introduce the general idea of efficient reverse-mode in a functional setting the next examples.
% We recall some basics of reverse-mode differentiation in Section~\ref{sec:background}.

\begin{example}[First-order term]
% On simple first-order terms, the output code is reminiscent of SSA code~\cite{cytron1989efficient}.
Let us consider the term !let w$_1$ = x$_1$ * x$_2$ in let w$_2$ = w$_1$ * x$_1$ in w$_2$! in the context $\Gamma:=\{x_1,x_2, x_3:\RR\}$.\\
After an (inefficient) reverse-mode transformation, we obtain:
\begin{center}
    \begin{tabular}{l}
        !let w$_1$,w$_1'$! = !< x$_1$ * x$_2$, fun (y$_1$,$\ldots$, y$_4$) -> (y$_1$+x$_2$*y$_4$, y$_2$+x$_1$*y$_4$, y$_3$)> in!\\
        !let w$_2$,w$_2'$! = !< w$_1$*x$_1$, fun (y$_1$,$\ldots$, y$_5$) -> w$_1'$!!(y$_1$+w$_1$*y$_5$, y$_2$, y$_3$, y$_4$+x$_1$*y$_5$)> in!\\
        !w$_2'$(0,0,0,0,1)!
    \end{tabular}
\end{center}
The part !(0,0,0,0,1)! comes as an equivalent to initialing the tangent variables in the imperative reverse-mode algorithm. 
After some general partial evaluation techniques that will be detailed further in the paper, we obtain:    

    \begin{center}
            \begin{tabular}{l}
                !let w$_1$ = x$_1$ * x$_2$ in!\\ 
                !let w$_2$ = w$_1$ * x$_1$ in!\\
                !let y$_1$,y$_2$,y$_3$,y$_4$,y$_5$ = 0,0,0,0,1 in!\\
                !let y$_1'$! != y$_1$+w$_1$*y$_5$ in!\\
                !let y$_4'$! != y$_4$+x$_1$*y$_5$ in!\\
                !(y$_1'$+x$_2$*y$_4'$!!, y$_2$+x$_1$*y$_4'$!!, y$_3$)!
            \end{tabular}
    \end{center}   
This is very close to the SSA form \cite{cytron1989efficient} of what the imperative reverse-mode differentiation of our initial term would be.
This term can be further optimized via constant propagation and algebraic simplifications to give
        \begin{center}
            \begin{tabular}{{c}}
                !let w$_1$ = x$_1$ * x$_2$ in!\\ 
                !let w$_2$ = w$_1$ * x$_1$ in!\\
                !(w$_1$+x$_2$*x$_1$, x$_1$*x$_1$, 0)!
            \end{tabular}
        \end{center}
    \end{example}

 \begin{example}[Simple operations on arrays]
    On arrays, three simple operations of interest are the dot-product of two vectors, and the product or sum of the elements of a vector.
    In a functional setting, these can be defined as follows:
\begin{center}
    \begin{tabular}{{r c l}}
        !prod(A)! &:=& !reduce * 1 A! \\
        !sum(A)! &:=& !reduce + 0 A! \\
        !dot(A,B)! &:=& !reduce + 0 (map2 * A B)!     
    \end{tabular}
\end{center}
where !reduce! is a known fold-left operator for which the function argument is associative. 
It is notably faster to execute than a fold-left as it is parallel friendly.

The gradient of each of these expressions w.r.t. !A! will be:
\begin{center}
    \begin{tabular}{{r c l}}
        $\nabla_A$!prod(A)! &:=& !map2 * (scanr * 1 (shift1L A)) (shift1R (scanl * 1 A))! \\
        $\nabla_A$!sum(A)! &:=& !map (x -> 1) A!\\
        $\nabla_A$!dot(A,B)! &:=& !B! 
    \end{tabular}
\end{center}
where:
\begin{itemize}
\item !scanl! is the scan-left operator that returns all the intermediate results of fold-left,
\item !scanr! is the scan-right operator that returns all the intermediate results of fold-right,
\item !shift1L [v$_1$,$\ldots$,v$_n$]! is the shift-left operator and returns ![v$_2$,$\ldots$,v$_{n}$]!, and 
\item !shift1R [v$_1$,$\ldots$,v$_n$]! is the shift-right operator and returns ![v$_1$,$\ldots$,v$_{n-1}$]!.
\end{itemize}
These gradients are some examples among numerous ones which are usually derived by hand, and are here obtained automatically as special cases of our work.
\end{example}   

\[
\begin{tikzcd}
    \Source \ar[rrrr,"\text{efficient }\Dsynrevsymbol (Fig.~\ref{fig:direct_diff_macro})"] \ar[dr,"(Fig.~\ref{fig:source_to_unf})"'] &&&& \Target \arrow[r,loop right,"\text{optim }(Fig.~\ref{fig:optim})"] \\
    & Source\UNFSymbol \ar[rr,"\Dsynrevsymbol (Fig.~\ref{fig:diff_macro})"] && Target\UNFSymbol \ar[ur,"(Fig.~\ref{fig:unf_to_target})"'] & 
\end{tikzcd}
\]

\subsection{Contributions}
We propose a source-code transformation on a simple purely functional language for purely functional reverse-mode differentiation.
Our transformation is comprised of a compilation scheme that is outlined in Figure~\ref{fig:outline}.
We make the following contributions:

\begin{itemize}[leftmargin=*]
\item We present our work with a simple yet expressive array-based language (with constructs such as !map2! and !reduce!) in Section~\ref{sec:simplediff}. 
We show how to directly compute an efficient reverse-mode AD for the expressions of this program (top of Figure~\ref{fig:outline}).  
Furthermore, we show how to extend our work to a richer language in Section~\ref{sec:generalization}.
\item One of the key insights behind efficient reverse-mode AD is to only consider unary operators. 
Inspired by this insight and following IRs such as SSA and ANF, we introduce a novel IR, UNF (Section~\ref{sec:unf})).
We introduce an alternative and easier to follow compilation pipeline for efficient reverse-mode AD (bottom of Figure~\ref{fig:outline}).
\item We prove the complexity guarantees for the reverse ADed programs. Furthermore, we show a list of optimizations that can further improve the constant factors (Section~\ref{sec:complexity}).
\item We proof the correctness of our transformations (top/bottom parts of Figure~\ref{fig:outline}) by defining the denotational semantics of our languages using multicategories and concategories (Section~\ref{sec:correctness}).
\end{itemize}

Next, we recall rudiments of automatic differentiation, forward and reverse-mode differentiation.

% \subsection{Summary of contributions}

% We propose a source-code transformation on a simple purely functional language for purely functional reverse-mode differentiation.
% Our transformation is comprised of a compilation scheme through a novel intermediate representation (IR): unary form (UNF).
% To emphasize the key ideas and for expository purposes, we present our work with a simple yet expressive language in Section~\ref{sec:simplediff}, 
% and show how to extend our work to a richer language in Section~\ref{sec:generalization}.

% We summarize the key ideas of our work. Reverse-mode can be made efficient
% \begin{itemize}
%    \item easily if there are only unary operators
%    \item in a purely functional way
%    \item on expressive constructs such as second-order array operators (e.g. !map2, reduce!) 
%    \item via standard functional optimization techniques
% \end{itemize}

% Our contribution makes these claims precise. In more details, we introduce
% \begin{itemize}
%     \item an expressive source language with array operations, yet restrictive enough for efficient differentiation (\S\ref{sub:sourcelang})
%     \item a method for extending our work and modular approaches to add and prove correct new primitives as needed in practice (\S\ref{sec:generalization})
%     \item a new intermediate representation (IR), UNF, essentially turning every operation into a unary one which inputs and outputs a tuple (\S\ref{sec:unf})
%     \item a transformation between the IR and the main language (\S\ref{sub:transformations to and from UNF})
%     \item a simple reverse-mode transformation on the IR (\S\ref{sub:Simple reverse mode transformation})
%     \item the direct reverse-mode transformation obtained from the source to the target language (\S\ref{sub:Macro for pure reverse mode transformation})
%     \item a denotational semantics of our languages and transformations using multicategories and concategories (\S\ref{sec:correctness})
%     \item correctness of the transformations (\S\ref{sec:correctness})
%     \item complexity guarantees (\S\ref{sec:complexity})
%     \item how efficiency is ensured via standard optimization techniques (\S\ref{sub:Optimizations})
%     \item several additions to the source language (\S\ref{sec:generalization})
% \end{itemize}

% \subsection{Outline of the paper}

% Our main transformation is present on the top of the diagram and presented in Section~\ref{sec:simplediff}. 
% It it decomposed 3 steps which are presented in Section~\ref{sec:unf}.

% \[
\begin{tikzcd}
    \Source \ar[rrrr,"\text{efficient }\Dsynrevsymbol (Fig.~\ref{fig:direct_diff_macro})"] \ar[dr,"(Fig.~\ref{fig:source_to_unf})"'] &&&& \Target \arrow[r,loop right,"\text{optim }(Fig.~\ref{fig:optim})"] \\
    & Source\UNFSymbol \ar[rr,"\Dsynrevsymbol (Fig.~\ref{fig:diff_macro})"] && Target\UNFSymbol \ar[ur,"(Fig.~\ref{fig:unf_to_target})"'] & 
\end{tikzcd}
\]

% In Section~\ref{sec:background} we recall rudiments of automatic differentiation, forward and reverse-mode differentiation.
% Next, in Section~\ref{sec:simplediff} we introduce the source and target languages for differentiation, and our reverse mode transformation. 
% In Section~\ref{sec:unf} we introduce a new intermediate representation UNF, and a simple reverse-mode macro on this representation.
% We introduce transformations from the source language to UNF and from UNF to the target language.
% Next, in Section~\ref{sec:correctness} we give denotational semantics to our languages and show that our decomposition respects the diagram above and that our reverse mode transformation is correct.
% In Section~\ref{sec:complexity}, we show that our transformation verifies the cheap gradient principle and extra general optimizations.
% In Section~\ref{sec:generalization} we show how to adapt our work for a richer source language. 
% Finally, related work, limitations to our approach and future work are presented in Section~\ref{sec:conclusion}. 
\section{Reverse-mode in Automatic Differentiation}
\label{sec:background}

% \subsection{Rudiments of differentiation and dual numbers.}
% Recall that the derivative of a function $f:\RR\to \RR$, if it exists, is a function
% $\nabla f:\RR\to \RR$ such that $\nabla f(x_0)=\frac {\dif f(x)}{\dif x}(x_0)$ is the gradient of $f$ at $x_0$. 

% To find $\nabla f$ in a compositional way, two generalizations are reasonable:
% \begin{itemize}
% \item We need both $f$ and $\nabla f$ when calculating $\nabla (f;g)$
% of a composition $f;g$, using the chain rule, so we are really interested in the pair $(f,\nabla f):\RR\to \RR\times \RR$;
% \item In building $f$ we will need to consider functions of multiple arguments, such as $+:\RR^2\to \RR$, and these functions should propagate derivatives.
% \end{itemize}
% Thus we are more generally interested in transforming a function $g:\RR^n\to \RR$ into a function
% $h:(\RR\times \RR)^n\to \RR\times \RR$ in such a way that for any
% $f_1\dots f_n:\RR\to\RR$, 
% \begin{equation}
%   \label{eqn:dualnumber}
%   (f_1,\nabla f_1,\dots, f_n,\nabla f_n);h
%   =
%   ((f_1,\dots, f_n);g,\nabla ((f_1, \dots, f_n);g))\text.
% \end{equation}

% An intuition for $h$ is often given in terms of dual numbers.
% The transformed function operates on pairs of numbers, $(x,x')$, and it is common
% to think of such a pair as $x+x'\epsilon$ for an `infinitesimal' $\epsilon$.
% But while this is a helpful intuition, the formalization of infinitesimals can be intricate, 
% and the development in this paper is focussed on the elementary formulation in~\eqref{eqn:dualnumber}. 

% The reader may also notice that $h$ encodes all the partial derivatives of
% $g$. For example, 
% if $g \colon \RR^2\to \RR$, then with $f_1(x)\defeq x$ and $f_2(x)\defeq x_2$, by applying \eqref{eqn:dualnumber} to $x_1$ we obtain
% $h(x_1,1,x_2,0)=(g(x_1,x_2), \frac {\partial g(x,x_2)}{\partial x}(x_1))$
% and similarly 
% $h(x_1,0,x_2,1)=(g(x_1,x_2), \frac {\partial g(x_1,x)}{\partial x}(x_2))$.
% And conversely, if $g$ is differentiable in each argument, then
% a unique $h$ satisfying \eqref{eqn:dualnumber} can be found by taking linear
% combinations of partial derivatives:
% \[\textstyle h(x_1,x_1',x_2,x_2')=(g(x_1,x_2),x_1' \cdot\frac {\partial g(x,x_2)}{\partial x}(x_1)+x_2'\cdot \frac {\partial g(x_1,x)}{\partial x}(x_2))\text.\]

% In summary, the idea of differentiation with dual numbers is 
% to transform a differentiable function
% $g:\RR^n\to \RR$ to a function $h:\RR^{2n}\to \RR^2$ which captures~$g$ and all its partial derivatives. We packaged this up in~\eqref{eqn:dualnumber} as a sort-of invariant which is useful for building derivatives of compound functions $\RR\to\RR$ in a compositional way.
% The idea of automatic differentiation is to perform this transformation at the source code level. 

% The two main ways in which it is performed in practice is by operator overloading \cite{} or by source code transformation. We focus our work on the latter.

% \subsection{Reverse-mode AD}

% AD comes in two main flavors: forward mode and reverse mode. 
% Forward mode computes a directional derivative of the Jacobian while reverse-mode computes a directional derivative of the transpose Jaobian. 
% This has a significant impact in practice as this means reverse mode can compute the whole gradient of a function $f:\RR^n\to \RR$ in one pass, whereas forward mode needs $n$ passes. 
% Assuming that both have about the same complexity, this can have a huge impact in practice. 
% Such functions are expremely commonly seen as loss functions in the context of machine learning for instance.

% Reverse-mode computes the derivatives backward, using the symmetry of the chain rule.
% This means the compute flow is reverted after the first pass to compute the value of the function and its intermediate values. 
% These values appear in the chain rule so should be stored or recomputed.
% These things make reverse-mode harder to implement efficiently, and especially harder to analyze and optimize. 
% Traditional implementations use references.
% For instance, let 
%  \begin{align*}
%      \trm\defeq &\letin{w_1}{\var_1 * \var_2}{\\&\letin{w_2}{w_1 * \var_1}{\\&w_2}}
%  \end{align*}
%  The imperative version of the reverse differentiation of $\trm$ is
% 	\begin{align*}
% 		&w_1=\var_1 * \var_2 \\
% 		&w_2=w_1 * \var_1 \\
%         &\var_1':=0~;~\var_2':=0 \\
%         &w_1':=0 \\
% 		&w_2':=1 \\
% 		&w_1'+= \var_1 * w_2'; \var_1'+= w_1*w_2' \\
% 		&\var_1'+= \var_2 * w_1'; \var_2'+=\var_1 * w_1'\\
% 		&\text{return }(\var_1',\var_2')  
% 	\end{align*}
% The tangent part for every variable from the context $\var_i'$ and every intermediate variable $w_i'$ is initialized at $0$, 
% except for the return variable $w_2$ for which  $w_2':=1$. 

% As noted in \cite{pearlmutter2008reverse}, in a functional language, one can have a purely functional reverse-mode using continuations. 
% This representation is notoriously inefficient, and the existing implementations of reverse mode in a functional setting have followed the lines of the imperative version above \cite{pearlmutter2008reverse, wang2018demystifying, baydin2016diffsharp}.

% In the next section, we first introduce a very simple first-order language. 
% Then we define reverse-mode on this language as a program transformation. 
% This first transformation is quite inefficient and we how we can use standard optimizations on functional languages to obtain a provably efficient pure program.
% The key idea behind that will be generalized and explored in detail in the rest of the paper is that of turning every operator into a unary one.     
% The remaining of the paper will generalize this idea and explore some consequences of this new approach. 

\subsection{Rudiments of forward-mode AD and dual numbers}

Recall that the derivative of a function $f:\RR\to \RR$, if it exists, is a function
$\nabla f:\RR\to \RR$ such that $\nabla f(x_0)=\frac {\dif f(x)}{\dif x}(x_0)$ is the gradient of $f$ at $x_0$. 

To find $\nabla f$ in a compositional way, two generalizations are reasonable:
\begin{itemize}
\item We need both $f$ and $\nabla f$ when calculating $\nabla (f;g)$
of a composition $f;g$, using the chain rule, so we are really interested in the pair $(f,\nabla f):\RR\to \RR\times \RR$;
\item In building $f$ we will need to consider functions of multiple arguments, such as $+:\RR^2\to \RR$, and these functions should propagate derivatives.
\end{itemize}
Thus we are more generally interested in transforming a function $g:\RR^n\to \RR$ into a function
$h:(\RR\times \RR)^n\to \RR\times \RR$ in such a way that for any
$f_1\dots f_n:\RR\to\RR$, 
\begin{equation}
  \label{eqn:dualnumber}
  (f_1,\nabla f_1,\dots, f_n,\nabla f_n);h
  =
  ((f_1,\dots, f_n);g,\nabla ((f_1, \dots, f_n);g))\text.
\end{equation}

An intuition for $h$ is often given in terms of dual numbers.
The transformed function operates on pairs of numbers, $(x,x')$, and it is common
to think of such a pair as $x+x'\epsilon$ for an `infinitesimal' $\epsilon$.
But while this is a helpful intuition, the formalization of infinitesimals can be intricate, 
and the development in this paper is focussed on the elementary formulation in~\eqref{eqn:dualnumber}. 

The reader may also notice that $h$ encodes all the partial derivatives of
$g$. For example, 
if $g \colon \RR^2\to \RR$, then with $f_1(x)\defeq x$ and $f_2(x)\defeq x_2$, by applying \eqref{eqn:dualnumber} to $x_1$ we obtain
$h(x_1,1,x_2,0)=(g(x_1,x_2), \frac {\partial g(x,x_2)}{\partial x}(x_1))$
and similarly 
$h(x_1,0,x_2,1)=(g(x_1,x_2), \frac {\partial g(x_1,x)}{\partial x}(x_2))$.
And conversely, if $g$ is differentiable in each argument, then
a unique $h$ satisfying \eqref{eqn:dualnumber} can be found by taking linear
combinations of partial derivatives:
\[\textstyle h(x_1,x_1',x_2,x_2')=(g(x_1,x_2),x_1' \cdot\frac {\partial g(x,x_2)}{\partial x}(x_1)+x_2'\cdot \frac {\partial g(x_1,x)}{\partial x}(x_2))\text.\]

In summary, the idea of differentiation with dual numbers is 
to transform a differentiable function
$g:\RR^n\to \RR$ to a function $h:\RR^{2n}\to \RR^2$ which captures~$g$ and all its partial derivatives. We packaged this up in~\eqref{eqn:dualnumber} as a sort-of invariant which is useful for building derivatives of compound functions $\RR\to\RR$ in a compositional way.
The idea of automatic differentiation is to perform this transformation at the source code level. 

The two main ways in which it is performed in practice is by operator overloading or by source code transformation (see e.g. \cite{griewank2008evaluating} Chapter 6). 
Our work focus on a compiled version so we focus on source code transformation which is more fitted in this case.

\subsection{Reverse-mode differentiation}

The problem with the previous approach shows up when one wants to compute the full gradient of a function $\RR^n\to\RR$, for a large $n$. 
Forward-mode only computes one directional derivative, for instance one partial derivative. This implies $n$ passes must be performed through the forward derivative to compute the whole gradient.
By using the symmetry in the chain rule, there is a way to compute the whole gradient faster, and this method is reverse-mode differentiation.
Suppose given a function $f=f_n\circ...\circ f_1 :\RR^n\to\RR$. 
The simplest way to see this mathematically, is first to say that forward mode essentially computes $(Jf)v=Jf_n(Jf_{n-1}(...(Jf_1v))...)$ for a direction $v\in\RR^n$. 
Reverse-mode, on the other hand, computes $(Jf)^Tv =J^Tf_1(J^Tf_{2}(...(J^Tf_nv))...)$ for a vector $v\in\RR$.
In particular, taking $v=1$ computes the gradient of $f$.

Because the compute flow of the function is reversed, the actual implementation of reverse-mode is way more tricky. 
%%TODO: from Matthijs' paper, need to rewrite.
Reverse-mode AD is only well-understood as a source-code transformation (also called define-then-run
style AD) on limited programming languages. Typically, its implementations
on more expressive languages that have features such as higher-order functions and conditionals
make use of define-by-run approaches. These approaches first build a computation graph during runtime, effectively evaluating the program until a straight-line
first-order program is left, and then they evaluate this new program \cite{carpenter2015stan,paszke2017automatic}. 
Such approaches have the severe downside that the differentiated code cannot benefit from existing optimizing compiler architectures. As such, these AD libraries
need to be implemented using carefully, manually optimized code, that for example does not contain any common subexpressions. This implementation process
is precarious and labour intensive. Further, some whole-program optimizations
that a compiler would detect go entirely unused in such systems.

\subsection{Purely functional inefficient reverse-mode}

Following \cite{pearlmutter2008reverse}, there is a simple way to define an inefficient yet purely functional reverse-mode transformation for first-order programs.
We review a slight modification of the transformation we presented, which is also better explained on an example. 

Let the term we want to differentiate be $x1:\RR,...,xn:\RR\vdash exp(cos(xi))$.
Following the chain rule, we need the jacobian matrices of $cos$ at $xi$ and of $exp$ at $cos(xi)$. 
Instead of considering these as operations from $\RR\to\RR$, we consider then as functions from the whole context. So $cos$ and $exp$ are seen as functions $\RR^n\to\RR$.
If we just do that though, we lose compositionality. 
So $cos$ will also return its context and is now seen as a function $\sem{cos}:\RR^n\to\RR^{n+1}$.
similarly for $exp$, but also taking the return value of $\sem{cos}$ as an extra argument, the one it will actually use and not simply return. 
We thus obtain $\sem{exp}:\RR^{n+1}\to\RR^{n+2}$. Now their jacobians matrices $J\sem{cos} \in Mat_{n,n+1}$, $J\sem{exp} \in Mat_{n+1,n+2}$ also compose nicely.
The same can be done for binary (or more) operators and let bindings (assuming no nested lets). 
This transforms a first-order program to a function $f:\RR^n\to\RR^{n+m}$ of the form $f_m\circ...\circ f_1$. 
If the original program was of type $\RR$, then the return value of the original program is the last compononent of $f$.
Following the mathematical presentation of reverse-mode above, the gradient of the the original program is then obtained as 
\[\nabla f= J^Tf(0,...,0,1)=J^Tf_1(J^Tf_{2}(...(J^Tf_m(0,...,0,1)))...)\]

To actually reverse the order of computation needed for this transpose of jacobians, we use a sort of simple continuation.
$f_i$ is turned into $\Dsynrevsymbol{f_i}:=<f_i, \lambda Y. Y\circ J^Tf_i>$ where $Y:\RR^{n+i-1}\to \RR^n$. 
We recover compositionality by noting that $<f_{i+1}(f_i), (\lambda Y. Y\circ J^Tf_{i+1}>)(\lambda Y. Y\circ J^Tf_i)>$ reduces to
$<f_{i+1}\circ f_i, \lambda Y. Y\circ J^Tf_i \circ J^Tf_{i+1}>$, and thus by induction we can obtain $<f, \lambda Y. YJ^Tf>$.
By applying the identity continuation $\RR^n\to\RR^n$ on the second component then the result to $(0,...,0,1)$, 
we have obtained a purely functional to compute $\nabla f$. 

This method is quite inefficient though. First, we have to carry a continuation and $\beta$-reduce a lot of higher-order functions.
Second, each $J^Tf_{i+1}$ is a potentially huge matrix if $n$ or $m$ is big.

The imperative version of reverse mode for a binary operator $op(x,y)$ is something like $x'+= \partial_1op(x,y);y'+= \partial_2op(x,y)$. 
It is for these cases that mutation is usually key. 
If we see out term as a directed graph, reverse mode needs to backpropagate from the end of the graph to the starting nodes via every path.
It's hard to keep track of all these information in parallel in a functional way.
One of the key simple ideas that we used was to transform every operator into a unary one. 
This sssentially trivialises the compute flow to a line. 
Even if the starting program was a straight line program, 
having non unary operators was a source of inefficiency which justified mutation in the first place.
By returning every variable every time, the problem of using a variable several time does not need to be dealt with via mutation. 
This simple idea of transforming a program into essentially a straight line is what our new intermediate representation unary-form does. 

\subsection{Insights for efficient reverse-mode}

If we look at $J^Tf_{i+1}$, we notice that this function is almost the identity, except at the last row. 
Even on the last row, if the original term was a unary or binary operator like $cos, exp, +, *$, 
the row is zero except for at most 2 indices (1 for unary operators).
This means we can use a more compact representation $J^T\sem{op(xi,xj)}:=\lambda (y1,...,y(n+i)).(y1,...,y(n+i))+[i]\partial_1op(xi,xj)+[j]\partial_2op(xi,xj)$, 
where $[k]$ means that the element is added at the $k$-th index of the tuple.

We know in advance that such jacobian functions are going to be applied one to another, so we can use partial evaluation to $\beta$-reduce all of these $\lambda$s.
Because each function is almost the identity, we obtain a lot of substitutions of the form $[x/y]$ where both $x$ and $y$ are variables. 
This allows us to drastically reduce the size of $J^Tf$. In fact, for simple programs this is basically enough to get an efficient purely functional reverse derivative transformation.
We develop this idea further for a richer language.
 
\section{Simple pure reverse-mode differentiation}
\label{sec:simplediff}

\subsection{Source Language}
\label{sub:sourcelang}

\label{sub:sourcelang}

We consider a standard call-by-value language. 
It consists of a first-order functional language with arrays and a few typical second-order array operations. 
The types !T1,T2! and terms !e1,e2! are given in Figure~\ref{fig:source_grammar}.
We have included a minimal set of array operations for the sake of illustration,  it is not hard to add more.
See Section~\ref{sec:generalization}.

\begin{figure*}[t]
\setlength{\tabcolsep}{0.3em}
\centering
\begin{tabular}{|l c l|l|}
\hline
\multicolumn{3}{|c|}{\textbf{Core Grammar}} & \multicolumn{1}{c|}{\textbf{Description}}\\\hline
!T! & \mbox{::=} & $\reals$ & \grammarcomment{Real Type} \\
& $\mid$ & !T! $\times$ !T! & \grammarcomment{Product Type}\\
& $\mid$ & $\Array{\reals}$ & \grammarcomment{Real Array Type}\\
\hline
!e! & \mbox{::=} & !x! & \grammarcomment{Variable}\\
& $\mid$ & !c! & \grammarcomment{Real constant}\\
& $\mid$ & !let x = e in e! & \grammarcomment{Variable Binding}\\
& $\mid$ & !< e, e >! $\mid$ $\pi_1$(!e) $\mid$ $\pi_2$(!e) & \grammarcomment{Pair Constructor/Destructor}\\
& $\mid$ & !e op2 e! $\mid$ !op1 e! & \grammarcomment{Binary/Unary operations}\\
& $\mid$ & !map (x.e) e! $\mid$ !map2 (x,y.e) e e! & \grammarcomment{Array map and map2}\\
& $\mid$ & !foldl (x,y.e) e e! & \grammarcomment{Array fold left}\\
\hline
\end{tabular}
\vspace{-0.2cm}
\caption{Grammar of the source language.}
\label{fig:source_grammar}
\end{figure*}


The typing rules are given in Figure~\ref{fig:source_typesystem}. 
For scalar operations, we assume given a set of operations, including $+$ and $*$. 
!op1! and !op2! denote respectively a unary and a binary operation on reals. 
These operations represent smooth total functions, 
but again this can be easily generalized (\S\ref{sec:generalization}).  
Typical examples include !cos, exp, +, *!. 
We use infix notation for binary operators.

!reduce! is a fold left operator for which the function is assumed to be associative, 
and the provided initial value should be a unit of the binary operation.
It is a well-known parallel friendly construct. For the sake of simplicity in the presentation, the bound function in !reduce! 
is restricted to having no free variables. For the same reason, we currently restrict to arrays of reals.
We show how to lift these restrictions in Section~\ref{sec:generalization}.

\begin{figure*}[tb]
    \centering
    \begin{tabular}{c} 
    \\\hline
    $\Gamma \vdash$ !x!: !T!
    \end{tabular}(!x!: !T!$\in\Gamma$)
    \hspace{0.5cm}
    \begin{tabular}{c} 
        \\\hline
        $\Gamma \vdash$ !c!: $\reals$
    \end{tabular}
    \hspace{0.5cm}
    \begin{tabular}{c}
    $\Gamma \vdash$ !e$_1$!: !T$_1$! $\quad$ $\Gamma \vdash$ !e$_2$!: !T$_2$! \\\hline  
    $\Gamma \vdash$ !<e$_1$,e$_2$>!: !T$_1$! $\times$ !T$_2$!
    \end{tabular}
    \hspace{0.5cm}
    \begin{tabular}{c}
        $\Gamma \vdash$ !e!: !T$_1$! $\times$ !T$_2$! \\\hline  
        $\Gamma \vdash$ $\pi_i$!e!: !T$_i$!
    \end{tabular}($i\in\{1,2\}$)

    \begin{tabular}{c}
    $\Gamma \vdash$ !e$_1$!: !T$_1$! $\quad$ $\Gamma$, !x!: !T$_1$! $\vdash$ !e$_2$!: !T$_2$! \\\hline
    !let x = e$_1$ in e$_2$!: !T$_2$!
    \end{tabular}
    \hspace{0.5cm}
    \begin{tabular}{c}
        $\Gamma \vdash$ !e!: $\reals$ \\\hline  
        $\Gamma \vdash$ !op1 e!: $\reals$
    \end{tabular}
    \hspace{0.5cm}
    \begin{tabular}{c}
        $\Gamma \vdash$ !e$_1$!: $\reals$ $\quad$ $\Gamma \vdash$ !e$_2$!: $\reals$ \\\hline  
        $\Gamma \vdash$ !e$_1$ op2 e$_2$!: $\reals$
        \end{tabular}
 
    \begin{tabular}{c}
        $\Gamma$, !x!: $\reals$, !y!: $\reals$ $\vdash$ !e$_1$!: $\reals$ 
        $\quad$ $\Gamma$ $\vdash$ !e$_2$!: $\Array{\reals}$
        $\quad$ $\Gamma$ $\vdash$ !e$_3$!: $\Array{\reals}$
        \\\hline  
        $\Gamma \vdash$ !map2 (x,y.e$_1$) e$_2$ e$_3$!: $\Array{\reals}$
    \end{tabular}

    \begin{tabular}{c}
        !x!: $\reals$, !y!: $\reals$ $\vdash$ !e$_1$!: $\reals$ 
        $\quad$ $\Gamma$ $\vdash$ !e$_2$!: $\reals$
        $\quad$ $\Gamma$ $\vdash$ !e$_3$!: $\Array{\reals}$
        \\\hline  
        $\Gamma \vdash$ !reduce (x,y.e$_1$) e$_2$ e$_3$!: $\reals$
    \end{tabular}
    \vspace{-0.2cm}
    \caption{Type system of the source language}
    \vspace{-0.4cm}
    \label{fig:source_typesystem}
    \end{figure*}

\subsection{Target Language}

The target language of our source-code transformation is an extension to the source language.
It is a higher-order language, as our purely functional reverse-mode introduces a continuation.
The set of scalar operations should also be closed under partial differentiation. 
In more detail, for every unary scalar operation !op1!, 
we assumed given an operator $\partial_1$!op1! whose semantics should be the derivative of !op1!.
For every binary operator !op2!, we assume given operators $\partial_1$!op2!, $\partial_2$!op2! 
respectively representing the first and second partial derivative of !op2!.

Similarly, the target language contains more array primitives which are used to define the reverse derivatives of array operations. 
Scan left !scanl! is similar to fold left but also stores all the intermediate results in an array, which it returns.
In the same vein, scan right !scanr! performs a fold left by reading the array from right to left and stores 
the intermediate results in an array from right to left.

Finally, we add two new shift operators !shift1L! and !shift1R!. 
They take an array of size $n$, and respectively forget the first and the last element of the array.
These somewhat ad-hoc operators naturally show up when differentiating fold-like operators.

The types and terms are presented in Figure~\ref{fig:target_grammar}.
Our lambda abstractions take $n$ arguments as we are not interested here with partial applications. 
In fact, the lambda abstractions introduced by reverse-mode will be removed during partial evaluation, 
and the notation with lambda abstractions having $n$ bound variables makes reading slightly easier.
We note that we don't actually need the full power of higher-order because we only use lambda abstractions over variables of ground types
and let expressions binding such lambda abstractions. We only need the target language to be second-order.
The type system for the extended grammar of the target language is presented in Figure~\ref{fig:target_typesystem}. 

\begin{figure*}[t]
    \setlength{\tabcolsep}{0.3em}
    \centering
    \begin{tabular}{|l c l|l|}
    \hline
    \multicolumn{3}{|c|}{\textbf{Core Grammar}} & \multicolumn{1}{c|}{\textbf{Description}}\\\hline
    !T! & \mbox{::=} & $\ldots$ & \grammarcomment{Same as Source} \\
    & $\mid$ & !T->T! & \grammarcomment{Function Type}\\ 
    \hline
    !e! & \mbox{::=} & $\ldots$ & \grammarcomment{Same as Source}\\
    & $\mid$ & !fun (x$_1$,$\ldots$,x$_n$) -> e! & \grammarcomment{Lambda Abstraction}\\
    & $\mid$ & !e(e$_1\ldots$e$_n$)! & \grammarcomment{Function Application}\\
    & $\mid$ & !scanl (x,y.e) e e! $\mid$ !scanr (x,y.e) e e! & \grammarcomment{Array scan left and right}\\
    & $\mid$ & !shift1L e! $\mid$ !shift1R e! & \grammarcomment{Array left/right shifting}\\
    \hline
    \end{tabular}\\ \vspace{0.2cm}
    \begin{tabular}{|c|}
    \hline
    \begin{tabular}{c}
        $\Gamma$, !x!: $\reals$, !y!: $\reals$ $\vdash$ !e$_1$!: $\reals$ 
        $\quad$ $\Gamma$ $\vdash$ !e$_2$!: $\reals$
        $\quad$ $\Gamma$ $\vdash$ !e$_3$!: $\Array{\reals}{n}$
        \\\hline  
        $\Gamma \vdash$ !scanl (x,y.e$_1$) e$_2$ e$_3$!: $\Array{\reals}{n+1}$
    \end{tabular}
\\
    \begin{tabular}{c}
        $\Gamma$, !x!: $\reals$, !y!: $\reals$ $\vdash$ !e$_1$!: $\reals$ 
        $\quad$ $\Gamma$ $\vdash$ !e$_2$!: $\reals$
        $\quad$ $\Gamma$ $\vdash$ !e$_3$!: $\Array{\reals}{n}$
        \\\hline  
        $\Gamma \vdash$ !scanr (x,y.e$_1$) e$_2$ e$_3$!: $\Array{\reals}{n+1}$
    \end{tabular}
\\
    \begin{tabular}{c}
        $\Gamma$, !x$_1$!: !G$_1$!, $\ldots$, !xn!: !G$_n$! $\vdash$ !e!: !T! 
        \\\hline  
        $\Gamma \vdash$ !fun (x$_1$,$\ldots$,x$_n$) -> e!: !G$_1\times\ldots\times$G$_n$->T!
    \end{tabular}
\\
    \begin{tabular}{c}
        $\Gamma$ $\vdash$ !e!: !G$_1\times\ldots\times$G$_n$ -> T!
        $\quad$ $\Gamma$ $\vdash$ !ei!: !Gi! for all $1\leq i\leq n$
        \\\hline  
        $\Gamma \vdash$ !e(e$_1\ldots$e$_n$)!: !T!
    \end{tabular}
\\
    \begin{tabular}{c}
        $\Gamma$ $\vdash$ !e!: $\Array{\reals}{n+1}$
        \\\hline  
        $\Gamma \vdash$ !shift1L e!: $\Array{\reals}{n}$
    \end{tabular}
    \hspace{0.5cm}
    \begin{tabular}{c}
        $\Gamma$ $\vdash$ !e!: $\Array{\reals}{0}$
        \\\hline  
        $\Gamma \vdash$ !shift1L e!: $\Array{\reals}{0}$
    \end{tabular}
\\
    \begin{tabular}{c}
        $\Gamma$ $\vdash$ !e!: $\Array{\reals}{n+1}$
        \\\hline  
        $\Gamma \vdash$ !shift1R e!: $\Array{\reals}{n}$
    \end{tabular}
    \hspace{0.5cm}
    \begin{tabular}{c}
        $\Gamma$ $\vdash$ !e!: $\Array{\reals}{0}$
        \\\hline  
        $\Gamma \vdash$ !shift1R e!: $\Array{\reals}{0}$
    \end{tabular} \\ \hline
    \end{tabular}
    \vspace{-0.4cm}
    \caption{Grammar and type system of the target language.}
    \label{fig:target_grammar}
    \vspace{-0.4cm}
    \end{figure*}
    

    

\begin{figure*}[tb]
    \centering

    \begin{tabular}{c}
        $\Gamma$, !x!: $\reals$, !y!: $\reals$ $\vdash$ !e1!: $\reals$ 
        $\quad$ $\Gamma$ $\vdash$ !e2!: $\reals$
        $\quad$ $\Gamma$ $\vdash$ !e3!: $\Array{\reals}$
        \\\hline  
        $\Gamma \vdash$ !reduce (x,y.e1) e2 e3!: $\reals$
    \end{tabular}

    \begin{tabular}{c}
        $\Gamma$, !x!: $\reals$, !y!: $\reals$ $\vdash$ !e1!: $\reals$ 
        $\quad$ $\Gamma$ $\vdash$ !e2!: $\reals$
        $\quad$ $\Gamma$ $\vdash$ !e3!: $\Array{\reals}$
        \\\hline  
        $\Gamma \vdash$ !scanl (x,y.e1) e2 e3!: $\Array{\reals}$
    \end{tabular}

    \begin{tabular}{c}
        $\Gamma$, !x!: $\reals$, !y!: $\reals$ $\vdash$ !e1!: $\reals$ 
        $\quad$ $\Gamma$ $\vdash$ !e2!: $\reals$
        $\quad$ $\Gamma$ $\vdash$ !e3!: $\Array{\reals}$
        \\\hline  
        $\Gamma \vdash$ !scanr (x,y.e1) e2 e3!: $\Array{\reals}$
    \end{tabular}

    \begin{tabular}{c}
        $\Gamma$, !x!: $\reals$, !y!: $\reals$ $\vdash$ !e1!: $\reals$ 
        $\quad$ $\Gamma$ $\vdash$ !e2!: $\Array{\reals}$
        $\quad$ $\Gamma$ $\vdash$ !e3!: $\Array{\reals}$
        $\quad$ $\Gamma$ $\vdash$ !e4!: $\Array{\reals}$
        \\\hline  
        $\Gamma \vdash$ !map3 (x,y.e1) e2 e3 e4!: $\Array{\reals}$
    \end{tabular}

    \begin{tabular}{c}
        $\Gamma$, !x1!: !G1!, $\ldots$, !xn!: !Gn! $\vdash$ !e!: !T! 
        \\\hline  
        $\Gamma \vdash$ !fun (x1,...,xn) -> e!: !G1x...xGn->T!
    \end{tabular}

    \begin{tabular}{c}
        $\Gamma$ $\vdash$ !e!: !G1x...xGn -> T!
        $\quad$ $\Gamma$ $\vdash$ !ei!: !Gi! for all $1\leq i\leq n$
        \\\hline  
        $\Gamma \vdash$ !ee1...en!: !T!
    \end{tabular}

    \vspace{-0.2cm}
    \caption{Type system of the target language}
    \vspace{-0.4cm}
    \label{fig:target_typesystem}
    \end{figure*}    

\subsection{Macro for pure reverse mode transformation} % (fold)
\label{sub:Macro for pure reverse mode transformation}

In Figure~\ref{fig:direct_diff_macro} we present our direct transformation from the source language to the target language for pure reverse mode differentiation.
We explain in the next Section (\ref{sub:Partial evaluation and optimization}) how to optimize it further and compute the gradient.
Admittedly, this transformation may be hard to read, and it is not straightforward to show its correctness directly. 
In Section~\ref{sec:unf} we decompose this transformation into three simpler steps via a novel intermediate representation (UNF).

\begin{example}
    The reverse-mode transformation of the terms from the introduction are given by

    \begin{tabular}{c l}
        &$\directD{\rho}{\Gamma}{Y}$(!let w$_1$ = x$_1$ * x$_2$ in let w$_2$ = w$_1$ * x$_1$ in w$_2$!) \\
        =& !let w$_1$,Y$_1$=!\\
        & \quad\quad !let y$_{11}$,Y$_{11}$= <x$_1$, fun (y$_1$,y$_2$,y$_3$,z) -> Y(y$_1$+z,y$_2$,y$_3$)> in! \\
        & \quad\quad !let y$_{12}$,Y$_{12}$= <x$_{2}$, fun (y$_{1}$,y$_{2}$,y$_{3}$,y$_{4}$,z) -> Y$_{11}$(y$_{1}$,y$_{2}$+z,y$_{3}$,y$_{4}$)> in! \\
        & \quad\quad !<y$_{11}$ * y$_{12}$, fun (y$_{1}$,y$_{2}$,y$_{3}$,z) -> Y$_{12}$(y$_{1}$,y$_{2}$,y$_{3}$,y$_{12}$*z,y$_{11}$*z) > in! \\
        & !let w$_{2}$,Y$_{2}$=!\\
        & \quad\quad !let y$_{21}$,Y$_{21}$= <w$_{1}$, fun (y$_{1}$,y$_{2}$,y$_{3}$,y$_{4}$,z) -> Y$_1$(y$_{1}$,y$_{2}$,y$_{3}$,y$_{4}$+z)> in! \\
        & \quad\quad !let y$_{22}$,Y$_{22}$= <x$_{1}$, fun (y$_{1}$,y$_{2}$,y$_{3}$,y$_{4}$,y$_{5}$,z) -> Y$_{21}$(y$_{1}$+z,y$_{2}$,y$_{3}$,y$_{4}$,y$_{5}$)> in! \\
        & \quad\quad !<y$_{21}$ * y$_{22}$, fun (y$_{1}$,y$_{2}$,y$_{3}$,y$_{4}$,z) -> Y$_{22}$(y$_{1}$,y$_{2}$,y$_{3}$,y$_{4}$,y$_{22}$*z,y$_{21}$*z) > in! \\
        & !let y,Y$_{3}$= <w, fun (y$_{1}$,y$_{2}$,y$_{3}$,y$_{4}$,z) -> Y$_{2}$(y$_{1}$,y$_{2}$,y$_{3}$,y$_{4}$+z)> in! \\
        & !<y, fun (y$_{1}$,y$_{2}$,y$_{3}$,z) -> Y$_{3}$(y$_{1}$,y$_{2}$,y$_{3}$,0,z) >!
    \end{tabular}
    \medskip

    \begin{tabular}{c l}
        &$\directD{\rho}{\Gamma}{Y}$(!prod(A)!) \\
        =& !let y,Y$_{1}$= <1, fun (X,z) -> Y(X)> in! \\
        & !let B,Y$_{2}$= <A, fun (X,x,Z) -> Y$_{1}$(X+Z,x)> in!\\
        & !let A$_{0}$= shift1R (scanl * y B) in!\\
        & !let A$_{1}$= shift1L (map2 (a,b.b) A$_{0}$ B) in!\\
        & !let A$_{2}$= map2 (a,b.a) A$_{0}$ B in!\\
        & !let A$_{3}$= scanr * 1 A$_{1}$ in!\\
        & !<prod(B), fun (X,z) -> Y$_{2}$(X,0,map2 (a,b. a*z*b) A$_{2}$ A$_{3}$)>! 
    \end{tabular}
\end{example}

The idea is that $\rho$ represents the return type of the derivative part, which should be !A$_{1} \times \ldots \times$ A$_n$! 
if we want the whole gradient of a term !e! in context !$\Gamma = \;$x$_{1}$:A$_{1}$,$\ldots$,x$_n$:A$_n$!.
The subscript $\Gamma$ denotes the current context, 
which is locally augmented, for instance when differentiating a !let! rule.
For non-unary operations, we differentiate the arguments from the left to the right and add their derivatives to the current stack, 
which is modeled by the continuation $Y$. 
Importantly for performance, each continuation variable $Y$ is only used once.

We now introduce several notations which are useful when defining the transformation for reverse-mode.

Ground types are defined inductively by 
$$G::= \RR \mid G\times \ldots \times G \mid A[G]$$

$(\RR,+,\underline{0})$ forms a monoid and this monoid structure extends canonically 
to a monoid structure $(G,\widehat{+},0_G)$ for every ground type $G$. 
It is defined inductively on $G$ as follows

\begin{tabular}{l c l}
    $0_\RR$  & $\defeq$ & $\underline{0}$ \\
    $0_{G_1\times \ldots \times G_n}$ & $\defeq$ &  $< O_{G_1},\ldots, 0_{G_n} >$ \\
    $0_{\Array{\reals}{n}}$& $\defeq$ & !ZerosLike(n)! \\
    $a\widehat{+}_\RR b$ & $\defeq$ & $ a+b$ \\
    $(a_1,\ldots,a_n)\widehat{+}_{G_1\times\ldots\times G_n}(b_1,\ldots,b_n)$ & $\defeq$ & $(a_1\widehat{+}_{G_1}b_1,\ldots,a_n\widehat{+}_{G_n}b_n)$ \\
    $A\widehat{+}_{A[\RR]}B $ & $\defeq$ & !map2! + $A$ $B$ 
\end{tabular}

A ground context is a context only containing variables of ground type.
The previous monoid structure again extends canonically on ground contexts $\Gamma$ by defining
$0_{x1:G1,\ldots,x_n:G_n}\defeq 0_{G_1},\ldots,0_{G_n}$ and 
$a_1,\ldots,a_n\widehat{+}_{x1:G_1,\ldots, x_n:G_n}b_1,\ldots,b_n\defeq a_1\widehat{+}_{G_1}b_1,\ldots,a_n\widehat{+}_{G_n}b_n$.

\begin{notation}
We introduce more notation in the table below.\\

\begin{tabular}{|l c l|}
    \hline
    !let x$_{1}$=e$_{1}$,$\ldots$,xn=en!  & \multirow{2}{*}{=} & !let x$_{1}$=e$_{1}$ in let x$_{2}$=e$_{2}$ in $\ldots$! \\
    !in e! && !let x$_n$ = e$_n$ in e!\\ \hline
    $Id_\Gamma$ \quad\quad\quad ($\Gamma \, = \, x_1:A_1,\ldots,x_n:A_n)$ & = & !fun! $(y_1:A_1,\ldots,y_n:A_n)$! -> !$(y_1,\ldots,y_n)$ \\ \hline
    $\grad_\Gamma e$ \quad\quad\hspace{0.6em}($\Gamma\vdash e:\RR$) & = & $\pi_2\directD{\Gamma}{\Gamma}{Id_\Gamma}(e)(0_\Gamma,\underline{1})$ \\ \hline
    !pos(x)! \quad(!x!$\in\Gamma=x_:A_1,\ldots,x_n:A_n)$ & = & position $i$ of !x! in $\Gamma$ \\ \hline
    ![i]e! \quad\quad(!e! of ground type $G_i$) & \multirow{2}{*}{=} &  $(0_{G_1},\ldots,0_{G_{i-1}},e,0_{G_{i+1}},\ldots,0_{G_n})$ \\
    && ($G_j$ are ground types) \\ \hline
    $\grad_{\Gamma_1}$(!e!) \quad\quad($\Gamma=\Gamma_1,\Gamma_2$, $|\Gamma_1|=k$)& = & $(e_1,\ldots,e_k)$ \\
    $\grad_{\Gamma_2}$(!e!) & = & $(e_{k+1},\ldots,e_n)$ \\
    && when $\grad_{\Gamma}($!e!$) = (e_1,\ldots,e_n)$ \\ \hline
    !ZerosLike(A)! & = & !map (x.0) A! \\ \hline
    !OnesLike(A)! & = & !map (x.1) A! \\ \hline
    !ZerosLike(n)! & = & !map (x.0) (x:\Array{\reals}{n})! \\ \hline
    !OnesLike(n)! & = & !map (x.1) (x:\Array{\reals}{n})! \\ \hline
\end{tabular}
TODO: I don't have map in the language, super annoying.
\end{notation}



We have the following typing lemma for $\directD{\rho}{\Gamma}{Y}$.
\begin{lemma}[Typing $\directD{\rho}{\Gamma}{Y}$]
    If $\Gamma \vdash$ !e: A!, then $\Gamma$!,Y: !$\Gamma\to \rho \vdash \directD{\rho}{\Gamma}{Y}$!(e): !$\directD{\rho}{\Gamma}{Y}$!(A)!.
\end{lemma}

\begin{figure*}[t]
    % \begin{tabular}{r c l}
    %     $\directD{\rho}{\Gamma}{Y}$(!A!) &=& !A $\times$ ($\Gamma$ $\times$ A->$\rho$)!\\
    % \end{tabular}
    % \medskip
    \small
    \begin{tabular}{|r c l|}
    \hline
        $\directD{\rho}{\Gamma}{Y}$(!A!) &=& !A $\times$ ($\Gamma$ $\times$ A->$\rho$)!\\ & & \\
        $\directD{\rho}{\Gamma}{Y}$(!c!) &=& 
            !<c, fun (x$_{1}$,$\ldots$,x$_n$,z) ->! \\
            && !Y(x$_{1}$,$\ldots$,x$_{n}$) >!\\
        $\directD{\rho}{\Gamma}{Y}$(!x!) &=& 
            !<x, fun (x$_{1}$,$\ldots$,x$_n$,z) ->! \\
            && !Y((x$_{1}$,$\ldots$,x$_n$)!$\widehat{+}$![pos(x)]z) >!\\
        $\directD{\rho}{\Gamma}{Y}$!let x:A = e$_{1}$ in e$_{2}$!) &=& 
            !let x,Y$_{1}$ = !$\directD{\rho}{\Gamma}{Y}$!(e$_{1}$) in! \\
            &&!let y,Y$_{2}$ = !$\directD{\rho}{\Gamma,x:A}{Y_1}$!(e$_{2}$) in!\\ 
            &&!<y, fun (x$_{1}$,$\ldots$,x$_n$,z) -> Y$_{2}$(x$_{1}$,$\ldots$,x$_n$,!$0_{A}$!,z)>!\\
        $\directD{\rho}{\Gamma}{Y}$(!< e$_{1}$, e$_{2}$ >!) &=&
            !let y$_{1}$,Y$_{1}$ = !$\directD{\rho}{\Gamma}{Y}$!(e$_{1}$) in! \\
            &&!let y$_{2}$,Y$_{2}$ = !$\directD{\rho}{\Gamma,x_1}{Y_1}$!(e$_{2}$) in!\\
            &&!< <y$_{1}$,y$_{2}$>, fun (x$_{1}$,$\ldots$,x$_n$,z) -> !\\
            &&!Y(x$_{1}$,$\ldots$,x$_n$,!$\pi_1$!(z),!$\pi_2$!(z)) >!\\ 
        $\directD{\rho}{\Gamma}{Y}$($\pi_1$(!e!:AxB)) &=&
            !let x,Y$_{1}$ = !$\directD{\rho}{\Gamma}{Y}$!(e) in! \\
            && !<!$\pi_1$!x, fun (x$_{1}$,$\ldots$,x$_n$,z) -> Y(x$_{1}$,$\ldots$,x$_n$,(z,!$0_B$!))>! \\
        $\directD{\rho}{\Gamma}{Y}$($\pi_2$(!e!:AxB)) &=&
            !let x,Y$_{1}$ = !$\directD{\rho}{\Gamma}{Y}$!(e) in! \\
            && !<!$\pi_2$!x, fun (x$_{1}$,$\ldots$,x$_n$,z) -> Y(x$_{1}$,$\ldots$,x$_n$,(!$0_A$,z!))>! \\
        $\directD{\rho}{\Gamma}{Y}$(!op1 e!) &=&  
            !let x,Y$_{1}$ = !$\directD{\rho}{\Gamma}{Y}$!(e) in! \\
            && !<op1 x, fun (x$_{1}$,$\ldots$,x$_n$,z) -> ! \\
            && !Y(x$_{1}$,$\ldots$,x$_n$,!$\partial$!op1(x)*z) >! \\
        $\directD{\rho}{\Gamma}{Y}$(!e$_{1}$ op2 e$_{2}$!) &=& 
            !let x$_{1}$,Y$_{1}$ = !$\directD{\rho}{\Gamma}{Y}$!(e$_{1}$) in! \\
            && !let x$_{2}$,Y$_{2}$ = !$\directD{\rho}{\Gamma,x_1}{Y_1}$!(e$_{2}$) in! \\
            && !<x$_{1}$ op2 x$_{2}$, fun (x$_{1}$,$\ldots$,x$_n$,z) ->! \\
            && !Y$_{2}$(x$_{1}$,$\ldots$,x$_n$,!$\partial_1$!op2(x$_{1}$,x$_{2}$)*z,!$\partial_2$!op2(x$_{1}$,x$_{2}$)*z>! \\
        $\directD{\rho}{\Gamma}{Y}$(!map2 (x,y.e$_{1}$) e$_{2}$ e$_{3}$!) &=&  
            !let A,Y$_{1}$ = !$\directD{\rho}{\Gamma}{Y}$!(e$_{2}$) in! \\
            && !let B,Y$_{2}$ = !$\directD{\rho}{\Gamma,A}{Y_1}$!(e$_{3}$) in! \\
            && !let G=!$\grad_{\Gamma}$!e$_{1}$ in!\\
            && !<map2 (x,y.e$_{1}$) A B, fun (x$_{1}$,$\ldots$,x$_n$,Z) -> !\\
            && !Y$_{2}$( (x$_{1}$,$\ldots$,x$_n$)!$\widehat{+}$!(G*(reduce + 0 Z)),!\\
            && \quad\quad! map2 (a,b.(!$\grad_{\{x\}}$!e$_{1}$)[a/x]*b) A Z,!\\
            && \quad\quad! map2 (a,b.(!$\grad_{\{y\}}$!e$_{1}$)[a/x]*b) B Z )>!\\
        $\directD{\rho}{\Gamma}{Y}$(!reduce (x,y.e$_{1}$) e$_{2}$ e$_{3}$!) &=&
            !let y$_{1}$,Y$_{1}$ = !$\directD{\rho}{\Gamma}{Y}$!(e$_{2}$) in! \\
            && !let A,Y$_{2}$ = !$\directD{\rho}{\Gamma,y_1}{Y_1}$!(e$_{3}$) in! \\
            && !let A$_{0}$=shift1R (scanl (x,y.e$_{1}$) y$_{1}$ A) in! \\
            && !let A$_{1}$=shift1L (map2! \\
            && !   (a,b.(!$\grad_{\{x\}}$!e$_{1}$)[a/x,b/y]) A$_{0}$ A) in! \\
            && !let A$_{2}$=map2 (a,b.(!$\grad_{\{y\}}$!)e$_{1}$[a/x,b/y]) A$_{0}$ A in! \\
            && !let A$_{3}$=scanr * 1 A$_{1}$ in! \\
            && !<reduce (x,y.e$_{1}$) y$_{1}$ A, fun (x$_{1}$,$\ldots$,x$_n$,z) ->! \\
            && !Y$_{2}$(x$_{1}$,$\ldots$,x$_n$, map2 (x,y. x*y*z) A$_{2}$ A$_{3}$)>! \\ \hline
        \end{tabular}
    \caption{Reverse-mode transformation from source to target language}
    \label{fig:direct_diff_macro}    
\end{figure*}

\subsection{Partial evaluation and optimization} % (fold)
\label{sub:Partial evaluation and optimization}

Given a term $\Gamma\vdash e : \reals$, we can compute its gradient $\grad_\Gamma e$ from a particular instance of 
$\directD{\rho}{\Gamma}{Y}(e)$. First, $\rho, Y$ specifies if we want to compute the whole gradient regarding the variables from $\Gamma$ or a subset of it.
For a subset $\rho\subset \Gamma$, one chooses $Y$ to be the projection function sending a variable 
$x_i:G$ of $\Gamma$ to $x_i$ if it belongs to $\rho$ and to $0_G$ otherwise.
In particular, we take $Y=Id_\Gamma$ to compute the whole gradient.
Next, the gradient will be given by the second part of the pair $\directD{\rho}{\Gamma}{Y}(e)$, 
and we need to initialize the tangent variables. All of them are set to $0$ except the one corresponding to the output value of !e!, 
which we initialize at $1$ to run the backpropagation. 
All in all, we compute the gradient via $\pi_2\directD{\rho}{\Gamma}{Id_\Gamma}(e)(0_\Gamma,1)$.

Following the insight from Section~\ref{subsec:insights}, 
we don't want to keep all the lambda abstractions as this is costly. 
The transformation is designed in such a way that all the lambda abstractions are given arguments,  
and we can use partial evaluation to do beta-reduce all these lambda abstractions.

By inspection in Figure~\ref{fig:direct_diff_macro}, 
we see that there is a linear usage of  each continuation $Y$ 
and that it is always applied to almost the identity. 
More precisely, we have applications of the form !fun (x1,$\ldots$,xn) -> Y(e1,$\ldots$,en)! 
where the !ei! are !xi! except for at most $k$ (independant of $n$) terms.
In fact we have $k=2$, except for the !map2! case which requires a more detailed analysis (see Section~\ref{sec:complexity}).

Using inlining as given in Figure~\ref{fig:optim}, this allows us to rewrite

\begin{tabular}{l}
!fun (x$_{1}$,$\ldots$,x$_n$) -> (fun (y$_{1}$,$\ldots$,y$_n$) -> (f$_{1}$,$\ldots$,f$_n$))(e$_{1}$,$\ldots$,e$_n$)! 
$\transto$ \\
!fun (x$_{1}$,$\ldots$,x$_n$) -> let y$_{1}$,$\ldots$,y$_n$ = e$_1$,$\ldots$,e$_n$ in (f$_{1}$,$\ldots$,f$_n$)!
\end{tabular}

Using the invariant above, most of the !e!$_{i}$ are variables. 
Without loss of generality, assume the onlt terms !e!$_{i}$ which are not variables are !e!$_{n-1}$ and !e!$_{n}$.
we can use forward subsitution on the other !e!$_{i}$ and we obtain

\begin{tabular}{l}
    !fun (x$_{1}$,$\ldots$,x$_n$) -> let y$_{1}$,$\ldots$,y$_n$ = e$_1$,$\ldots$,e$_n$ in (f$_{1}$,$\ldots$,f$_n$)!  $\transto$ \\
    !fun (x$_{1}$,$\ldots$,x$_n$) -> let y$_{n-1}$,y$_n$ =e$_{n-1},$e$_1$ in (f$_{1}$[x$_{1}$/y$_{1}$],$\ldots$,f$_{n-2}[$x$_{n-2}$/y$_{n-2}]$,f$_{n-1}$,f$_{n}$)!
\end{tabular}
It means that it does not change the evaluation cost of most of the !f!$_{i}$ for $1\leq i \leq n-2$.
The new evaluation cost is reduced to the sum of the cost of evaluating 
the !f!$_{i}$ in addition to the cost of evaluating !e!$_{n-1}$ and !e!$_{n}$, O($n$) movement of variables.

\begin{example}
    After forward-substitution and inlining the inner !Y!$_{i}$, the gradient of the terms from the introduction reduces to

    \begin{tabular}{c l}
        & $\grad_\Gamma$(!let w$_{1}$ = x$_{1}$ * x$_{2}$ in let w$_{2}$ = w$_{1}$ * x$_{1}$ in w$_{2}$!) \\
        =& !let w$_1$,Y$_1$= <x$_1$ * x$_{2}$, fun (y$_{1}$,y$_{2}$,y$_{3}$,z) -> Y(y$_{1}$+x$_2$*z,y$_{2}$+x$_1$*z,y$_{3}$) > in! \\
        & !let w$_{2}$,Y$_{2}$= <w$_{1}$ * x$_{1}$, fun (y$_{1}$,y$_{2}$,y$_{3}$,y$_{4}$,z) -> Y$_{1}$(y$_{1}$+w$_1$*z,y$_{2}$,y$_{3}$,y$_{4}$+x$_{1}$*z) > in! \\
        & !let y,Y$_{3}$= <w$_{2}$, fun (y$_{1}$,y$_{2}$,y$_{3}$,y$_{4}$,z) -> Y$_{2}$(y$_{1}$,y$_{2}$,y$_{3}$,y$_{4}$+z)> in! \\
        & !<y, fun (y$_{1}$,y$_{2}$,y$_{3}$,z) -> Y$_{3}$(y$_{1}$,y$_{2}$,y$_{3}$,0,z) >!
    \end{tabular}

After another simplification step, we obtain 

\begin{tabular}{c l}
    & $\grad_\Gamma$(!let w$_{1}$ = x$_{1}$ * x$_{2}$ in let w$_{2}$ = w$_{1}$ * x$_{1}$ in w$_{2}$!) \\
    =& !let w$_1$= x$_1$ * x$_{2}$ in let w$_{2}$= w$_{1}$ * x$_{1}$ in! \\
        & !<w$_{2}$, fun (y$_{1}$,y$_{2}$,y$_{3}$,z) -> let y'$_{1}$=y$_{1}$+w$_1$*z in !\\
        & \quad\quad\quad!let z'=y$_4$+x$_1$*z in Y(y'+x$_2$*z',y$_{2}$+x$_1$*z',y$_{3}$) >!
\end{tabular}

Similarly, for the gradient of !prod(A)! we obtain

\begin{tabular}{c l}
        & $\grad_\Gamma$(!prod(A)!) \\
        & !let A$_{0}$= shift1R (scanl * 1 A) in!\\
        & !let A$_{1}$= shift1L (map2 (a,b.b) A$_{0}$ A) in!\\
        & !let A$_{2}$= map2 (a,b.a) A$_{0}$ A in!\\
        & !let A$_{3}$= scanr * 1 A$_{1}$ in!\\
        & !<prod(A), fun (X,z) -> Y(X+map2 (a,b. a*z*b) A$_{2}$ A$_{3}$)>! 
    \end{tabular}
\end{example}

Though this transformation after the inlining above already has the right complexity, 
its purity allows us to make the most out of generic optimizations.  
A list of common optimizations that are useful for this language can be found in Figure~\ref{fig:optim}.

\begin{example}
As shown in appendix \ref{sub:gradintro}, the optimizations from Figure~\ref{fig:optim} 
are sufficient to show that the gradients of the terms of the introduction reduce to the following.

\begin{tabular}{{r c l}}
    $\nabla_A$!prod(A)! &=& !map2 * (scanr * 1 (shift1L A)) (shift1R (scanl * 1 A))!\\
    $\nabla_A$!sum(A)! &=& !map (x -> 1) A!\\
    $\nabla_A$!dot(A,B)! &=& !B! 
\end{tabular}
\end{example}
\section{UNF}
\label{sec:unf}

\subsection{Source UNF} % (fold)
\label{sub:Source UNF}

Following the intuition highlighted in Section~\ref{subsec:insights}, we present a new language which we call unary form (UNF). 
It simply consists of a composition of unary operators. That said, we want to compile our source language to this intermediate representation, 
and we need to remember some information about the initial term. 

The grammar of our source UNF is given in Figure~\ref{fig:unf_source_grammar}. 

\begin{figure*}[t]
    \setlength{\tabcolsep}{0.3em}
    \centering
    \begin{tabular}{|l c l|l|}
    \hline
    \multicolumn{3}{|c|}{\textbf{Core Grammar}} & \multicolumn{1}{c|}{\textbf{Description}}\\\hline
    !T! & \mbox{::=} & ![A1,$\ldots$,An]! & \grammarcomment{Lists of types from source} \\
    \hline
    !e! & \mbox{::=} & !var!$_{T;i}$ & \grammarcomment{Variable}\\
    & $\mid$ & !op!$_{T;n}$ & \grammarcomment{Operations, for $0\leq n\leq 2$}\\
    & $\mid$ & !pair!$_{T;A\times B}$ & \grammarcomment{Pairing a pair of variables}\\
    & $\mid$ & !proj!$_{T_1;T_2;T_3}$  & \grammarcomment{Projection}\\
    & $\mid$ & !e!$\comp$!e! & \grammarcomment{Sequential composition}\\
    & $\mid$ & !map2!$_{T;x,y.e}$ & \grammarcomment{Map2}\\
    & $\mid$ & !reduce!$_{T;x,y.e;e}$ & \grammarcomment{Reduce}\\
    \hline
    \end{tabular}
    \vspace{-0.2cm}
    \caption{Grammar of the source UNF}
    \label{fig:unf_source_grammar}
    \end{figure*}

There are a few notable things in this syntax. 
Every term is indexed by a well-defined contex $\Gamma$ from the source language.
In addition, a variable has a integer index $1\leq i\leq n$ where $n$ is the size of the context $\Gamma$.
Every constant, unary, or binary operator is summarised as an $n$-ary operator !op!$_n$.
Sequential composition is denoted by !;! and !e1;e2! means that !e1! should be performed, and then !e2!.
Every array operator !map!, !map2! and !foldl! has an extra index which represents a well-formed term in the source language.

The types are the same as in the source language. 
A context $\Gamma=\{$!x1: T1,...,xn:Tn!$\}$ can always be seen as a type !T1!$\times$...$\times$!Tn!.
UNF is very syntactic and does not really use variables. 
In fact it consists only of constants and composition. 
For this reason, we denote contexts as types in the indices of the terms, and $\times$ is used for context extension.
But we do emphasize that we mean contexts and the actual variables are remembered and
this will be key when compiling to the target language after differentiation. 

The typing rules are detailed in Figure~\ref{fig:source_unf_typesystem}.
By $\reals^{\times n}$ we mean the product $\reals\times...\times\reals$ of $n$ factors $\reals$.

\MH{need to explain that return values produce some kind of ANF and introduce new vars!}

\begin{figure*}[tb]
    \centering
    \begin{tabular}{c} 
    \\\hline
    $\Gamma \vdash$ !var!$_{\Gamma,i}$: $\Gamma\times$!Ti!
    \end{tabular}($\Gamma=$!T1!$\times\ldots\times$!Tn!)
    \hspace{0.5cm}
    \begin{tabular}{c}
        \\\hline
        $\Gamma\times\reals^{\times n} \vdash$ !op!$_{\Gamma,n}$ : $\Gamma\times\reals^{\times(n+1)}$
    \end{tabular}

    \begin{tabular}{c}
    !T0! $\vdash$ !e1!: !T1! $\quad$ !T1! $\vdash$ !e2!: !T2! \\\hline
    !T0! $\vdash$ !e1; e2!: !T2!
    \end{tabular}
    \hspace{0.5cm}
    \begin{tabular}{c}
        \\\hline  
        $\Gamma \times \Array{\reals} \vdash$ !map!$_{\Gamma, x.e}$: $\Gamma \times \Array{\reals} \times \Array{\reals}$
    \end{tabular}

    \begin{tabular}{c}
        \\\hline  
        $\Gamma \times \Array{\reals} \times \Array{\reals} \vdash$ !map2!$_{\Gamma, x,y.e}$: $\Gamma \times \Array{\reals} \times \Array{\reals} \times \Array{\reals}$
    \end{tabular}

    \begin{tabular}{c}
        \\\hline  
        $\Gamma \times \Array{\reals} \vdash$ !foldl!$_{\Gamma, x,y.e1, e2}$: $\Gamma \times \Array{\reals} \times \reals$
    \end{tabular}
    \vspace{-0.2cm}
    \caption{Type system of the Source UNF}
    \vspace{-0.4cm}
    \label{fig:source_unf_typesystem}
    \end{figure*}

% subsubsectionSource UNF (end)

\subsection{Target UNF} % (fold)
\label{sub:Target UNF}

Now our program in UNF is in an equivalent form to straight-line programs. 
Before presenting our differentiation macro, we present the target language for this macro. 
We call this intermediate representation target UNF.
The main idea is similar to source UNF where we only have constants and sequential composition.
That said, the key difference is that now every constant represents a pair of an operation and what is meant to represent its Jacobian.
Following the insights from Section~\ref{sec:background}, the composition of pairs reverses the order on the second component. T
his represents the fact that reverse-mode reverts the computation flow, 
and that pre-composition (using a continuation) is the simplest way to do this in a purely functional setting.
On pairs, !;! still represents sequential composition for the first component, and reversed composition is represented by $\circ$ on the second component.
The grammar and types are given in Figure~\ref{fig:unf_target_grammar}. 
For every constant $C$ from the source UNF, we have a new constant $J^TC$ representing its transpose Jacobian.
One thing to note is that one can only form pairs in this language. 
Similarly to the source UNF, sequential composition is the only operation which is not a constant of the language.

\begin{figure*}[t]
    \setlength{\tabcolsep}{0.3em}
    \centering
    \begin{tabular}{|l c l|l|}
    \hline
    \multicolumn{3}{|c|}{\textbf{Core Grammar}} & \multicolumn{1}{c|}{\textbf{Description}}\\\hline
    !T! & \mbox{::=} & ![A1,...,An]! & \grammarcomment{Lists of types from target} \\
    \hline
    !e! & \mbox{::=} & ... & \grammarcomment{Same as source UNF}\\
    & $\mid$ & !J!$^T$!var!$_{T;i}$ & \grammarcomment{Jacobian for variable}\\
    & $\mid$ & !J!$^T$!op!$_{T;n}$ & \grammarcomment{Jacobian for operation, $0\leq n\leq 2$}\\
    & $\mid$ & !J!$^T$!pair!$_{T;A\times B}$ & \grammarcomment{Jacobian for pairing}\\
    & $\mid$ & !J!$^T$!proj!$_{T1;T2;T3}$ & \grammarcomment{Jacobian for projection}\\
    & $\mid$ & !J!$^T$!map2!$_{T;x,y.e}$ & \grammarcomment{Jacobian for map2}\\
    & $\mid$ & !J!$^T$!reduce!$_{T;x,y.e;e}$ & \grammarcomment{Jacobian for reduce}\\
    & $\mid$ & !<e, e>! & \grammarcomment{Term pairing}\\
    & $\mid$ & $\icomp$ & \grammarcomment{Internal function composition}\\
    \hline
    \end{tabular}
    \vspace{-0.2cm}
    \caption{Grammar of the target UNF}
    \label{fig:unf_target_grammar}
\end{figure*}

Technically, all our terms in target UNF should have an extra index $\rho$ which is a type from Source.
This will be the return type of the continuation for the jacobian accumulation. 
In other words, if we want the full gradient of a term $\Gamma\vdash$!e!: $\reals$, $\rho$ should be chosen to be $\Gamma$.
For every type !T!, let $\Dtype$(!T!)$:=$!T!$\times$(!T!->$\rho$). 
The typesystem is given in Figure~\ref{fig:target_unf_typesystem}. 

\begin{figure*}[tb]
    \centering
    \begin{tabular}{c} 
        \\\hline
        !T,A$_i$! $\vdash$ !J!$^T$!var!$_{T;i}$: !T!
        \end{tabular}~(!T=A$_1$,$\ldots$,A$_n$!)
        \hspace{0.5cm}
        \begin{tabular}{c}
            \\\hline
            !T,!$\reals^{\times(n+1)} \vdash$ !J!$^T$!op!$_{T;n}$ : !T,!$\reals^{\times(n)}$
        \end{tabular}~(!T=A$_1$,$\ldots$,A$_n$!)
    
        \begin{tabular}{c}
            \\\hline
            !T,A$\times$B! $\vdash$ !J!$^T$!pair!$_{T;A\times B}$ : !T,A,B!
        \end{tabular}~(!T=A$_1$,$\ldots$,A$_n$!)
    
        \begin{tabular}{c}
            \\\hline
            !T$_1$,T$_3$! $\vdash$ !J!$^T$!proj!$_{T_1;T_2;T_3}$ : !T$_1$,T$_2$,T$_3$!
        \end{tabular}
    
        \begin{tabular}{c}
            !x$_1$:A$_1$,$\ldots$,x$_n$:A$_n$,x:!$\reals$!,y:!$\reals$ $\vdash$ !e!: $\reals$ \quad in Source Language
            \\\hline  
            !T,!$\Array{\reals}{n},\Array{\reals}{n},\Array{\reals}{n} \vdash$ !J!$^T$!map2!$_{T; x,y.e}$: !T,!$\Array{\reals}{n}$,$\Array{\reals}{n}$
        \end{tabular}~(!T=A$_1$,$\ldots$,A$_n$!)
    
        \begin{tabular}{c}
            !x:!$\reals$!,y:!$\reals$ $\vdash$ !e$_1$!: $\reals$ \quad !x$_1$:A$_1$,$\ldots$,x$_n$:A$_n$! $\vdash$ !e$_2$!:$\reals$ \quad in Source Language
            \\\hline  
            !T,!$\Array{\reals}{n},\reals \vdash$ !J!$^T$!reduce!$_{T; x,y.e_1; e_2}$: !T,!$\Array{\reals}{n}$
        \end{tabular}~(!T=A$_1$,$\ldots$,A$_n$!)

        \begin{tabular}{c}
            !T! $\vdash$ !e$_1$: T$_1$!  \quad !T! $\vdash$ !e$_1$: T$_2$!
            \\ \hline
            !T! $\vdash$ !<e$_1$, e$_2$>: T$_1$,T$_2$!
        \end{tabular}
        \hspace{0.5cm}
        \begin{tabular}{c}
            !T1! $\vdash$ e$_1$: [\underline{T$_3$} -> B]  \quad !T$_2$ $\vdash$ e$_2$: T$_3$! 
            \\ \hline
            !T1! $\vdash$ !e$_1$! $\icomp$ !e$_2$!:[\underline{T$_2$}! -> B!]
        \end{tabular}

    \vspace{-0.2cm}
    \caption{Type system of the Target UNF}
    \vspace{-0.4cm}
    \label{fig:target_unf_typesystem}
    \end{figure*}


\subsection{Simple reverse mode transformation} % (fold)
\label{sub:Simple reverse mode transformation}

We are now able to present a simple transformation for purely functional reverse-mode on UNF.
This transformation is somehow straightforward as some of the work was put into the design of UNF.
Our main takeaway for this paper is that reverse-mode is complicated because there is a lot happening at the same time. 
But by decoupling the problems, it becomes much easier. UNF is somehow dealing with the inherent type-dependency of reverse-mode.
Then the macro $\Dsynrevsymbol$ adds Jacobians. It allows to compute gradients in a compositional way.
Finally, we'll see in the next section how to compile back to a more standard language and optimize our transformation to get efficiency.

The differentiation macro is given in Figure~\ref{fig:diff_macro}.
One might be surprised that the macro does not seem to do much, and it is partly correct.
We tried to decompose reverse-mode into more elementary small steps, and each step is slightly non trivial and dealing with one aspect of reverse-mode.
The perhaps slightly more involved transformation is the one coming next,  going from target UNF to our target language.

\begin{figure*}[t]
\begin{tabular}{|c|}
\hline
    \begin{tabular}{r c l}
    $\Dsynrevsymbol_{\rho}$(!A$_1$,$\ldots$,A$_n$!) &=& !A$_1$,$\ldots$,A$_n$,(A$_1\times\ldots\times$A$_n$)->$\rho$!\\ \\
    $\Dsynrevsymbol_{\rho}$(!var!$_{T;i}$) &=& $F$(!var$_{T;i}$, J$^T$var$_{T;i}$!) \\
    $\Dsynrevsymbol_{\rho}$(!op!$_{\Gamma;n}$) &=& $F$(!op$_{T;n}$,J$^T$op$_{T;n}$!) \\ 
    $\Dsynrevsymbol_{\rho}$(!pair!$_{T;A\times B}$) &=& $F$(!pair$_{T;A\times B}$, J$^T$pair$_{T;A\times B}$!) \\
    $\Dsynrevsymbol_{\rho}$(!proj!$_{T_1;T_2;T_3}$) &=& $F$(!proj$_{T_1;T_2;T_3}$,J$^T$proj$_{T1;T2;T3}$!) \\
    $\Dsynrevsymbol_{\rho}$(!e$_1\comp$e$_2$!) &=& $\Dsynrevsymbol_{\rho}$(!e$_1$!)$\comp$ $\Dsynrevsymbol_{\rho}$(!e$_2$!)\\ 
    $\Dsynrevsymbol_{\rho}$(!map2!$_{T;x,y.e}$) &=& $F$(!map2$_{T;x,y.e}$, J$^T$map2$_{T;x,y.e}$!) \\
    $\Dsynrevsymbol_{\rho}$(!reduce!$_{T;x,y.e_1;e_2}$) &=& $F$(!reduce$_{T;x,y.e_1;e_2}$, J$^T$reduce$_{T;x,y.e_1;e_2}$!) \\
    \end{tabular}\\
    where !$F$(A,B)$\defeq$ <proj$_{T;\underline{T}->\rho;[]} \comp$ A, proj$_{[];T;\underline{T}->\rho} \icomp$ (proj$_{T;\underline{T}->\rho;[]}$ $\comp$ B)>!\\\hline
    \end{tabular}
    \vspace{-0.4cm}
    \caption{Reverse-mode differentiation from Source UNF to Target UNF}
    \label{fig:diff_macro}    
    \vspace{-0.4cm}
\end{figure*}

Even though the target UNF is designed to achieve this purpose, we have the Following

\begin{proposition}
    If $\Gamma \vdash$!e!: !T! in source UNF, then
    $\Dtype(Gamma)\vdash \Dsynrevsymbol$(!e!):  $\Dtype$(!T!) in target UNF.
\end{proposition}

\subsection{transformations to and from UNF} % (fold)
\label{sub:transformations to and from UNF}

We give a translation from our source language to source UNF in Figure~\ref{fig:source_to_unf}.
UNF acts as a sort of stack. 
The input represents the current stage of the stack, and we return the new value added to the stack.
All results are kept, so the stack keeps growing. One might be surprised that the transformation for pairs and let is similar.
The order of UNF really follows what the trace of doing a sequential call-by-value evaluation would look like if the return value does not matter.
In other words, when evaluating !let x= e1 in e2! in call by value, we first evaluate !e1! and then !e2!. 
Similarly in !<e1, e2>!, we evaluate !e1! and then !e2!. 
Our ultimate goal is to return a gradient, not the original term, so what really matters is the order of evaluation, and what is evaluated.

$\widehat{;}$ is similar to $;$, with a sort of context propagation effect. 
It is a bit reminiscent of monadic composition versus the usual composition.
Technically though, the transformation $\UNFSymbol$ is not quite compositional.
\MH{explain in detail how to define it.}   

\begin{proposition}
    Let $\Gamma\vdash$!e!: !T! be a term in the source language.\\
    Then $\Gamma \vdash \UNFSymbol$(!e!): $\Gamma\times\Delta\times$!T! in source UNF for a certain type $\Delta$. 
\end{proposition}

$\Delta$ corresponds to the list of all the intermediate results from !e!. 
A different way to say it would be that $\Delta$ is the context made of the bound variables of !e! when !e! is in ANF.

\begin{figure*}[t]
    \begin{tabular}{r c l}
    $\UNFSymbol$($\Gamma\vdash $ !c!) &=& !c!$_{\Gamma,0}$ constant seen as a 0-ary operator\\
    $\UNFSymbol$($\Gamma\vdash $ !x!) &=& !var!$_{\Gamma,i}$ where !x! is the $i$-th variable in $\Gamma$ \\
    $\UNFSymbol$($\Gamma\vdash $ !let x:A = e1 in e2:B!) &=& $\UNFSymbol$(!e1!) $\comp$ $\UNFSymbol$(!e2!) $\comp$ !proj!_${\Gamma;A;B}$  \\ 
    $\UNFSymbol$($\Gamma\vdash $ !< e1, e2 >:AxB!) &=& $\UNFSymbol$(!e1!) $\pcomp$ $\UNFSymbol$(!e2!) $\comp$ !pair!$_{\Gamma,A\times B}$ \\ 
    $\UNFSymbol$($\Gamma\vdash \pi_i$(!e!)) &=& $\UNFSymbol$(!e!)$\comp $$\pi_i$ $\comp$ !proj!_${\Gamma;A_1\times A_2;A_i}$ seen as a unary operator\\
    $\UNFSymbol$($\Gamma\vdash $ !e1 op2 e2!) &=& $\UNFSymbol$(!e1!) $\pcomp$ $\UNFSymbol$(!e2!)$\comp$ !op!$_{\Gamma,2}$ $\comp$ !proj!$_{\Gamma;\reals,\reals;\reals}$ \\
    $\UNFSymbol$($\Gamma\vdash $ !op1 e!) &=& $\UNFSymbol$(!e!) $\comp$ !op!$_{\Gamma,1}$ $\comp$ !proj!$_{\Gamma;\reals;\reals}$ \\
    $\UNFSymbol$($\Gamma\vdash $ !map2 (x,y.e1) e2 e3!) &=& $\UNFSymbol$(!e2!) $\pcomp$ $\UNFSymbol$(!e3!) $\comp$ !map2!$_{\Gamma, x,y.e1}$ $\comp$ !proj!$_{\Gamma;\Array{\reals},\Array{\reals};\Array{\reals}}$ \\ 
    $\UNFSymbol$($\Gamma\vdash $ !reduce (x,y.e1) e2 e3!) &=& $\UNFSymbol$(!e3!)$\comp$ !reduce!$_{\Gamma; x,y.e1; e2}$ $\comp$ !proj!$_{\Gamma;\Array{\reals};\Array{\reals}}$ \\ 
    \end{tabular}
    \caption{UNF transformation from Source to Source UNF}
    \label{fig:source_to_unf}
    \end{figure*}

The transformation from the target UNF to the target language is presented in Figure~\ref{fig:unf_to_target}.

\begin{figure*}[t]
    \begin{tabular}{r c l}
    $\invUNFSymbol$(!var!$_{\Gamma,i}$) &=& $x_i$: !Ti! \\
    && if $x_i$: !Ti! is the $i$-th element of the context $\Gamma$ \\
    $\invUNFSymbol$(!op!$_{\Gamma,n}$) &=& !op!$_n(y_k,...,y_{k+n})$ \\
    && where $(y_k,...,y_{k+n})$ are the last $n$ variables of $\Gamma$ \\ 
    $\invUNFSymbol$(!e1;e2!) &=& !let y=!$\invUNFSymbol$(!e1!) !in! $\invUNFSymbol$(!e2!) \\ 
    && TODO: \\ 
    $\invUNFSymbol$(!map!$_{\Gamma,x.e}$) &=&  \\ 
    $\invUNFSymbol$(!map2!$_{\Gamma,x,y.e}$) &=&  \\ 
    $\invUNFSymbol$(!foldl!$_{\Gamma,x,y.e}$) &=&  \\ 
    $\invUNFSymbol$(!< e1, e2>!) &=&  \\ 
    $\invUNFSymbol$(!e2!$\circ$!e2!) &=& \\
    $\invUNFSymbol$(!J!$^T$!var!$_{\Gamma,i}$) &=& !fun Y -> fun (x1:T1,...,xm:Tm,z:Ti) -> Y(x1,...,xm)! \\
    &&where $\Gamma=$!T1!$\times...\times$!Tm! \\
    $\invUNFSymbol$(!J!$^T$!op!$_{\Gamma,n}$) &=& !fun Y -> fun (x1:T1,...,xm:Tm,z:Ti) ->!\\ 
    && Y$(x_1,...,x_{m-n},x_{m-n+1}+\partial_1op_n*z,...,xm+\partial_nop_n*z)$ \\
    && where $\partial_iop_n$ is the $i$-th partial derivative of $op_n$ \\
    $\invUNFSymbol$(!J!$^T$!map!$_{\Gamma,x.e}$) &=&  !fun Y -> fun (x1:T1,...,xm:Tm,X',Z) ->!\\
    && Y((x1,...,xm)+$\nabla_{\Gamma}e *$(!reduce! + 0 Z),\\
    && !map2 (fun (a,b) -> a+!$\nabla_{\{x\}}e *$!b) X' Z'!) \\
    $\invUNFSymbol$(!J!$^T$!map2!$_{\Gamma,x,y.e}$) &=&  !fun Y -> fun (x1:T1,...,xm:Tm,X1',X2',Z) ->!\\
    && Y((x1,...,xm)+$\nabla_{\Gamma}e *$(!reduce! + 0 Z),\\
    && !map2 (fun (a,b) -> a+!$\nabla_{\{x1\}}e *$!b) X1' Z'!, \\
    && !map2 (fun (a,b) -> a+!$\nabla_{\{x2\}}e *$!b) X2' Z'!) \\
    && where $X1'$, $X2'$ are the tangent arrays for the two arrays arguments \\
    && of !map2! and $Z$ the tangent part for the return array of !map2! \\
    $\invUNFSymbol$(!J!$^T$!foldl!$_{\Gamma,x,y.e1}$) &=&  !fun Y -> fun (x1:T1,...,xm:Tm,X',z) ->!\\
    && !let A = scanl (fun (x,y) -> e1) e2 e3 in! \\
    && !let B = map2 (fun (x,y) ->! $\nabla_{\{x1\}}$!e1(x,y)) A e3 in!\\
    && !let C = shift1 (map2 (fun (x,y) ->! $\nabla_{\{x2\}}$!e1(x,y)) A e3) in!\\
    && !let D = scanr (fun (x,y) -> x*y) 1 B in!\\
    && Y(x1,...,xm, !map3 (fun (a,b,c) -> a+b*c*z) X' D C!)
    \end{tabular}
    \caption{UNF transformation from Target UNF to Target}
    \label{fig:unf_to_target}
    \end{figure*}

\section{Complexity}
\label{sec:complexity}

\subsection{Cost model}
\label{sub:costModel}

We follow a simple model similar to the one in \cite{griewank2008evaluating}.
We assume the cost is divided into 4 elementary measures, being the number of MOVES, ADDS, MULTS, and NLOPS.
MOVES assumes a flat memory and represents moving a fixed size information  (e.g. 64 bits). 
ADDS represents the number of additions, 
MULTS the number of multiplications, 
and NLOPS the number of elementary non linear operations like !cos!, !exp!.

The gives a complexity function $\cost$ valued in $\RR^4$. 
For instance, we have 

\begin{itemize}
    \item $\cost(*)=(3,0,1,0)$
    \item $\cost(+)=(3,1,0,0)$
    \item $\cost(c)=(1,0,0,0)$
    \item $\cost($!sin!$)=(2,0,0,1)$
\end{itemize}

\MH{need cost array operations}
\MH{need to extend the cost function in a compositional way}


\subsection{Cheap gradient principle}

\begin{theorem}
    The reverse mode transformation satisfies the cheap gradient principle.
\end{theorem}

\subsection{Optimizations} % (fold)
\label{sub:Optimizations}

In Figure~\ref{fig:optim} we present a list of simple yet efficient optimizations.

\begin{figure*}[t]
    \begin{tabular}{|l c l|}
        \hline
        \textit{Inlining and forward substitution}  & &\\ \hline
        !(fun (x$_1$,$\ldots$,x$_n$) -> e)(e$_1$,$\ldots$,e$_n$)! & \multirow{2}{*}{\transto} & !let x$_1$=e$_1$ in $\ldots$! \\
        && !let x$_n$=e$_n$ in e! \\ \hline
        !let x$_1$=e$_1$ in e$_2$! \quad(!x$_1$!$\not\in$!FV(e$_2$))! & \transto & !e$_2$!  \\ \hline
        !let x$_1$=x$_2$ in e! & \transto & !e[x$_2$/x$_1$]! \\ \hline
        !let x$_1$= c in e!  & \transto & !e[c/x$_1$]! \\ 
        (!c= 0,1,ZerosLike! or !OnesLike!) && \\ \hline
        \hline \hline
        \textit{Algebraic simplifications}  & & \\ \hline
        !0*e! & \transto & 0 \\ \hline
        !0+e, 1*e! & \transto & !e! \\
        \hline \hline
        \textit{Array algebraic simplifications}  & & \\ \hline
        !map2 * A OnesLike(B)!  & \multirow{3}{*}{\transto} & \\
        !map2 + A ZerosLike(B)! && !A!\\
        !map (x.x) A! && \\ \hline
        !map2 * A ZerosLike(B)! & \transto & !ZerosLike(B)! \\ \hline
        !map2 (x,y.e) A B! \quad\quad(!y! not free in !e! ) & \transto & !map x.e A!  \\ \hline
        !map2 (x,y.e) A B! \quad\quad(!x! not free in !e! ) & \transto & !map y.e B!  \\ \hline
        !scanl * 1 OnesLike(A)! && \\
        !scanr * 1 OnesLike(A)! & \transto & !1! \\
        !reduce * 1 OnesLike(A)! && \\ \hline
        !scanl + 0 ZerosLike(A)! && \\
        !scanr + 0 ZerosLike(A)! & \transto & !0! \\
        !reduce + 0 ZerosLike(A)! && \\ \hline
        !shift1L OnesLike(n+1)! & \multirow{2}{*}{\transto} & !OnesLike(n)! \\ 
        !shift1R OnesLike(n+1)! && \\ \hline
       
        \hline \hline
        \textit{Classic array simplifications}  & & \\ \hline
        !map (x.e$_1$) (map (y.e$_2$) A)! & \transto & !map (y. let x=e$_2$ in e$_1$) A! \\ \hline
        !map (x.e$_1$) (map2 (y$_1$,y$_2$.e$_2$) A B)! & \transto & !map2 (y$_1$,y$_2$.let x=e$_2$ in e$_1$) A B! \\ \hline
        \textit{Tuple partial evaluation}  & & \\ \hline
        $\pi_i$!<e$_1$,$\ldots$,en>! & \transto & !ei! \\
        \hline \hline
        \textit{Let normalisation}  & & \\ \hline
        !let x=(let y=e$_1$ in e$_2$) in e$_3$! & \transto & !let y=e$_1$ in let x=e$_2$ in e$_3$! \\ \hline
        !f(let x=e$_1$ in e$_2$)! & \transto & !let x=e$_1$ in f(e$_2$)! \\
        \hline \hline
        \textit{Conditionals} & & \\ \hline
        !if e$_1$ then e$_2$ else e$_2$! & \transto & e$_2$ \\ \hline
        !if true then e$_2$ else e$_3$! & \transto & e$_2$ \\ \hline
        !if false then e$_2$ else e$_3$! & \transto & e$_3$ \\ \hline
        !f(if e$_1$ then e$_2$ else e$_3$)! & \transto & !if e$_1$ then f(e$_2$) else f(e$_3$)! \\ \hline
        \end{tabular}
    \caption{Optimizations}
    \label{fig:optim} 
\end{figure*}
\section{Correctness}
\label{sec:correctness}

This section provides a categorical analysis of our language and transformations.
The main result at the end of section is the correctness of the reverse mode transformation from Section~\ref{sub:Macro for pure reverse mode transformation}:

 \begin{theorem}
     For every $\Gamma \vdash$ !e:!$\reals$, we have 
     $\sem{\grad_\Gamma e}= \grad_\Gamma\sem{e}$.
 \end{theorem}
where $\sem{-}$ is an appropriate denotational semantics of smooth functions.

\subsection{Denotational semantics source and Target} % (fold)
\label{sub:Denotational semantics source and Target}

It is standard to give a denotational semantics of a first-order language in a Cartesian category.
An alternative approach consists of giving semantics in a multicategory.
A multicategory generalizes a category by allowing multimorphisms, that is, morphisms from a list of objects to an object.
Most categorical structures from category theory can be phrases similarly in multicategories.

In terms of semantics, a term $x_1:A_1,\ldots,x_n:A_n\vdash e:A$ is interpreted as a morphism $\sem{e}:[\sem{A_1},\ldots,\sem{A_n}]\to \sem{A}$ in a multicategory.
Substitution is still interpreted as composition. 

One can consider a syntactic model for a language, which consists in a free multicategory.
Our source and target languages induce syntactic multicategories as follows.

\begin{definition}[Syntactic multicategory for Source]
    Let $\SynSource$ be the multicategory whose objects are types of Source, and where a morphism 
    $[A_1,\ldots,A_n]\to A$ is a term $x_1:A_1,\ldots,x_n:A_n\vdash e:A$ of Source modulo the $\eta\beta$-laws.
    Composition is by substitution.
\end{definition}

We similarly define $\SynTarget$, the syntactic multicategory for Target.

As is standard, $\SynSource$ satisfies the following universal property: 
for every Cartesian multicategory $\catC$ with arrays map and reduce,
and every object $F(\reals)\in\catC$ and morphisms $F(\underline{c})\in\catC(1;F(\reals))$, 
$F(op1)\in\catC(F(\reals);F(\reals))$, $F(op2)\in\catC(F(\reals),F(\reals);F(\reals))$, there is a unique
multifunctor $F:\SynSource\to\catC$ respecting the interpretation and preserving all the structure.

We can, for instance, use this to give a simple semantics of Source 
in the multicategory of Cartesian spaces and smooth maps between them. 

\begin{definition}[$\CartSp$]
Let $\CartSp$ be the Cartesian category whose objects are Euclidean spaces
and whose morphisms are smooth functions between Euclidean spaces.
TODO:it's not a multicategory yet

We can interpret Source in $\CartSp$ as follows.
    \begin{tabular}{r c l}
    $\seml \reals \semr$ & $\defeq$& $\RR$ \\
    $\seml$!T1xT2!$\semr$ & $\defeq$& $\seml$!T1!$\semr \times\seml$!T2!$\semr$ \\
    $\seml \Array{\reals}{n}\semr$ & $\defeq$ & $\prod_{1\leq i \leq n} \RR$ 
    \end{tabular}
A context $\Gamma=\{x_1:A_1,\ldots,x_n:A_n\}$ is interpreted as the product $\prod_{1\leq i \leq n}\sem{A_i}$.
Well typed terms $\Gamma\vdash$!e: !$A$ are interpreted as functions $\sem{\Gamma}\to\sem{A}$. This is
routinely defined by induction on the structure of typing derivations. Constants
!c! $\reals$ are interpreted as constant functions; and the primitives operations !op1,op2!
are interpreted by composing with the corresponding functions, which are smooth. 
Variables are interpreted as $\sem{x_i}(\rho)\defeq \rho_i$. 
The remaining constructions are interpreted as follows.
\begin{tabular}{r c l}
    $\seml$!map2 (x,y.e1) e2 e3!$\semr$ &=& TODO\\
    $\seml$!reduce (x,y.e1) e2 e3!$\semr$ &=& TODO
\end{tabular}
\end{definition}

TODO: what about semantics Target?

\subsection{Semantics for UNF using concategories} % (fold)
\label{sub:Semantics for UNF using concategories}

One main reason for introducing UNF is to have a better handle over
the computation flow of the term, similar to what ANF or CPS provide. 
A convenient categorical setting for this is to use string diagrams.
To better fit the standard denotational semantics of languages, 
we can instead use concategories. 

Concategories are a generalization of categories and multicategories in that 
they allow morphisms both from and to a list of objects. 
Similarly to categories and multicategories, 
most categorical abstractions used for the semantics of languages are easily adapted to them.

In particular, one can form a syntactic concategory, 
which satisfies a similar universal property as syntactic multicategories.
Two such examples are of particular interest for us, 
because they will allow us to interpret Source and Target UNF, 
but also explain the UNF and $UNF^{-1}$ transformations.

\begin{definition}[$\ConcatS$]
Let $\ConcatS$ the syntactic Cartesian concategory whose types are those of Source and 
with primitives given by $op1: \reals \to \reals$, $op2: \reals,\reals \to \reals$ $map2_{x,y.e}$
and $reduce_{x,y.e1;e2}$.
TODO: actually I just need projection and a representing object for pairs, it might be weaker than Cartesian? edit: I think I need Cartesian.
\end{definition}

One may notice that the syntax of $\ConcatS$ is somewhere in between the syntax of Source and of Source UNF.

\begin{notation}[Morphisms in a Cartesian concategory]
    TODO: projection, unique map, $\concatcomp$, swap, pair, id, ...
\end{notation}

We can interpret Source UNF in $\ConcatS$ as follows.

\begin{tabular}{l c l}
   $\seml$!var!$_{T;i} \semr$ &=& $<id_\Gamma,\pi_i>$ \\
   $\seml$!op!$_{T;n} \semr$ &=& $id_\Gamma\concatcomp op_n$\\
   $\seml$!pair!$_{T;A\times B} \semr$ &=& $pair_{T;A\times B}$ \\
   $\seml$!proj!$_{T1;T2;T3} \semr$ &=& $id_{T1}\concatcomp 1_{T2}\concatcomp id_{T3}$\\
   $\seml$!e1!$\comp$!e2!$\semr$  &=& $\seml$!e1!$\semr \comp \seml$!e2!$\semr$ \\
   $\seml$!e1!$\pcomp$!e2!$\semr$ &=& $\seml$!e1!$\semr \comp (\seml$!e2!$\semr\concatcomp id )\comp swap$ \\
   $\seml$!map2!$_{T;x,y.e}\semr$  &=& $id_\Gamma\concatcomp map2_{x,y.e}$ \\
   $\seml$!reduce!$_{T;x,y.e;e}\semr$ &=& $id_\Gamma\concatcomp reduce_{x,y.e1;e2}$ \\
\end{tabular}

Similarly, we introduce a second concategory for the Target part of our transformations.

\begin{definition}[$\ConcatT$]
Let $\ConcatT$ be the syntactic Cartesian concategory whose types are those of Target and 
with primitives given by 
$op1: \reals \to \reals$, 
$op2: \reals,\reals \to \reals$, 
$map2_{x,y.e}: \Array{\reals}{n},\Array{\reals}{n}\to \Array{\reals}{n}$, 
$reduce_{x,y.e1;e2}: \Array{\reals}{n} \to \Array{\reals}{n}$, 
$J^Top1: \reals \to \reals$, 
$J^Top2: \reals \to \reals, \reals$, 
$J^Tmap2_{x,y.e}: \Array{\reals}{n} \to \Array{\reals}{n}, \Array{\reals}{n}$, 
$J^Treduce_{x,y.e1;e2}: \Array{\reals}{n} \to \Array{\reals}{n}$, 
TODO: not too sure, should I say what a Cartesian concat is? do I need $J^Tproj$?
\end{definition}

We can interpret Target UNF in $\ConcatT$ as follows. 
The part common with Source UNF is interpreted in the same way as for Source UNF.

TODO: I think need to explain here the (co)monoid structure induced by $D$

\begin{tabular}{l c l}
    $\seml$!J!$^T$!var!$_{T;i} \semr$ &=& TODO \\
    $\seml$!J!$^T$!op!$_{T;n} \semr$ &=&  \\
    $\seml$!J!$^T$!pair!$_{T;A\times B} \semr$ &=& \\
    $\seml$!J!$^T$!proj!$_{T1;T2;T3} \semr$ &=& \\
    $\seml$!J!$^T$!map2!$_{T;x,y.e}\semr$  &=& \\
    $\seml$!J!$^T$!reduce!$_{T;x,y.e;e}\semr$  &=& \\
    $\seml$!<e1, e2>!$\semr$  &=& !<!$\seml$!e1!$\semr$, $\seml$!e2!$\semr$!>! \\
    $\seml$!e1!$\icomp$!e2!$\semr$ &=&  \\
 \end{tabular}
TODO: Do I need any kind of proposition saying that this makes sense?
Might need to define on types as well to be clearer: It's an identity on types functor...

\subsection{Semantics for UNF transformations} % (fold)
\label{sub:Semantics for UNF transformations}

We interpret Source in a new multicategory, 
whose morphisms are particular morphisms of $\ConcatS$.
As Source UNF is itself interpreted in $\ConcatS$, 
this gives us a way to compare terms in Source with terms in Source UNF.
This comparison precisely gives the UNF transformation.

\begin{definition}[Mutlicat from concat]
    A multicategory $\catC$ naturally defines a concategory with the same objects as $\catC$ and
    with morphisms $A\to [B_1,\ldots,B_n]$ being $n$ morphisms $A\to Bi$ of $\catC$. 
\end{definition}

\begin{proposition}[construction above gives UNF]
    TODO
\end{proposition}

In a similar vein, we form a concategory from the syntactic multicategory for Target.
We then use the universal property of $\ConcatT$ to construct a functor from $\ConcatT$ to this concategory.
This allows us to compare the terms of Target UNF and the terms of Target. 
$UNF^-1$ arises in this way.

\begin{definition}[Concat from multicat]
    TODO
\end{definition}

\begin{proposition}[Initiality of concat2 gives $UNF^-1$]
    TODO
\end{proposition}

What remains to explain now is the reverse mode transformation between Source UNF and Target UNF.
We construct a functor $D:\ConcatS\to\ConcatT$. 
This functor will easily be shown to compute reverse-mode derivatives.
Because terms of Source UNF in $\ConcatS$, we observe the effect of $D$ on them
and show that it matches the syntactic $\Dsynrevsymbol_{\rho}$ 
from Section~\ref{sub:Simple reverse mode transformation}.

\begin{definition}[D as a lax functor]
    TODO
\end{definition}

\begin{proposition}[semantic of syntactic D matches D lax functor]
    TODO
\end{proposition}


\subsection{Correctness theorem} % (fold)
\label{sub:Correctness theorem}

First, we start from the correctness of $\Dsynrevsymbol_{\rho}$:Source UNF $\to$ Target UNF, which is easy to establish, 
and then propagate this information to Source and Target via the $\UNFSymbol, \UNFSymbol^{-1}$ transformations.

\begin{proposition}[Correctness $\Dsynrevsymbol_{\rho}$]
    For every term !$\reals^{\times n} \vdash$ e: $\reals^{\times n+1}$! in Source UNF,\\ 
    $\pi_2 \seml \Dsynrevsymbol_{\rho}$!e!$\semr(x_1,\ldots x_n,Id_{\RR^n})$=$J^T_{(x_1,\ldots x_n)}\seml$!e!$\semr$.
\end{proposition}

This is routinely proved by induction as the language is first-order. 
This uses the fact that for every primitive constant $A$, $\sem{J^TA}=J^T\sem{A}$.

Recall that the intuition from Source UNF is that it consists of terms of Source that also return their context.
From there, the intuition for the Jacobian of a primitive in Target UNF is that it should be the Jacobian of
the corresponding term in Target. This is easily checked for scalar operations.  

\begin{proposition}
    \begin{center}
\begin{tabular}{r c l}
    !$\seml\UNFSymbol^{-1}(J^T$map2$_{x,y.e})\semr$! &=& !$J^T\seml\UNFSymbol^{-1}($map2$_{x,y.e})\semr$!\\
    !$\seml\UNFSymbol^{-1}(J^T$reduce$_{x,y.e_1;e_2})\semr$! &=& !$J^T\seml\UNFSymbol^{-1}($reduce$_{x,y.e_1;e_2})\semr$!
\end{tabular}
\end{center}
\end{proposition}

This is proved in Appendix \ref{sub:Reverse derivative of array operations}.

From this, we now deduce that the composite transformation $\UNFSymbol$, $\Dsynrevsymbol_{\rho}$, $\UNFSymbol^{-1}$ is correct
in the sense that it produces a term that computes the gradient of the original term.

\begin{proposition}
    If !x$_1:\reals$,$\ldots$,x$_n:\reals$ $\vdash$ e: $\reals$! then \\
    $\seml \pi_2 \UNFSymbol^{-1}(\Dsynrevsymbol_{\rho}(\UNFSymbol($!e!$)))\semr(x_1,\ldots x_n,Id_{\RR^n})$=$J^T_{(x_1,\ldots x_n)}$!<!$Id_{\RR^n}$,$\seml$!e!$\semr$!>!
\end{proposition}

\begin{proof}
    TODO
\end{proof}

By inspecting what that composition of transformations does on terms of Source, 
we show that this indeed gives computes the same as the transformation 
$\directD{\rho}{\Gamma}{Y}$ from Section~\ref{sub:Macro for pure reverse mode transformation}. 

\begin{proposition}
    $\seml\UNFSymbol^{-1}(\Dsynrevsymbol_{\rho}(\UNFSymbol($!e!$)))\semr$ = $\seml\directD{\rho}{\Gamma}{Y}($!e!$)\semr$ 
\end{proposition}

\begin{proof}
    TODO
\end{proof}

Combining the the previous propositions, we have shown that $\directD{\rho}{\Gamma}{Y}$ is correct.

\begin{theorem}
    For every $\Gamma \vdash$ !e:!$\reals$, we have 
    $\sem{\grad_\Gamma e}= \grad_\Gamma\sem{e}$.
\end{theorem} 
\section{Generalizations}
\label{sec:generalization}


\subsection{Lifting the restriction on reduce}
\label{sub:Lifting the restriction on reduce}

Assume we allow $\Gamma$, !x!: $\reals$, !y!: $\reals$ $\vdash$ !e!: $\reals$ to be the function argument in
!reduce (x,y.e) v A!. We need to add a term depending on $\grad_{\{xi\}}e$ to !yi! in the continuation.
Writing the derivative becomes quite cumbersome, the notation is more compact using arrays of tuples. 
Similarly to he monoid $\widehat{+},0_\Gamma$ defined in Section~\ref{sub:Macro for pure reverse mode transformation},
we use $\widehat{\times},1_\Gamma$ for obvious extension of the monoid $(\reals,\times,1)$.

We can now compute the reverse-mode transformation as follows.
\begin{center}
\begin{tabular}{r c l}
$\directD{\rho}{\Gamma}{Y}$($\Gamma\vdash $ !reduce (x,y.e1) e2 e3!) 
&=& ... as before up until !A3! \\
&& !let B0 = map2 (a,b ->(!$\grad_\Gamma$!e)[a/x,b/y] A0 A) in!\\
&& !let B1 = scanr !$1_\Gamma$ $\widehat{\times}$ !B0 in! \\
&& !let B2 = reduce !$\widehat{+}$ $0_\Gamma$! B1 in! \\
&& !<reduce (x,y.e1) y1 A, fun (x1,...,xn,z) ->! \\
&& !Y((x1,...,xn)!$\widehat{+}$!B2,0,map2 (a,b. a*b*z) A2 A3)!
\end{tabular}
\end{center}

\subsection{Adding more array operators}
\label{sub:Adding more array operators}

There are two main differences with fold left !foldl! compared to !reduce!. 
First, in !foldl (x,y.e) v A! the starting accumulation element !v! is not a unit for !(x,y.e)!,
so it will have non-trivial derivatives in general and we need to account for that.
Second, we will need a more general !scanl! which allows !(x,y.e)! as a function argument. 
In other words, we need the general scan left computing the intermediate values of a fold left.
This should be a different primitive but we will still call it !scanl! in this section.

Finally, the former point implies we need a few more array manipulations. 
We write !c::[v1,...,vn]! to be the array ![c,v1,...,vn]!. 
Given an array ![v1,...,vn]! of size $n>0$, we write !let x,A=[v1,...,vn]! to mean 
that !x=v1! and !A=[v2,...,vn]!. The reverse derivative of fold left is then

TODO: reduce recomputed after scanl can be fixed here, explain let v,B notation somewhere
\begin{center}
\begin{tabular}{r c l}
    $\directD{\rho}{\Gamma}{Y}$($\Gamma\vdash $ !foldl (x,y.e1) e2 e3!) &=&
            !let v,Y1 = !$\directD{\rho}{\Gamma}{Y}$!(e2) in! \\
            && !let A,Y2 = !$\directD{\rho}{\Gamma,v}{Y1}$!(e3) in! \\
            && !let A0 = (scanl (x,y.e1) v A)[1..n] in! \\
            && !let A1 = map2 (a,b.(!$\grad_{\{x\}}$!e1)[a/x,b/y]) A0 A! \\
            && !let A2 = map2 (a,b.(!$\grad_{\{y\}}$!)e1[a/x,b/y]) A0 A! \\
            && !let A3 = scanr * 1 A1! \\
            && !<reduce (x,y.e1) v A, fun (x1,...,xn,z) ->! \\
            && !let y1,B = map2 (x,y. x*y*z) (1::A2) A3 in! \\
            && !Y2(x1,...,xn,y1,B)>! \\
\end{tabular}
\end{center}

TODO: other examples/recipe

\subsection{Adding more scalar operators} % (fold)
\label{sub:Adding more scalar operators}

We assumed the unary and binary operators were denoted by smooth functions $\RR^n\to\RR$. 
There is no additional difficulty in considering operators which are partial functions 
like division or operators which are not smooth at a point like square root.

These functions are then given intentional derivatives which provide valid derivatives 
on the domain of definition and differentiability of the operator. 
These functions are well known to be the bete noire of AD \cite{griewank2008evaluating} 
and we do not provide novel solutions to these.  
Several recent work have shown how to give semantics to such operators in the context of AD \cite{vakar2020denotational,mazza2021automatic,sherman2021,lee2020correctness}.

\subsection{Adding conditionals} % (fold)
\label{sub:Adding conditionals}

Adding conditionals to the source language can be done easily, for instance via defining

\begin{tabular}{r c l}
$\directD{\rho}{\Gamma}{Y}(\Gamma \vdash$ !if e1 then e2 else e3!) &=& !if e1 then !$\directD{\rho}{\Gamma}{Y}$!(e2) else !$\directD{\rho}{\Gamma}{Y}$!(e3)! 
\end{tabular}

The problem is that this could break the complexity of reverse mode because of the non-linear usage of $Y$, and makes everything harder to optimize.

A slightly better option would be to define 

\begin{tabular}{r c l}
    $\directD{\rho}{\Gamma}{Y}(\Gamma \vdash$ !if e1 then e2 else e3!) 
    &=& !let b=e1 in!   \\
    && !<if b then e2 else e3, fun (x1,...,xn,z) ->! \\
    && !Y(b*!$\directD{\rho}{\Gamma}{Y}$!(e2)(x1,...,xn,z)+!\\
    && \quad!(1-b)*!$\directD{\rho}{\Gamma}{Y}$!(e3)(x1,...,xn,z))>!
\end{tabular}

It is slightly better because both derivatives of !e2! and !e3! are put together, and this might unlock some optimizations, 
but there is still a non-linear usage of the continuation variable !Y!.
One solution, if we know we want to compute the whole gradient of the expression, is to define instead

\begin{tabular}{r c l}
    $\directD{\rho}{\Gamma}{Y}(\Gamma \vdash$ !if e1 then e2 else e3!) 
    &=& !let b=e1 in!   \\
    && !<if b then e2 else e3, fun (x1,...,xn,z) ->! \\
    && !Y((x1,...,xn)!$\widehat{+}$($\grad_\Gamma$!(e2)*b!$\widehat{+}\grad_\Gamma$!(e3)*(1-b))*z)>!\\
\end{tabular}

This time, there is a linear usage of the continuation variable !Y! (and of the variables !x1,...xn! too).

We note that adding conditionals does not break differentiability as long as there are no non-trivial primitives $\reals\to\BB$.
Non smooth functions such as the Rectified Linear Unit (ReLU) in machine learning which can be defined as $reLU(x)\defeq max(0,x)$ 
requires a non smooth primitive such as $>0: \reals\to\BB$. In several AD systems such as TensorFlow, 
the condition is evaluated before differentiation is applied, and so conditionals are never directly differentiated.

\subsection{Adding general arrays} % (fold)
\label{sub:Adding general arrays}

We now show how to generalise our reverse-mode transformation to be defined on arrays over any ground type $G$.
That is, we need to adapt the reverse derivatives of !map2! and !reduce! when they have more general function arguments.

A ground type $G$ is interpreted as an Euclidean space $A$. 
It is in particular a real vector space.
Similarly, a ground context $\Gamma=x1:G1,...,xn:Gn$ is interpreted as $\bigoplus_{1\leq i\leq n}Ai$, where $Ai\defeq\sem{Gi}$.
The denotation of the gradient of a term $\Gamma \vdash$!e: !$G$ at a point is then a matrix, more precisely an element of $()\bigoplus_{1\leq i\leq n}Ai)\otimes A$
where $\otimes$ is the tensor product of real vector spaces. This space is isomorphic to $\bigoplus_{1\leq i\leq n}Ai\otimes A$.

We can define $\otimes$ on the types of our language inductively by

\begin{tabular}{r c l}
    $\reals \otimes A$ & $\defeq$ & $A$ \\
    $(A1 \times ... \times An)\otimes A$ & $\defeq$ & $(A1\otimes A) \times ... \times (An \otimes A)$ \\
    $\Array{A1}{n} \otimes A$ & $\defeq$ & $\Array{A1\otimes A}{n}$
\end{tabular}

With this definition, we recover that the gradient of $\Gamma \vdash$!e: !$\reals$ is a tuple of type $A1\times...\times An$ as expected.
TODO: define *, finish D(reduce)

We will need a new primitive $\widehat{*}$ whose denotation is this map.

Then, using this notation, there are only minimal changes to the reverse derivatives of !map2! and !reduce!.

\begin{center}
\begin{tabular}{r c l}
    $\directD{\rho}{\Gamma}{Y}$($\Gamma\vdash $ !map2 (x,y.e1: G) e2 e3!) &=&  
    !let A,Y1 = !$\directD{\rho}{\Gamma}{Y}$!(e2) in! \\
    && !let B,Y2 = !$\directD{\rho}{\Gamma,A}{Y1}$!(e3) in! \\
    && !let C=!$\grad_{\Gamma}$!e1 in!\\
    && !<map2 (x,y.e1) A B, fun (x1,...,xn,Z) -> !\\
    && !Y2( (x1,...,xn)!$\widehat{+}$!(C!$\widehat{*}$!(reduce + 0 Z)),!\\
    && \quad\quad! map2 (a,b.(!$\grad_{\{x\}}$!e1)[a/x]!$\widehat{*}$!b) A Z,!\\
    && \quad\quad! map2 (a,b.(!$\grad_{\{y\}}$!e1)[a/x]!$\widehat{*}$!b) B Z )>!\\
    \end{tabular}
    \end{center}

\begin{center}
\begin{tabular}{r c l}
        $\directD{\rho}{\Gamma}{Y}$($\Gamma\vdash $ !reduce (x,y.e1) e2 e3!) 
        &=&  !let y1,Y1 = !$\directD{\rho}{\Gamma}{Y}$!(e2) in! \\
        && !let A,Y2 = !$\directD{\rho}{\Gamma,y1}{Y1}$!(e3) in! \\
        && !let A0 = scanl (x,y.e1) y1 A in! \\
        && !let A1 = map2 (a,b.(!$\grad_{\{x\}}$!e1)[a/x,b/y]) A0 A! \\
        && !let A2 = shift1 (map2 (a,b.(!$\grad_{\{y\}}$!)e1[a/x,b/y]) A0 A)! \\
        && !let A3 = scanr * 1 A1! \\
        && !<reduce (x,y.e1) y1 A, fun (x1,...,xn,z) ->! \\
        && !Y((x1,...,xn),0,map2 (a,b. a*b*z) A2 A3)!
\end{tabular}
\end{center}

Evidently, one can combine all the generalizations from the previous subsections.
Even though this transformation has the correct complexity, it is open for future research to find
even better representations to allow for more optimizations. 
In particular, representations looking like Einsum \cite{van2011numpy} could be of interest 
and has been recently studied in the context of AD \cite{laue2018computing,laue2020simple}.
More generally, there is growing interest in tensor calculus \cite{liao2019differentiable,bernstein2020differentiating}.

\section{Discussion and future work} % (fold)
\label{sec:discussion_and_future_work}

\noindent \textbf{More dynamic language}
JAX \cite{bradbury2020jax,frostig2018compiling} is a modern framework that provides state-of-the-art automatic differentiation. 
One feature that makes it popular is that it uses JIT compilation, 
which seems to be the best avenue to help best ML practitioners as it allows them 
to quickly experiment but also benefit from the full power of static optimizations.
We are interested in seeing how our work could sit in a JIT compiler. 

\noindent \textbf{More optimizations.}
As pointed in \ref{sub:Adding general arrays} it would be interesting to have a more principled representation 
of tensors which could lead to more optimizations. 
There is always a tension between dynamicity and the possibility for static optimizations 
and we believe our language offers a good tradeoff, but it will be interesting to tailor it 
further for specific applications.

\noindent \textbf{Cost function and memory accesses}
Our cost model \ref{sub:costModel} is simple but has limits. 
In particular, it does not capture well cost in term of
parallel computing, and memory management (cache optimization is one of best way to optimize such linear algebra intensive computations).
For instance, in the earlier days of automatic differentiation, 
advanced checkpointing techniques were used to avoid storing all the intermediate values just to save memory 
and try avoiding saturation of RAM. 
These problems are harder to analyze and, eventually, it seems to come down to a lot of experimenting and benchmarking.

\noindent \textbf{Parallelism and GPU}
We tried to locally preserve parallelism. Having a pure transformation should help with found parallelism. 
It would be interesting to implement on transformations and see how they perform on GPU.
It is well-known that the addition of references provoked by the standard reverse-mode transformation
restricts a lot of optimizations and it is not hard to find examples in C where the compiler does not detect simple optimizations 
because of the references introduced by reverse-mode.

% \noindent \textbf{implementation, BLaS, benchmark}

\noindent \textbf{Higher-order functions}
Matthijs'work, use of defunctionalization, question efficiency and usefulness, 
interested in functions like root of function or minimal of a function. Implicit function theorem
Some recent work \cite{vakar2020reverse,sherman2021} present reverse-mode in a higher-order language.
\cite{vakar2020reverse} uses categorical semantics to show correctness of the reverse-mode transformation 
and \cite{sherman2021} uses sophisticated higher-order primitives such as root finding, max, argmax or integral. 
Their work focuses on computable reals which is hard to compare in terms of efficiency 
with our more standard approach of AD.

\noindent \textbf{More array primitives}
filter, flatten, gather, scanl, ...
can be derived by hand once and then it's ok

\noindent \textbf{Recursion}
TODO

\noindent \textbf{Higher derivatives}
TODO

\noindent \textbf{Sparse data and matrix representation}
TODO

\noindent \textbf{Further applications}
Reverse-mode differentiation is notoriously difficult to grasp, implement, and optimize.
We hope this work could help develop a new perspective on pure reverse-mode automatic differentiation.
Other applications include the use of fancier datatypes e.g. dependent types, 
in probabilistic programming, e.g. for HMC or Variational Inference. 
It is also of interest to see how to extend this work with other techniques which 
have had renewed interest such as implicit differentiation \cite{blondel2021efficient,lorraine2020optimizing}.

% 	https://github.com/apple/swift/blob/master/docs/DifferentiableProgramming.md#approaches-to-automatic-differentiation

\MH{understand related work sentence from Diff. Curry paper: The idea of using closures as back-propagators is receiving recent attention. 
    For example Julia Zygote \cite{innes2019zygote} and Swift AD \cite{wei2018first} chose this design. 
    Other recent work follows similar ideas [27, 26] but is using meta-programming as an implementation technique.}

\MH{should talk a bit about the recent paper: reverse AD in Dex \cite{paszke2021getting}}

\section{Related Work} % (fold)
\label{sec:related_work}

\noindent \textbf{Usage of iteration mechanics}
A immense effort in machine learning for the past decade has been in finding
good architectures, to limit computational costs, 
avoid vanishing and exploding gradients, 
and have better building blocks of large complicated systems than traditional layers of a neural network.
Different approaches 
(Dynamic neural networks \cite{jin2017manipulability,wu2016deep}, 
Recursive NN \cite{socher2011parsing,biancofiore2017recursive}, 
Reccurent NN \cite{bahdanau2014neural,luong2015effective}, 
Tree LSTM \cite{tai2015improved,chen2016enhanced}, 
Dynamic Recursive NN \cite{guo2019dynamic}, 
Top-down Tree LSTM \cite{zhang2015top}, 
Recursion in DNN \cite{jeong2018improving}) 
have found that recursive data structures such as trees are good candidates.
We have emphasized here on differentiating fold left on arrays for efficiency, 
but one should be able to adapt this on any algebraic data type. 
It will be interesting to see if and how we recover efficient backpropagation on the proposed architectures 
which is usually derived by hand and one main goal of these papers.

\noindent \textbf{Correctness of AD in functional languages.}
Several recent work \cite{huot2020correctness,vakar2020reverse,vakar2020denotational,brunel2019backpropagation,barthe2020versatility,mazza2021automatic,ee2020correctness} have focused on correctness of AD in a purely functional setting, 
often leaving efficiency on the side, especially for reverse-mode differentiation. 
We see our work as a complement and a first bridge between these works 
and more practical considerations of efficiency and implementations, 
which often require a lot more care than is acknowledegd in more theoretical works.

An exception is \cite{abadi-plotkin2020} which gives a reverse-mode transformation that is proved correct and with the right complexity.
This work is purely first-order, does not have arrays and uses a non standard semantics. 
Our work is more canonical in the sense that it can more directly be implemented in a standard call-by-value language.
They do allow conditionals and recursion but their semantics suggest that they don't have a canonical transformation on these constructs. 
Instead, they follow the principle of unrolling for recursive calls and evaluating the conditional before differentiating, 
which requires to recompute the whole gradient when taken at new values that can change the value of the conditional. 
This is a well-known limit in TensorFlow \cite{abadi2016tensorflow} which TeaserFlow EagerMode \cite{agrawal2019tensorflow} tries to bypass. 

\noindent \textbf{Array Languages and AD}

Given the enormous computation needs for state-of-the-art large scale machine learning applications, 
which require extremely efficient tensor computations and automatic differentiation for backpropagation, 
combining array languages and automatic differentiation (tensor calculus) in the best 
fitted intermediate representation for optimizations is of key interest and active research.
\cite{bernstein2020differentiating} 
TODO
\cite{laue2018computing,laue2020simple}

\noindent \textbf{Comparison to other recent papers.}

TODO
\cite{ee2020correctness}: non diff function
\cite{sherman2021}: non diff HO functions
\cite{pearlmutter2008reverse}: based on this
\cite{elliott2018simple}: bit similar because using CT as useful language, but we take impl and opti more seriously.
\cite{shaikhha2019efficient}: similar approach but here more complicated for reverse mode, would be interesting to investigate more loop fusions here.

Relatively recent work in AD (2008-2020):
\cite{mak2020differential,elliotthigher,vytiniotis2019differentiable,innes2018don,baydin2017automatic,huot2020correctness,gallagher-sdg,manzyuk2012confusion,wang2018demystifying,beck1994if,wang2018backpropagation,betancourt2018geometric,elliott2018simple,carpenter2015stan,paszke2017automatic,shaikhha2019efficient,innes2019zygote,griewank2008evaluating,kucukelbir2017automatic,brunel2019backpropagation,barthe2020versatility,abadi2019simple,cockett2019reverse,van2018automatic,hascoet2013tapenade,abadi2016tensorflow,pearlmutter2008reverse,bergstra2010theano,fong2019backprop,ehrhard2003differential,agrawal2019tensorflow,bettencourt2019taylor,cruttwell2017cartesian,manzyuk2012simply,laue2018computing,bernstein2020differentiating}

Good recent surveys: \cite{van2018automatic,baydin2017automatic}

\section{Conclusion}
\label{sec:conclusion}

We have introduced a transformation on programs to compute provably efficient (\ref{sec:complexity}) 
gradients via reverse derivatives in a purely functional way (\ref{sub:Macro for pure reverse mode transformation})
on a simple yet expressive language with functions on arrays (\ref{sub:sourcelang}), 
combined with standard functional optimizations (\ref{fig:optim}).  
We introduced a novel intermediate representation, unary form (\ref{sec:unf}) 
to decompose our transformation into simpler transformations.
We gave denotational semantics to our languages and our transformations (\ref{sec:correctness}) 
and showed correctness of the reverse-mode transformation (\ref{sub:Correctness theorem}).
We showed (\ref{sec:generalization}) how to lift some of the restrictions that
we introduced on arrays and how to extend our approach to a richer language with more primitive operations and conditionals.
\clearpage 

\begin{acks}
We have benefited from discussing this work with many people, including Y.~Kaddar, J.~Sigal, M.~V\'ak\'ar, E. Arrighi, S. Staton and others. 
\end{acks}

\bibliography{refs}

\appendix

 \section{Appendix}

 \subsection{Operational semantics}

In Figure~\ref{fig:op_semantics_target} a small step call-by-value operational semantics for the language. 
The evaluation contexts aregiven in Figure~\ref{fig:ev_contexts}. 

\begin{figure*}[t]
    \begin{tabular}{|l c l|}
        \hline
        \multicolumn{3}{|l|}{Evaluation contexts} \\
        $E$ & \mbox{::=} & 
        [] 
        $\mid$ !let x = E in e! 
        $\mid$ !<E, e>!
        $\mid$ !<v, E>!
        $\mid$ $\pi_i$!(E)!
        $\mid$ !E op2 e!
        $\mid$ !v op2 e!
        $\mid$ !op1 E! \\
        && $\mid$ !map (x.e) E!
        $\mid$ !map2 (x,y.e) E e!
        $\mid$ !map2 (x,y.e) v E! \\
        && $\mid$ !foldl (x,y.e) E e!
        $\mid$ !foldl (x,y.e) v e! \\
        && $\mid$ !reduce (x,y.e) E e! 
        $\mid$ !reduce (x,y.e) v e! \\
        && $\mid$ !scanl (x,y.e) E e!
        $\mid$ !scanl (x,y.e) v e! \\
        && $\mid$ !scanr (x,y.e) E e!
        $\mid$ !scanr (x,y.e) v e! \\
        && $\mid$ !map3 (x,y.e) E e e!
        $\mid$ !map3 (x,y.e) v E e!
        $\mid$ !map3 (x,y.e) v v E! \\
        && $\mid$ !shift1L E! 
        $\mid$ !shift1R E!
        $\mid$ !E(e$\ldots$e)!
        $\mid$ !e(v$\ldots$vEe$\ldots$e)!
        $\mid$ !if E then e else e! \\
        && $\mid$ ![v,$\ldots$,v,E,e$\ldots$,e]!
        \\ \hline
        \multicolumn{3}{|l|}{Values} \\ 
        !v! & \mbox{::=} & 
        \cnst{}  
        $\mid$ !<v, v>!
        $\mid$ !true! 
        $\mid$ !false!
        $\mid$ !fun (x1,$\ldots$,xn) -> e!
        $\mid$ ![v,$\ldots$,v]! 
        \\ \hline
        \end{tabular}
    \caption{Evaluation contexts and values}
\label{fig:ev_contexts}    
\end{figure*}

Some operators are less conventional which we now explain.
!shift1! takes an array of reals of size $n$ and adds a $1$ as first component and forgets about the last component of the array, returning an array of size !n!.
!scanr! is similar to !scanl! except that the folding starts from the end of the array. 
It is equivalent to first reversing the input array, performing !scanl!, and reversing the resulting array.

\MH{need to explain that the !scanl!, !scanr! coming from !reduce! are also parallel ones}

\MH{might need to extend with the operators from the generalized section}

\begin{figure*}[tb]
\begin{tabular}{|l c l|}
    \hline
    !op1! \cnst{} & \transto & \underline{op1(c)} \\ \hline
    \cnst{} !op2! \cnst{}' & \transto & \underline{op2(c,c')}\\ \hline
    !let x=v in e! & \transto & !e[v/x]!  \\ \hline
    $\pi_i$!<v1, v2>! & \transto & !vi!\\ \hline
    !(fun (x1,...,xn) -> e)(v1...vn)! & \transto & !e[v1/x1,...,vn/xn]! \\ \hline
    !shift1 [v1,...,vn]! & \transto & ![1,v1,...,v(n-1)]! \\ \hline
    !scanl (x,y.e) v []! & \transto & ![v]! \\\hline
    !scanl (x,y.e) v [v1,...,vn]! & \transto & !v::(scanl (x,y.e) e[v/x,v1/y] [v2,...,vn])!\\ \hline
    !scanr (x,y.e) v []! & \transto & ![v]! \\ \hline
    !scanr (x,y.e) v [v1,...,vn]! & \transto & !(scanl (x,y.e) e[v/x,v1/y] [v2,...,vn])::v! \\ \hline
    !foldl (x,y.e) v []! & \transto & !v! \\ \hline
    !foldl (x,y.e) v [v1,...,vn]! & \transto & !foldl (x,y.e) e[v/x,v1/y] [v2,...,vn])! \\ \hline
    !reduce (x,y.e) v []! & \transto & !v! \\ \hline
    !reduce (x,y.e) v [v1,...,vn]! & \transto  & !reduce (x,y.e) e[v/x,v1/y] [v2,...,vn])!\\ \hline
    !map (x.e) [v1,...,vn]! & \transto & ![e[v1/x],...,e[vn/x]]! \\ \hline
    !map2 (x,y.e) [v11,...,v1n]! & \multirow{2}{*}{\transto} & ![e[v11/x,v21/y],...,e[v1n/x,v2n/y]]! \\ 
    ![v21,...,v2n]! && \\ \hline
    !map3 (x,y,z.e) [v11,...,v1n]!  & \multirow{2}{*}{\transto} & ![e[v11/x,v21/y,v31/z],...,! \\
    ![v21,...,v2n] [v31,...,v3n]! && \hspace{0.6em}!e[v1n/x,v2n/y,v3n/z]]! \\ \hline
    !if true then e1 else e2! & \transto  & !e1! \\ \hline
    !if false then e1 else e2! & \transto  & !e2! \\ \hline
    \end{tabular}
\vspace{-0.2cm}
\caption{Operational semantics of the source and target languages}
\vspace{-0.4cm}
\label{fig:op_semantics_target}
\end{figure*}

 \subsection{Reverse derivative of array operations}

We now prove that the reverse mode transformation is correct on array operations. 
The proofs consist in showing that unrolling the computation before differentiation 
and then differentiating gives the same as differentiation and then unrolling. 

 \begin{proposition}
     The reverse derivative of !map! is correct.
 \end{proposition}

\begin{proof}
    
\end{proof}

 \begin{proposition}
    The reverse derivative of !map2! is correct.
\end{proposition}

\begin{proof}
    
\end{proof}

\begin{lemma}
    if !op2! is an associative binary operation with unit $\Gamma \vdash$ !v:! $\reals$, then 
    $\frac{\partial op2(v,e)}{\partial y_1}\times\frac{\partial v}{\partial x_i}=0$ 
    and $\frac{\partial op2(e,v)}{\partial y_2}\times\frac{\partial v}{\partial x_i}=0$ for all $x_i$.
\end{lemma}

\begin{proof}
    For any $\Gamma \vdash$ !e:! $\reals$, we have !v op2 e!=!e!.
    Differentiating and using the chain rule we get 
    $$\frac{\partial op2(v,e)}{\partial y_1}\times\frac{\partial v}{\partial x_i}
    +\frac{\partial op2(v,e)}{\partial y_2}\times\frac{\partial e}{\partial x_i}
    = \frac{\partial e}{\partial x_i}$$
As $\frac{\partial e}{\partial x_i}$ is arbitrary, 
this shows that $\frac{\partial op2(v,e)}{\partial y_2}=1$ and $\frac{\partial op2(v,e)}{\partial y_1}\times\frac{\partial v}{\partial x_i}=0$.
Similarly for the other case.
\end{proof}

\begin{proposition}
    The reverse derivative of !reduce! is correct.
\end{proposition}

\begin{proof}
    
\end{proof}

\subsection{gradient from the introduction}
\label{sub:gradintro}

We show that the gradients from Section \ref{sec:intro} are obtained as instances of our general construction. 
The proofs consist in instantiating the general derivatives to these cases 
and showing that each rewrite step is a simple known optimization.

Similarly to numpy, we use the notation OnesLike(A) to mean map (x -> 1) A 
and ZerosLike(A) to mean map (x -> 0) A. 
Finding these constant arrays is key to a lot of optimizations that leverage the ring algebraic structure of the reals to arrays.

\MH{be more explicit about optimisations. 0 should be ZerosLike. A bit outdated compared to current D transformation.
Also, I am freely referring to the variables from the context or bound in the original term, should be clarified.}

 \begin{lemma}
     $\nabla_A$!prod(A)! = !map2 * (scanr x 1 A) (shift1(scanl * 1 A))!
 \end{lemma}

 \begin{proof}
The gradient of  derivative $\nabla_A$!prod(A)! is given by

\begin{tabular}{c l r}
    & $\nabla_A$ !prod(A)! & \\
    =&  $\nabla_A$(!reduce * 1 A!) & \\
    $\defeq$ & \Big($\lambda$ !A',z. let A1 = scanl * 1 A in! & \\
    & !let A2 = map2 (x,y -> y) A1 A in! & \\ 
    & !let A3 = shift1 (map2 (x,y -> x) A1 A) in! & \\
    & !let A4 = scanr * 1 A2 in! &\\
    & !map3 (a,b,c -> a+b*c*z) A' A4 A3! & \\
    & \Big)(0,1)\\
     $\stackrel{\beta-reduction}{=}$  & !let A1 = scanl * 1 A in! & \\
    & !let A2 = map2 (x,y -> y) A1 A in! & !A2=A! \\
    & !let A3 = shift1 (map2 (x,y -> x) A1 A) in! & !A3=shift1 A1!\\
    & !let A4 = scanr * 1 A2 in! & \\
    & !map3 (a,b,c -> a+b*c*1) 0 A4 A3! & !map2 * A4 A3! \\
    = & !let A1 = scanl * 1 A in! & \\
    & !let A3 = shift1 A1 in! &\\
    & !let A4 = scanr * 1 A in! & forward substitution !A2!\\
    & !map2 * A4 A3! & \\
    $\stackrel{\eta-reduction}{=}$ & !map2 * (scanr x 1 A) (shift1(scanl * 1 A))! &
\end{tabular}
 \end{proof}

 \begin{lemma}
     $\nabla_A$!sum(A)! = !map (x -> 1) A!
 \end{lemma}

 \begin{proof}
The gradient of !sum(A)! is given by

    \begin{tabular}{c l r}
    & $\nabla_A$ !sum(A)! & \\
    = & $\nabla_A$ !reduce + 0 A! & \\
    $\defeq$ & \Big($\lambda$ !A',z. let A1 = scanl * 1 A in! \\
    & !let A2 = map2 (x,y -> 1) A1 A in! & \\
    & !let A3 = shift1 (map2 (x,y -> 1) A1 A) in! & \\
    & !let A4 = scanr * 1 A2 in! & \\
    & !map3 (a,b,c -> a+b*c*z) A' A4 A3!\Big)(0,1) & \\
    $\stackrel{\beta-reduction}{=}$  & !let A1 = scanl * 1 A in! & \\
    & !let A2 = map2 (x,y -> y) A1 A in! & !A2=OnesLike(A)! \\
    & !let A3 = shift1 (map2 (x,y -> x) A1 A) in! & !A3=shift1 OnesLike(A)!\\
    & !let A4 = scanr * 1 A2 in! & \\
    & !map3 (a,b,c -> a+b*c*1) 0 A4 A3! & !map2 * A4 A3! \\
    = & !let A1 = scanl * 1 A in! & \\
    & !let A2= OnesLike(A) in! &  \\
    & !let A3 = shift1 OnesLike(A) in! & !A3=OnesLike(A)! \\
    & !let A4 = scanr * 1 A2 in! & !A4=OnesLike(A2)! \\
    & !map2 * A4 A3! & \\
    = & !map2 * OnesLike(A) OnesLike(A)! & \\
    = & !OnesLike(A)!
    \end{tabular}
\end{align*}
 \end{proof}

 \begin{lemma}
     $\nabla_A$!dot(A,B)! = !B! 
 \end{lemma}

 \begin{proof}
The reverse derivative of !map2 * A B! is given by 

\begin{tabular}{c l}
    & $\lambda$ !A',B',C'.! \\
    & !let C1 =map2 (a,b -> b) A B in! \\
    & !let C2 =map2 (a,b -> a) A B in! \\
    & !(map3 (a,b,c -> a+b*c) A' C1 C', map3 (a,b,c -> a+b*c) A' C2 C')!
\end{tabular}

Let's call this term Y.
For convenience, let us also rewrite !map2 * B (map (x -> 1) A)! 
as !let C = map (x -> 1) A in map2 * B C!.

Then the gradient of !dot(A,B)! is given by

\begin{tabular}{c l r}
    & $\nabla$ !dot(A,B)! & \\
    = & $\nabla$ !let C = map (x -> 1) A in map2 * B C! & \\
    $\defeq$ & $\Big(\lambda$ A',B',C',z. & \\
    & !let A1 = scanl * 1 A in!  \hspace{4cm}\rdelim\}{5}{3mm}[\parbox{40mm}{same reduction\\as previously}]\\
    & !let A1 = scanl * 1 A in! & \\
    & !let A2 = map2 (x,y -> 1) A1 A in! & \\
    & !let A3 = shift1 (map2 (x,y -> 1) A1 A) in! &\\
    & !let A4 = scanr * 1 A2 in! &\\
    & !Y(A', B', map3 (a,b,c -> a+b*c*z) A' A4 A3)!\Big) &  \\
    & \quad !(0,0,0,1)! & \\
    = & !Y(0,0,OnesLike(C))! & \\
    $\defeq$ & !let C1=map2 (a,b -> b) A B in! & !C1 = B! \\
    & !let C2=map2 (a,b -> a) A B in! & !C2 = A! \\
    & !(map3 (a,b,c -> a+b*c) 0 C1 OnesLike(C)!, & !map2 C1 OnesLike(C)! \\
    & !map3 (a,b,c -> a+b*c) 0 C2 OnesLike(C))!  & !map2 C2 OnesLike(C)! \\
    = & !let C1=B! & forward substitution \\
    & !let C2=A! & forward substitution \\
    & !map2 C1 OnesLike(C)! & !C1! \\
    & !map2 C2 OnesLike(C)! & !C2! \\
    = & !(B, A)!
\end{tabular}

So if we are only interested in the gradient w.r.t. A, this indeed gives us B.
\end{proof}

\end{document}
\endinput
