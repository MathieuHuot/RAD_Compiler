\documentclass{article}
\usepackage[utf8]{inputenc}
\usepackage{proof}
\usepackage{mathpartir}
\usepackage{a4wide}
\usepackage{datetime}
\usepackage{mathtools}
\usepackage{amssymb}
\usepackage{xcolor}
\usepackage[all,color]{xy}
\usepackage[normalem]{ulem}
\usepackage{contour}
\usepackage{microtype}
\usepackage{cleveref}
\usepackage{amsmath}
\usepackage{a4wide}
\usepackage{datetime}
\usepackage{mathtools}
\usepackage{amssymb}
\usepackage{todonotes}
\usepackage{amsmath}
\usepackage{a4wide}
\usepackage{datetime}
\usepackage{mathtools}
\usepackage{amssymb}
\usepackage{stmaryrd}
\usepackage[backend=biber,sorting=nty]{biblatex}
\usepackage{amsmath}
\usepackage{mathtools}
\usepackage{suffix}
\usepackage{tikz-cd}
\usepackage{mathpartir}
\usepackage{enumitem}
\usepackage{stmaryrd}
\usepackage[all]{xy}
\usepackage{twoopt}
\usepackage{array}
\usepackage{listings}
\usepackage{float}

\bibliography{biblio.bib}
\newcommand{\SSS}[1]{\todo[inline,author=SS]{#1}}
\newcommand{\sss}[1]{\todo[size=\tiny]{SS: #1}{}}
\newcommand{\AG}[1]{\todo[inline,author=AG]{#1}}
\newcommand{\ag}[1]{\todo[size=\tiny]{ag: #1}{}}
\newcommand{\MH}[1]{\todo[inline,author=MH]{#1}}
\newcommand{\mh}[1]{\todo[size=\tiny]{mh: #1}{}}

\newcommand\cat[1]{\mathcal{#1}}
\newcommand\catB{\cat{B}}
\newcommand\catC{\cat{C}}
\newcommand\catD{\cat{D}}
\newcommand\catE{\cat{E}}
\newcommand\liftedto{\stackrel{\cdot}{\to}}
\newcommand\projf{\mathrm{proj}}


\lstdefinelanguage{llql}%
{morekeywords={
  if,then,else,let,in,
  op2,op1,map,map2,foldl,
  scanl,scanr,map3,fun,reduce,
  op,var,J,
  sum,prod,dot,fst,
  zip
  },%
  sensitive,%
  morecomment=[l]//,%
  morecomment=[s]{/*}{*/},%
  morestring=[b]",%
  morestring=[b]',%
  showstringspaces=false,%
  morecomment=[s][\color{gray}]{@}{\ },%
    breaklines=true,%
  mathescape=true,%
showspaces=false,
showtabs=false,
showstringspaces=false,
breakatwhitespace=true,
  aboveskip=1pt,
  belowskip=1pt,
  lineskip=-0.2pt,
   numbersep=5pt,
   numberstyle=\tiny\ttfamily,
   basicstyle=\small\ttfamily,
   keywordstyle=\bfseries\color{blue!70!black},%
   columns=fullflexible,
  frame=single,
  escapeinside={(*@}{@*)}
}[keywords,comments,strings]%

\lstset{language=llql}
\lstMakeShortInline[columns=fixed]!

\newtheorem{notation}{Notation}
\newtheorem{remark}{Remark}

\newcommand{\code}[1]{\texttt{#1}}

%-- Syntax --
\makeatletter
\newcommand\@TyAlph[1]{%
\ifcase #1\or \tau\or \sigma\or \rho\else \@ctrerr \fi%
}
\newcommand\ty[1][1]{{\@TyAlph{#1}}}

\newcommand\tvar[1][1]{{\@TyVarAlph{#1}}}
\newcommand\@TyVarAlph[1]{%
\ifcase #1\or \alpha\or \beta\or \gamma\else \@ctrerr \fi%
}

\newcommand\var[1][1]{{\@VarAlph{#1}}}
\newcommand\@VarAlph[1]{%
\ifcase #1\or x\or y\or z\or u\or v\or w\else \@ctrerr \fi%
}

\newcommand\trm[1][1]{{\@TermAlph{#1}}}
\newcommand\@TermAlph[1]{%
\ifcase #1\or t\or s\or r\else \@ctrerr \fi%
}

\newcommand\val[1][1]{%
\ifcase #1\or v\or w\or u\else \@ctrerr \fi%
}

\newcommand\op[1][1]{%
\ifcase #1\or \mathsf{op}\or \mathsf{op}'\or \mathsf{op}''\else \@ctrerr \fi%
}
\makeatother

\newcommand\Dsynsymbol[1][]{\scalebox{0.8}{$\overrightarrow{\mathcal{D}}$}_{#1}}
\newcommand\Dsyn[2][]{\Dsynsymbol[#1](#2)}

\newcommand\Dsynrevsymbol[1][]{\scalebox{0.8}{$\overleftarrow{\mathcal{D}}$}_{#1}}
\newcommand\Dsynrev[2][]{\Dsynrevsymbol[#1](#2)}

\newcommand\letin[3]{\mathbf{let}\,#1=\,#2\,\mathbf{in}\,#3}
\newcommand\SSSynHO{\mathbf{SSSynHO}}
\newcommand\SSSyn{\mathbf{SSSyn}}
\newcommand\TSSyn{\mathbf{TSSyn}}
\newcommand\TTSynHO{\mathbf{TTSynHO}}
\newcommand\TTSyn{\mathbf{TTSyn}}

\newcommand\Syn{\mathbf{Syn}}
\newcommand\Polysyn{\mathbf{Poly[Syn]}}

\newcommand\IH{\stackrel{I.H.}{=}}

\newcommand\UNFSymbol{\mathbf{UNF}}
\newcommand\UNF[1]{\UNFSymbol(#1)}
\newcommand\invUNFSymbol{\mathbf{UNF}^{-1}}
\newcommand\invUNF[1]{\UNFSymbol^{-1}(#1)}
\newcommand\tPair[2]{\langle #1, #2\rangle}
\newcommand\inllambda{\widetilde{\lambda}}

\newcommand\bp[1]{\boldsymbol{(}#1\boldsymbol{)}}
\newcommand\ctx{\Gamma}
\newcommand\tinf{\vdash}
\newcommand\Ginf[3][]{\ctx #1\tinf #2 : #3}
\newcommand\subst[2]{#1{}[#2]}
\newcommand\Op{\mathsf{Op}}
\newcommand\cnst{\underline{c}}
\newcommand\sigmoid{\varsigma}
\newcommand\nat{\mathbf{nat}}
\newcommand\reals{\mathbf{R}}
\newcommand\bool{\mathbf{bool}}
\newcommand{\sPair}[2]{( #1, #2 )}
\newcommand{\sTriple}[3]{(#1, #2, #3)}
\newcommand{\sTuple}[1]{(#1)}
\newcommand\tUnit{\bp{}}
\newcommand\tTriple[3]{\langle #1, #2, #3\rangle}
\newcommand\tTuple[1]{\langle #1\rangle}
\newcommand\bProd[2]{\bp{#1 \t* #2}}
\newcommand\tProd[3]{\bp{#1 \t* #2 \t* #3}}

\newcommand\syncat[1]{\mspace{-25mu}\synname{#1}}
\newcommand\synname[1]{\qquad\text{#1}}
\newenvironment{syntax}[1][]{%
\(
  \begin{array}[t]{#1l@{\quad\!\!}*3{l@{}}@{\,}l}
}{
\end{array}
\)%
}

\newcommand\gdefinedby{::=}
\newcommand\gor{\mathrel{\lvert}}

\newcommand\TyDinter[2]{\mathsf{TyD}_{#1}(#2)}
\newcommand\TyD[1]{\mathsf{TyD}(#1)}

\newcommand{\NN}{\mathbb{N}}
\newcommand{\BB}{\mathbb{B}}
\newcommand{\RR}{\mathbb{R}}
\newcommand\ifelse[3]{\mathbf{if}\,#1\,\mathbf{then}\,#2\,\mathbf{else}\,#3\,}
\newcommand\ArrayMake[2]{\vbuildk\,#1\,#2}
\newcommand\ArrayGet[2]{\vgetk\,#1\,#2}
\newcommand\ArrayFold[3]{\vifoldk\,#1\,#2\,#3}
\newcommand\ArrayLength[1]{\vlengthk\,#1}
\newcommand\TensorMake[2]{\boxplus_{i=0}^{#1}#2}
\newcommand\TensorSum[2]{\sum_{i=0}^{#1}#2}
\newcommand\TensorAccess[2]{#1{[}#2{]}}
\newcommand\TensorPred[2]{{[}#1{]}#2}

\newcommand\viteratek{\mathbf{ifold}}
\newcommand\vifoldk{\viteratek}
\newcommand\vbuildk{\mathbf{build}}
\newcommand\vlengthk{\mathbf{length}}
\newcommand\vgetk{\mathbf{get}}
\newcommand\ArraySym{\mathbf{A}}
\newcommand\Array[1]{\ArraySym[#1]}
\newcommand\To{\to}
\newcommand\tMatch[4][\,]{\mathbf{case}\,#2\,\mathbf{of}#1\tTuple{#3}\To#4}
\newcommand{\sem}[1]{\llbracket #1\rrbracket}
\newcommand{\semu}[1]{\llbracket #1\rrbracket_\mathbf{U}}
\newcommand{\grammarcomment}[1]{\textit{\small #1}}

\newcommand{\Source}{\mathbf{Source}}
\newcommand{\Target}{\mathbf{Target}}

\newcommand\Dtype{\mathbf{D}}

%\newcommand\Set{\mathbf{Set}}

\usepackage{bussproofs} 
\usepackage{boxedminipage}

\newenvironment{framed}[0]{\begin{boxedminipage}{\linewidth}}{\end{boxedminipage}}

\title{Categorical semantics for UNF}
\author{Mathieu Huot}

\begin{document}
{\large
\texttt{Begun Mai 2021. Draft of \today, \currenttime}
}
\begingroup
\let\newpage\relax
\maketitle
\endgroup

\section{Recall UNF}

\begin{figure}[H]
\setlength{\tabcolsep}{0.3em}
\centering
\begin{tabular}{|l c l|l|}
\hline
\multicolumn{3}{|c|}{\textbf{Core Grammar}} & \multicolumn{1}{c|}{\textbf{Description}}\\\hline
!T! & \mbox{::=} & $\reals$ & \grammarcomment{Real Type} \\
& $\mid$ & !T! $\times$ !T! & \grammarcomment{Product Type}\\
& $\mid$ & $\Array{\reals}$ & \grammarcomment{Real Array Type}\\
\hline
!e! & \mbox{::=} & !x! & \grammarcomment{Variable}\\
& $\mid$ & !c! & \grammarcomment{Real constant}\\
& $\mid$ & !let x = e in e! & \grammarcomment{Variable Binding}\\
& $\mid$ & !< e, e >! $\mid$ $\pi_1$(!e) $\mid$ $\pi_2$(!e) & \grammarcomment{Pair Constructor/Destructor}\\
& $\mid$ & !e op2 e! $\mid$ !op1 e! & \grammarcomment{Binary/Unary operations}\\
& $\mid$ & !map (x.e) e! $\mid$ !map2 (x,y.e) e e! & \grammarcomment{Array map and map2}\\
& $\mid$ & !foldl (x,y.e) e e! & \grammarcomment{Array fold left}\\
\hline
\end{tabular}
\vspace{-0.2cm}
\caption{Grammar of the source language.}
\label{fig:source_grammar}
\end{figure}

\begin{figure}[H]
    \setlength{\tabcolsep}{0.3em}
    \centering
    \begin{tabular}{|l c l|l|}
    \hline
    \multicolumn{3}{|c|}{\textbf{Core Grammar}} & \multicolumn{1}{c|}{\textbf{Description}}\\\hline
    !T! & \mbox{::=} & $\ldots$ & \grammarcomment{Same as source} \\
    \hline
    !e! & \mbox{::=} & !var!$_{\Gamma,i}$ & \grammarcomment{Variable}\\
    & $\mid$ & !op!$_{\Gamma,n}$ & \grammarcomment{Operations, for $0\leq n\leq 2$}\\
    & $\mid$ & !e;e! & \grammarcomment{Sequential composition}\\
    & $\mid$ & !map!$_{\Gamma,x.e}$ $\mid$ !map2!$_{\Gamma,x,y.e}$ & \grammarcomment{Map and map2}\\
    & $\mid$ & !foldl!$_{\Gamma,x,y.e}$ & \grammarcomment{Fold left}\\
    \hline
    \end{tabular}
    \vspace{-0.2cm}
    \caption{Grammar of the source UNF}
    \label{fig:unf_source_grammar}
    \end{figure}

\begin{figure}[H]
    \begin{tabular}{r c l}
    $\UNFSymbol$($\Gamma\vdash $!c!) &=& !c!$_{\Gamma,0}$ constant seen as a 0-ary operator\\
    $\UNFSymbol$($\Gamma\vdash $!x!) &=& !var!$_{\Gamma,i}$ where !x! is the $i$-th variable in $\Gamma$ \\
    $\UNFSymbol$($\Gamma\vdash $ !let x = e1 in e2!) &=& $\UNFSymbol$(!e1!)$\widehat{;}$ $\UNFSymbol$(!e2!) \\ 
    $\UNFSymbol$($\Gamma\vdash $ !< e1, e2 >!: !T1!$\times$!T2!) &=& $\UNFSymbol$(!e1!)$\widehat{;}$ $\UNFSymbol$(!e2!) $\widehat{;}$ $\pi_{T1\times T2}$\\ 
    $\UNFSymbol$($\Gamma\vdash \pi_i$(!e!)) &=& $\UNFSymbol$(!e!)$\widehat{;}$ $\pi_i$ seen as a unary operator\\
    $\UNFSymbol$($\Gamma\vdash $ !e1 op2 e2!) &=& $\UNFSymbol$(!e1!)$\widehat{;}$ $\UNFSymbol$(!e2!)$\widehat{;}$ !op!$_{\Gamma,2}$\\
    $\UNFSymbol$($\Gamma\vdash $ !op1 e!) &=& $\UNFSymbol$(!e!) $\widehat{;}$ !op!$_{\Gamma,1}$ \\
    $\UNFSymbol$($\Gamma\vdash $ !map (x.e1) e2!) &=& $\UNFSymbol$(!e2!)$\widehat{;}$ !map!$_{\Gamma,x.e}$\\
    $\UNFSymbol$($\Gamma\vdash $ !map2 (x,y.e1) e2 e3!) &=& $\UNFSymbol$(!e2!)$\widehat{;}$ $\UNFSymbol$(!e3!)$\widehat{;}$ !map2!$_{\Gamma, x,y.e}$ \\ 
    $\UNFSymbol$($\Gamma\vdash $ !foldl (x,y.e1) e2 e3!) &=& $\UNFSymbol$(!e2!)$\widehat{;}$ $\UNFSymbol$(!e3!)$\widehat{;}$ !foldl!$_{\Gamma, x,y.e}$ \\  
    \end{tabular}
    \caption{UNF transformation from Source to Source UNF}
    \label{fig:source_to_unf}
    \end{figure}

$\widehat{;}$ is similar to $;$, with a sort of context propagation effect. 
It is a bit reminiscent of monadic composition versus the usual composition.

Formally, if $\Gamma\vdash e1: A$, $\Gamma\vdash \UNFSymbol(e1): \Gamma\times\Delta_1\times A$, $\Gamma\vdash e2: B$, $\Gamma\vdash \UNFSymbol(e2): \Gamma\times\Delta_2\times B$, then $\Gamma\vdash \UNFSymbol(e1) \widehat{;} \UNFSymbol(e2): \Gamma\times \Delta_1\times \Delta_2 \times A \times B$.
This is for most cases above except notably for let binding in which case it is: 
 $\Gamma\vdash \UNFSymbol(e1) \widehat{;} \UNFSymbol(e2): \Gamma\times A \times \Delta_1\times \Delta_2 \times B$. 

 These types $\Gamma\times A \times \Delta_1\times \Delta_2 \times B$ ought to be unordered and represent just a set of variables, thus the extra annoyance with permutations in an ordered setting. 

 Semantically, $\UNFSymbol(e1) \widehat{;} \UNFSymbol(e2)$ is a macro for $\UNFSymbol(e1) ; \UNFSymbol(\widetilde{e2}); \sigma$ where $\Gamma,\Delta_1,A\vdash \widetilde{e2}:B$ is the weakening of $e2$, and $\sigma: \Gamma\times\Delta_1\times A \times \Delta_2 \times B \to \Gamma\times\Delta_1\times \Delta_2 \times A\times B$ is the obvious permutation with this type.
The weakening and permutation are slightly changed in the obvious way for let binding to make the thing typecheck. 


 \MH{I probably need to add permutations $\sigma$ in the language of UNF then.}


\section{Semantics attempts}

Because UNF inspects the code of its agument, it is not invariant under $\beta,\eta$ and so cannot be directly expressed as a structure preserving functor from the syntactic category of the source language modulo $\beta,\eta$ to another category. 

\subsection{Syntactic source category}

Let $\Syn$ be the syntactic category of the source language. Its objects are types of source, and a morphism $\tau \to \sigma$ is a well-typed term of source, modulo only alpha-renaming. Composition is by the usual capture-free substitution. It is a free Cartesian category (+ some List-like structure for arrays, map, and fold) where we do not impose any arithmetic equalities, for instance !cos!(\underline{7}) $\neq$ $\underline{cos(7)}$ in the category.


\subsection{The Poly category} 

We define the bicategory $\Polysyn$ with objects those of $\Syn$, $\Polysyn(A,B)=\cup_{P\in\Syn}\Syn(A,P\times B)$, and there is a 2-cell from $f:A\to P_1\times B$ to $g:A\to P_2\times B$ if there is an epimorphism $p:P_1\to P_2$ in $\Syn$ such that $g=f;id\times p$.
Composition of $f:A\to P_1\times B$ and $g:B\to P_2\times C$ in $\Polysyn$ is given by $f;P_1\times g;\sigma$ in $\Syn$, where $\sigma:P_1\times(P_2\times C)\to (P_1\times P_2)\times C$ is the obvious isomorphism.

Composition of 1-cells is strictly associoative, but the identity $f;id\cong f\cong id;f$ is not strict as $(1\times P)\times A\cong P\times A$ is usually not the identity.


\subsection{$\UNFSymbol$ categorically} 

The nice thing with $\Polysyn$ is that its composition reflects the weakening part of $\widehat{;}$. In addition, the hiding of $P$ in $\Polysyn(A,B)=\cup_{P\in\Syn}\Syn(A,P\times B)$ reflects the type $\Delta$ appearing from $\UNFSymbol$ obtained by collecting bound variables in the term. 

The problem is that the hidden state $P$ cannot be accessed anymore. This made me fail defining $\UNFSymbol:\Syn\to\Polysyn$ as a functor.


My attempt was to define $\UNFSymbol:\Syn\to\Polysyn$ as follows. 
\begin{itemize}
	\item $\UNF{u_A} =\langle id_A,u_A\rangle$
	\item $\UNF{id_A} = \langle u_A,id_A \rangle$
	\item $\UNF{\Delta_A} = \langle u_A,\Delta_A \rangle$
	\item $\UNF{\pi_1} = \langle pi_2, \pi_1 \rangle$
	\item $\UNF{op_2:\RR^2\to\RR} = \langle id_{\RR^2},op_2\rangle$ 
\end{itemize}
where $u_A$ is the unique map $A\to 1$.

The problem comes from the fact that asking $\UNFSymbol$ to be a functor, because it means it should be defined on $\Delta$. The semantics of !let e1 in e2! which is $\Delta;(id\times\sem{e1});\sem{e2}$ is sent to something like $\UNF{\Delta};(id\times \UNF{\sem{e1}});\UNF{\sem{e2}}$, which is not of the form I would like, i.e. $\UNF{\sem{e1}};\UNF{\sem{e2}}$. 
With the semantics above, everything is duplicated too much. For instance, the semantics of  !let y=z in cos(y)!  via this UNF would have type $1,\Gamma\to \Gamma\times (\Gamma-A)\times A, A$.


In practice, representing terms with string diagrams, I want to put a box around $e1$ and $\Delta$ and call this $\UNF{e1}$, whereas the categorical semantics forces me to draw a box around each component separately. 

\MH{another possible way to look at it is that UNF does some sort of localization, turning every operator to a reversible one, which is simply done by making it also return its context. I wonder if it could simply be seen as the localization of $!$. Assuming $1\times A=A$, we have $\Delta;id\times !=id$, so as $!$ would become invertible, so would be $\Delta$. This is sort of what I need in practice for reverse mode: not seeing any $\Delta$, only unary maps, and an iso can always be seen as such in this context, i.e. by making $\Delta$ an iso, it becomes essentially invisible to reverse mode. One obvious problem is that I don't know if I want all types to be isomorphic to 1.}


\end{document}