\section{Simple reverse-mode differentiation}
\label{sec:simplediff}

\subsection{Source Language} 

We consider a standard call-by-value language. 
It is a first-order language with array constructs and a few typical second-order array operations. 
The types !T1,T2! and terms !e1,e2! are in Figure~\ref{fig:source_grammar}.

\begin{figure*}[t]
\setlength{\tabcolsep}{0.3em}
\centering
\begin{tabular}{|l c l|l|}
\hline
\multicolumn{3}{|c|}{\textbf{Core Grammar}} & \multicolumn{1}{c|}{\textbf{Description}}\\\hline
!T! & \mbox{::=} & $\reals$ & \grammarcomment{Real Type} \\
& $\mid$ & !T! $\times$ !T! & \grammarcomment{Product Type}\\
& $\mid$ & $\Array{\reals}$ & \grammarcomment{Real Array Type}\\
\hline
!e! & \mbox{::=} & !x! & \grammarcomment{Variable}\\
& $\mid$ & !c! & \grammarcomment{Real constant}\\
& $\mid$ & !let x = e in e! & \grammarcomment{Variable Binding}\\
& $\mid$ & !< e, e >! $\mid$ $\pi_1$(!e) $\mid$ $\pi_2$(!e) & \grammarcomment{Pair Constructor/Destructor}\\
& $\mid$ & !e op2 e! $\mid$ !op1 e! & \grammarcomment{Binary/Unary operations}\\
& $\mid$ & !map (x.e) e! $\mid$ !map2 (x,y.e) e e! & \grammarcomment{Array map and map2}\\
& $\mid$ & !foldl (x,y.e) e e! & \grammarcomment{Array fold left}\\
\hline
\end{tabular}
\vspace{-0.2cm}
\caption{Grammar of the source language.}
\label{fig:source_grammar}
\end{figure*}


The typing rules are in Figure~\ref{fig:source_typesystem}. We have included a minimal set of array operations for the sake of illustration, 
but it is not difficult to add further operations. For scalar operations, we assume given a set of operations. 
!op1! and !op2! denote respectively unary and binary smooth operations on reals. 
We use infix notation for binary operators.
Typical examples include !cos, exp, +, *!. 
For the sake of simplicity in the development, the function in the array operations !map, map2, foldl! are quite restricted with and free variables. 
We also restrict to arrays of reals.
The general case is also useful and presented in Section~\ref{sec:generalisation}.

\begin{figure*}[tb]
    \centering
    \begin{tabular}{c} 
    \\\hline
    $\Gamma \vdash$ !x!: !T!
    \end{tabular}(!x!: !T!$\in\Gamma$)
    \hspace{0.5cm}
    \begin{tabular}{c} 
        \\\hline
        $\Gamma \vdash$ !c!: $\reals$
    \end{tabular}
    \hspace{0.5cm}
    \begin{tabular}{c}
    $\Gamma \vdash$ !e$_1$!: !T$_1$! $\quad$ $\Gamma \vdash$ !e$_2$!: !T$_2$! \\\hline  
    $\Gamma \vdash$ !<e$_1$,e$_2$>!: !T$_1$! $\times$ !T$_2$!
    \end{tabular}
    \hspace{0.5cm}
    \begin{tabular}{c}
        $\Gamma \vdash$ !e!: !T$_1$! $\times$ !T$_2$! \\\hline  
        $\Gamma \vdash$ $\pi_i$!e!: !T$_i$!
    \end{tabular}($i\in\{1,2\}$)

    \begin{tabular}{c}
    $\Gamma \vdash$ !e$_1$!: !T$_1$! $\quad$ $\Gamma$, !x!: !T$_1$! $\vdash$ !e$_2$!: !T$_2$! \\\hline
    !let x = e$_1$ in e$_2$!: !T$_2$!
    \end{tabular}
    \hspace{0.5cm}
    \begin{tabular}{c}
        $\Gamma \vdash$ !e!: $\reals$ \\\hline  
        $\Gamma \vdash$ !op1 e!: $\reals$
    \end{tabular}
    \hspace{0.5cm}
    \begin{tabular}{c}
        $\Gamma \vdash$ !e$_1$!: $\reals$ $\quad$ $\Gamma \vdash$ !e$_2$!: $\reals$ \\\hline  
        $\Gamma \vdash$ !e$_1$ op2 e$_2$!: $\reals$
        \end{tabular}
 
    \begin{tabular}{c}
        $\Gamma$, !x!: $\reals$, !y!: $\reals$ $\vdash$ !e$_1$!: $\reals$ 
        $\quad$ $\Gamma$ $\vdash$ !e$_2$!: $\Array{\reals}$
        $\quad$ $\Gamma$ $\vdash$ !e$_3$!: $\Array{\reals}$
        \\\hline  
        $\Gamma \vdash$ !map2 (x,y.e$_1$) e$_2$ e$_3$!: $\Array{\reals}$
    \end{tabular}

    \begin{tabular}{c}
        !x!: $\reals$, !y!: $\reals$ $\vdash$ !e$_1$!: $\reals$ 
        $\quad$ $\Gamma$ $\vdash$ !e$_2$!: $\reals$
        $\quad$ $\Gamma$ $\vdash$ !e$_3$!: $\Array{\reals}$
        \\\hline  
        $\Gamma \vdash$ !reduce (x,y.e$_1$) e$_2$ e$_3$!: $\reals$
    \end{tabular}
    \vspace{-0.2cm}
    \caption{Type system of the source language}
    \vspace{-0.4cm}
    \label{fig:source_typesystem}
    \end{figure*}

\subsection{Source Unary Form} 

Following the intuition higlighted in Section~\ref{subsec:insights}, we present a new language which we call unary form (UNF). 
It simply consists of a composition of unary operators. That said, we want to compile our source language to this intermediate representation, 
and we need to remember some information about the initial term. 

The grammar of our source UNF is given in Figure~\ref{fig:unf_source_grammar}. 

\begin{figure*}[t]
    \setlength{\tabcolsep}{0.3em}
    \centering
    \begin{tabular}{|l c l|l|}
    \hline
    \multicolumn{3}{|c|}{\textbf{Core Grammar}} & \multicolumn{1}{c|}{\textbf{Description}}\\\hline
    !T! & \mbox{::=} & ![A1,$\ldots$,An]! & \grammarcomment{Lists of types from source} \\
    \hline
    !e! & \mbox{::=} & !var!$_{T;i}$ & \grammarcomment{Variable}\\
    & $\mid$ & !op!$_{T;n}$ & \grammarcomment{Operations, for $0\leq n\leq 2$}\\
    & $\mid$ & !pair!$_{T;A\times B}$ & \grammarcomment{Pairing a pair of variables}\\
    & $\mid$ & !proj!$_{T_1;T_2;T_3}$  & \grammarcomment{Projection}\\
    & $\mid$ & !e!$\comp$!e! & \grammarcomment{Sequential composition}\\
    & $\mid$ & !map2!$_{T;x,y.e}$ & \grammarcomment{Map2}\\
    & $\mid$ & !reduce!$_{T;x,y.e;e}$ & \grammarcomment{Reduce}\\
    \hline
    \end{tabular}
    \vspace{-0.2cm}
    \caption{Grammar of the source UNF}
    \label{fig:unf_source_grammar}
    \end{figure*}

There are a few notable things in this syntax. 
Every term is indexed by a well-defined contex $\Gamma$ from the source language.
In addition, a variable has a integer index $1\leq i\leq n$ where $n$ is the size of the context $\Gamma$.
Every constant, unary, or binary operator is summarised as an $n$-ary operator !op!$_n$.
Sequential composition is denoted by !;! and !e1;e2! means that !e1! should be performed, and then !e2!.
Every array operator !map!, !map2!, !foldl! has an extra index which represents a well-formed term in the source language.

The types are the same as in the source language. 
A context $\Gamma=\{$!x1: T1,...,xn:Tn!$\}$ can always be seen as a type !T1!$\times$...$\times$!Tn!.
UNF is very syntactic and does not really use variables. 
In fact it consists only of constants and composition. 
For this reason, we denote contexts as types in the indices of the terms, and $\times$ is used for context extension.
But we do emphasize that we mean contexts and the actual variables are remembered and
this will be key when compiling to the target language after differentiation. 

The typing rules are detailed in Figure~\ref{fig:source_unf_typesystem}.
By $\reals^{\times n}$ we mean the product $\reals\times...\times\reals$ of $n$ factors $\reals$.

\MH{need to explain that return values produce some kind of ANF and introduce new vars!}

\begin{figure*}[tb]
    \centering
    \begin{tabular}{c} 
    \\\hline
    $\Gamma \vdash$ !var!$_{\Gamma,i}$: $\Gamma\times$!Ti!
    \end{tabular}($\Gamma=$!T1!$\times\ldots\times$!Tn!)
    \hspace{0.5cm}
    \begin{tabular}{c}
        \\\hline
        $\Gamma\times\reals^{\times n} \vdash$ !op!$_{\Gamma,n}$ : $\Gamma\times\reals^{\times(n+1)}$
    \end{tabular}

    \begin{tabular}{c}
    !T0! $\vdash$ !e1!: !T1! $\quad$ !T1! $\vdash$ !e2!: !T2! \\\hline
    !T0! $\vdash$ !e1; e2!: !T2!
    \end{tabular}
    \hspace{0.5cm}
    \begin{tabular}{c}
        \\\hline  
        $\Gamma \times \Array{\reals} \vdash$ !map!$_{\Gamma, x.e}$: $\Gamma \times \Array{\reals} \times \Array{\reals}$
    \end{tabular}

    \begin{tabular}{c}
        \\\hline  
        $\Gamma \times \Array{\reals} \times \Array{\reals} \vdash$ !map2!$_{\Gamma, x,y.e}$: $\Gamma \times \Array{\reals} \times \Array{\reals} \times \Array{\reals}$
    \end{tabular}

    \begin{tabular}{c}
        \\\hline  
        $\Gamma \times \Array{\reals} \vdash$ !foldl!$_{\Gamma, x,y.e1, e2}$: $\Gamma \times \Array{\reals} \times \reals$
    \end{tabular}
    \vspace{-0.2cm}
    \caption{Type system of the Source UNF}
    \vspace{-0.4cm}
    \label{fig:source_unf_typesystem}
    \end{figure*}

We give a translation from our source language to source UNF in Figure~\ref{fig:source_to_unf}.

\MH{explain kind of monadic ; need theorem saying this makes sense.}

\begin{figure*}[t]
    \begin{tabular}{r c l}
    $\UNFSymbol$($\Gamma\vdash $ !c!) &=& !c!$_{\Gamma,0}$ constant seen as a 0-ary operator\\
    $\UNFSymbol$($\Gamma\vdash $ !x!) &=& !var!$_{\Gamma,i}$ where !x! is the $i$-th variable in $\Gamma$ \\
    $\UNFSymbol$($\Gamma\vdash $ !let x:A = e1 in e2:B!) &=& $\UNFSymbol$(!e1!) $\comp$ $\UNFSymbol$(!e2!) $\comp$ !proj!_${\Gamma;A;B}$  \\ 
    $\UNFSymbol$($\Gamma\vdash $ !< e1, e2 >:AxB!) &=& $\UNFSymbol$(!e1!) $\pcomp$ $\UNFSymbol$(!e2!) $\comp$ !pair!$_{\Gamma,A\times B}$ \\ 
    $\UNFSymbol$($\Gamma\vdash \pi_i$(!e!)) &=& $\UNFSymbol$(!e!)$\comp $$\pi_i$ $\comp$ !proj!_${\Gamma;A_1\times A_2;A_i}$ seen as a unary operator\\
    $\UNFSymbol$($\Gamma\vdash $ !e1 op2 e2!) &=& $\UNFSymbol$(!e1!) $\pcomp$ $\UNFSymbol$(!e2!)$\comp$ !op!$_{\Gamma,2}$ $\comp$ !proj!$_{\Gamma;\reals,\reals;\reals}$ \\
    $\UNFSymbol$($\Gamma\vdash $ !op1 e!) &=& $\UNFSymbol$(!e!) $\comp$ !op!$_{\Gamma,1}$ $\comp$ !proj!$_{\Gamma;\reals;\reals}$ \\
    $\UNFSymbol$($\Gamma\vdash $ !map2 (x,y.e1) e2 e3!) &=& $\UNFSymbol$(!e2!) $\pcomp$ $\UNFSymbol$(!e3!) $\comp$ !map2!$_{\Gamma, x,y.e1}$ $\comp$ !proj!$_{\Gamma;\Array{\reals},\Array{\reals};\Array{\reals}}$ \\ 
    $\UNFSymbol$($\Gamma\vdash $ !reduce (x,y.e1) e2 e3!) &=& $\UNFSymbol$(!e3!)$\comp$ !reduce!$_{\Gamma; x,y.e1; e2}$ $\comp$ !proj!$_{\Gamma;\Array{\reals};\Array{\reals}}$ \\ 
    \end{tabular}
    \caption{UNF transformation from Source to Source UNF}
    \label{fig:source_to_unf}
    \end{figure*}

\subsection{Target Unary Form}

Now our program in UNF is in an equivalent form to stright line programs. 
Before presenting our differentiation macro, we present the target language for this macro. 
We call this intermediate representation target UNF.
The main idea is similar to source UNF where we only have constants and sequential composition.
That said, the key difference is that now every constant represents a pair of an operation and what is meant to represent its Jacobian.
Following the insights from Section~\ref{sec:background}, composition of pairs reverses the order on the second component. T
his represents the fact that reverse-mode reverts the computation flow, 
and that pre-composition (using a continuation) is the simplest way to do this in a purely functional setting.
On pairs, !;! still represents sequential composition for the first component, and reversed composition is represented by $\circ$ on the second component.
The grammar and types are given in Figure~\ref{fig:unf_target_grammar}.
For every constant $C$ from the source UNF, we have a new constant $J^TC$ representing its transpose Jacobian.

\begin{figure*}[t]
    \setlength{\tabcolsep}{0.3em}
    \centering
    \begin{tabular}{|l c l|l|}
    \hline
    \multicolumn{3}{|c|}{\textbf{Core Grammar}} & \multicolumn{1}{c|}{\textbf{Description}}\\\hline
    !T! & \mbox{::=} & ![A1,...,An]! & \grammarcomment{Lists of types from target} \\
    \hline
    !e! & \mbox{::=} & ... & \grammarcomment{Same as source UNF}\\
    & $\mid$ & !J!$^T$!var!$_{T;i}$ & \grammarcomment{Jacobian for variable}\\
    & $\mid$ & !J!$^T$!op!$_{T;n}$ & \grammarcomment{Jacobian for operation, $0\leq n\leq 2$}\\
    & $\mid$ & !J!$^T$!pair!$_{T;A\times B}$ & \grammarcomment{Jacobian for pairing}\\
    & $\mid$ & !J!$^T$!proj!$_{T1;T2;T3}$ & \grammarcomment{Jacobian for projection}\\
    & $\mid$ & !J!$^T$!map2!$_{T;x,y.e}$ & \grammarcomment{Jacobian for map2}\\
    & $\mid$ & !J!$^T$!reduce!$_{T;x,y.e;e}$ & \grammarcomment{Jacobian for reduce}\\
    & $\mid$ & !<e, e>! & \grammarcomment{Term pairing}\\
    & $\mid$ & $\icomp$ & \grammarcomment{Internal function composition}\\
    \hline
    \end{tabular}
    \vspace{-0.2cm}
    \caption{Grammar of the target UNF}
    \label{fig:unf_target_grammar}
\end{figure*}

The typesystem is given in Figure~\ref{fig:target_unf_typesystem}. 
One thing to note is that one can only form pairs. 

\MH{it's currently a bit wrong. need to slightly change to match with UNF-1.}

\begin{figure*}[tb]
    \centering
    \begin{tabular}{c} 
        \\\hline
        !T,A$_i$! $\vdash$ !J!$^T$!var!$_{T;i}$: !T!
        \end{tabular}~(!T=A$_1$,$\ldots$,A$_n$!)
        \hspace{0.5cm}
        \begin{tabular}{c}
            \\\hline
            !T,!$\reals^{\times(n+1)} \vdash$ !J!$^T$!op!$_{T;n}$ : !T,!$\reals^{\times(n)}$
        \end{tabular}~(!T=A$_1$,$\ldots$,A$_n$!)
    
        \begin{tabular}{c}
            \\\hline
            !T,A$\times$B! $\vdash$ !J!$^T$!pair!$_{T;A\times B}$ : !T,A,B!
        \end{tabular}~(!T=A$_1$,$\ldots$,A$_n$!)
    
        \begin{tabular}{c}
            \\\hline
            !T$_1$,T$_3$! $\vdash$ !J!$^T$!proj!$_{T_1;T_2;T_3}$ : !T$_1$,T$_2$,T$_3$!
        \end{tabular}
    
        \begin{tabular}{c}
            !x$_1$:A$_1$,$\ldots$,x$_n$:A$_n$,x:!$\reals$!,y:!$\reals$ $\vdash$ !e!: $\reals$ \quad in Source Language
            \\\hline  
            !T,!$\Array{\reals}{n},\Array{\reals}{n},\Array{\reals}{n} \vdash$ !J!$^T$!map2!$_{T; x,y.e}$: !T,!$\Array{\reals}{n}$,$\Array{\reals}{n}$
        \end{tabular}~(!T=A$_1$,$\ldots$,A$_n$!)
    
        \begin{tabular}{c}
            !x:!$\reals$!,y:!$\reals$ $\vdash$ !e$_1$!: $\reals$ \quad !x$_1$:A$_1$,$\ldots$,x$_n$:A$_n$! $\vdash$ !e$_2$!:$\reals$ \quad in Source Language
            \\\hline  
            !T,!$\Array{\reals}{n},\reals \vdash$ !J!$^T$!reduce!$_{T; x,y.e_1; e_2}$: !T,!$\Array{\reals}{n}$
        \end{tabular}~(!T=A$_1$,$\ldots$,A$_n$!)

        \begin{tabular}{c}
            !T! $\vdash$ !e$_1$: T$_1$!  \quad !T! $\vdash$ !e$_1$: T$_2$!
            \\ \hline
            !T! $\vdash$ !<e$_1$, e$_2$>: T$_1$,T$_2$!
        \end{tabular}
        \hspace{0.5cm}
        \begin{tabular}{c}
            !T1! $\vdash$ e$_1$: [\underline{T$_3$} -> B]  \quad !T$_2$ $\vdash$ e$_2$: T$_3$! 
            \\ \hline
            !T1! $\vdash$ !e$_1$! $\icomp$ !e$_2$!:[\underline{T$_2$}! -> B!]
        \end{tabular}

    \vspace{-0.2cm}
    \caption{Type system of the Target UNF}
    \vspace{-0.4cm}
    \label{fig:target_unf_typesystem}
    \end{figure*}

\subsection{Macro for reverse mode}

\begin{figure*}[t]
\begin{tabular}{|c|}
\hline
    \begin{tabular}{r c l}
    $\Dsynrevsymbol_{\rho}$(!A$_1$,$\ldots$,A$_n$!) &=& !A$_1$,$\ldots$,A$_n$,(A$_1\times\ldots\times$A$_n$)->$\rho$!\\ \\
    $\Dsynrevsymbol_{\rho}$(!var!$_{T;i}$) &=& $F$(!var$_{T;i}$, J$^T$var$_{T;i}$!) \\
    $\Dsynrevsymbol_{\rho}$(!op!$_{\Gamma;n}$) &=& $F$(!op$_{T;n}$,J$^T$op$_{T;n}$!) \\ 
    $\Dsynrevsymbol_{\rho}$(!pair!$_{T;A\times B}$) &=& $F$(!pair$_{T;A\times B}$, J$^T$pair$_{T;A\times B}$!) \\
    $\Dsynrevsymbol_{\rho}$(!proj!$_{T_1;T_2;T_3}$) &=& $F$(!proj$_{T_1;T_2;T_3}$,J$^T$proj$_{T1;T2;T3}$!) \\
    $\Dsynrevsymbol_{\rho}$(!e$_1\comp$e$_2$!) &=& $\Dsynrevsymbol_{\rho}$(!e$_1$!)$\comp$ $\Dsynrevsymbol_{\rho}$(!e$_2$!)\\ 
    $\Dsynrevsymbol_{\rho}$(!map2!$_{T;x,y.e}$) &=& $F$(!map2$_{T;x,y.e}$, J$^T$map2$_{T;x,y.e}$!) \\
    $\Dsynrevsymbol_{\rho}$(!reduce!$_{T;x,y.e_1;e_2}$) &=& $F$(!reduce$_{T;x,y.e_1;e_2}$, J$^T$reduce$_{T;x,y.e_1;e_2}$!) \\
    \end{tabular}\\
    where !$F$(A,B)$\defeq$ <proj$_{T;\underline{T}->\rho;[]} \comp$ A, proj$_{[];T;\underline{T}->\rho} \icomp$ (proj$_{T;\underline{T}->\rho;[]}$ $\comp$ B)>!\\\hline
    \end{tabular}
    \vspace{-0.4cm}
    \caption{Reverse-mode differentiation from Source UNF to Target UNF}
    \label{fig:diff_macro}    
    \vspace{-0.4cm}
\end{figure*}
