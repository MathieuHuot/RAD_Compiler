\section{Simple reverse-mode differentiation}
\label{sec:simplediff}

\subsection{Source Language} 

We consider a standard call-by-value language. 
It is a first-order language with array constructs and a few typical second-order array operations. 
The types !T1,T2! and terms !e1,e2! are in Figure~\ref{fig:source_grammar}.

\begin{figure*}[t]
\setlength{\tabcolsep}{0.3em}
\centering
\begin{tabular}{|l c l|l|}
\hline
\multicolumn{3}{|c|}{\textbf{Core Grammar}} & \multicolumn{1}{c|}{\textbf{Description}}\\\hline
!e! & \mbox{::=} & !sum(x in e) e! & \grammarcomment{Collection Aggregation} \\
% & $\mid$ & !build(x <- e) e! \\
& $\mid$ & !{ e -> e, ... }! & \grammarcomment{Dictionary Construction}\\
% & $\mid$ & ![[ e ]]! & \grammarcomment{Set Construction}\\
% & $\mid$ & !dom(e)! & \grammarcomment{Key Set of Dictionary}\\
& $\mid$ & !e(e)! & \grammarcomment{Dictionary Lookup}\\
& $\mid$ & !< a = e, ... >! & \grammarcomment{Record Construction}\\
& $\mid$ & !e.a! & \grammarcomment{Field Access}\\
& $\mid$ & !let x = e in e! & \grammarcomment{Variable Binding}\\
& $\mid$ & !if(e) then e else e! & \grammarcomment{Conditional}\\
& $\mid$ & !e + e! $\mid$ !e * e! & \grammarcomment{Addition \& Multiplication}\\
& $\mid$ & !n! $\mid$ !r! $\mid$ !false! $\mid$ !true! & \grammarcomment{Numeric \& Boolean Values}\\
& $\mid$ & !dn! & \grammarcomment{Dense Integer Value}\\
% & $\mid$ & !"some_text"! & \grammarcomment{String Literal}\\ 
\hline
!T! & \mbox{::=} & !{ T -> T }! & \grammarcomment{Dictionary Type} \\
& $\mid$ & !< a:T, ... >! & \grammarcomment{Record Type}\\
& $\mid$ & !S! & \grammarcomment{Scalar Type}\\
!S! & \mbox{::=} & !int! $\mid$ !real! $\mid$ !bool! & \grammarcomment{Numeric \& Boolean Types}\\
& $\mid$ & !denseint! & \grammarcomment{Dense Integer Type}\\
% & $\mid$ & !string! & \grammarcomment{String Type}\\
\hline
\end{tabular}
\vspace{0.2cm}
\caption{Grammar of the core part of the language.}
% \vspace{-0.4cm}
\label{fig:lang}
\end{figure*}


The typing rules are in Figure~\ref{fig:source_typesystem}. We have included a minimal set of array operations for the sake of illustration, 
but it is not difficult to add further operations. For scalar operations, we assume given a set of operations. 
!op1! and !op2! denote respectively unary and binary smooth operations on reals. 
We use infix notation for binary operators.
Typical examples include !cos, exp, +, *!. 
For the sake of simplicity in the development, the function in the array operations !map, map2, foldl! are quite restricted with and free variables. 
We also restrict to arrays of reals.
The general case is also useful and presented in Section~\ref{sec:generalisation}.

% \begin{figure*}[t]
%     \centering
%     \begin{tabular}{c}
%     $\Gamma \vdash$ !e1!: !{ T1 -> T2 }! $\quad$ $\Gamma$, !x!: !<key:T1,val:T2>! $\vdash$ !e2!: !T3! \\\hline
%     !sum(x <- e1) e2!: !T3!
%     \end{tabular}
    
%     \begin{tabular}{c}
%     !k_1!: !T1! $\quad$ !v_1!: !T2! $\quad$ ... $\quad$ !k_n!: !T1! $\quad$ !v_n!: !T2!\\\hline
%     !{ k_1 -> v_1, ..., k_n -> v_n }!: !{ T1 -> T2 }!
%     \end{tabular}
%     \hspace{0.5cm}
%     \begin{tabular}{c}
%     !e1!: !{ T1 -> T2 }! $\quad$ !e2!: !T1!\\\hline
%     !e1(e2)!: !T2!
%     \end{tabular}
%     \hspace{0.5cm}
%     \begin{tabular}{c}
%     !e_1!: !T_1! $\quad$ ... $\quad$ !e_n!: !T_k!\\\hline
%     !<a_1=e_1,...,a_k=e_k>!: !<a_1:T_1,...,a_n:T_k>!
%     \end{tabular}
%     \hspace{0.5cm}
%     \begin{tabular}{c}
%     !e!: !<a_1:T_1,...,a_k:T_k>!\\\hline
%     !e.a_i!: !T_i!
%     \end{tabular}
%     \hspace{0.5cm}
%     \begin{tabular}{c}
%     $\Gamma \vdash$ !e1!: !T1! $\quad$ $\Gamma$, !x!: !T1! $\vdash$ !e2!: !T2! \\\hline
%     !let x = e1 in e2!: !T2!
%     \end{tabular}
    
%     \begin{tabular}{c}
%     !e1!: !bool! $\quad$ !e2!: !T! $\quad$ !e3!: !T! \\\hline
%     !if(e1) then e2 else e3!: !T!
%     \end{tabular}
%     \hspace{0.5cm}
%     \begin{tabular}{c}
%     !e1!: !T! $\quad$ !e2!: !T!\\\hline
%     !e1 + e2!: !T!
%     \end{tabular}
%     \hspace{0.5cm}
%     \begin{tabular}{c}
%     !e1!: !T1! $\quad$ !e2!: !T2!\\\hline
%     !e1 * e2!: !T1$\otimes$T2!
%     \end{tabular}
    
    
%     \begin{tabular}{c}
%     !e!: !S1$\otimes$S2!\\\hline
%     !e!: !S1$\vee$S2!
%     \end{tabular}
%     \hspace{0.5cm}
%     \begin{tabular}{c}
%     !e!: !S$\otimes${T1->T2}!\\\hline
%     !e!: !{T1->S$\otimes$T2}!
%     \end{tabular}
%     \hspace{0.5cm}
%     \begin{tabular}{c}
%     !e!: !S$\otimes$<a1:T1, ...>!\\\hline
%     !e!: !<a1:S$\otimes$T1, ...>!
%     \end{tabular}
    
%     \begin{tabular}{c}
%     !e!: !{T1->T2}$\otimes$T0!\\\hline
%     !e!: !{T1->T2$\otimes$T0}!
%     \end{tabular}
%     \hspace{0.5cm}
%     \begin{tabular}{c}
%     !e!: !<a1:T1, ...>$\otimes$T0!\\\hline
%     !e!: !<a1:T1$\otimes$T0, ...>!
%     \end{tabular}
%     \vspace{-0.2cm}
%     \caption{Type system of \lang.}
%     \vspace{-0.4cm}
%     \label{fig:typesystem}
%     \end{figure*}

    \begin{figure}[tb]
        \fbox{
          \parbox{\textwidth}{
      
            \[
                \begin{array}{ll}
                  \inferrule{
                    (\var : \ty) \in \Gamma
                  }{
                    \Gamma \vdash \var\ty
                  }
                &
                 \inferrule{
                    (\var : \ty) \in \Gamma
                  }{
                    \Gamma \vdash \var\ty
                  }
                \end{array} 
                \]
        \caption{Type system}
        \label{fig:type_system}}
        \end{figure}

\subsection{Source Unary Form} 

Following the intuition higlighted in Section~\ref{subsec:insights}, we present a new language which we call unary form (UNF). 
It simply consists of a composition of unary operators. That said, we want to compile our source language to this intermediate representation, 
and we need to remember some information about the initial term. 

The grammar of our source UNF is given in Figure~\ref{fig:unf_source_grammar}. 

\begin{figure*}[t]
    \setlength{\tabcolsep}{0.3em}
    \centering
    \begin{tabular}{|l c l|l|}
    \hline
    \multicolumn{3}{|c|}{\textbf{Core Grammar}} & \multicolumn{1}{c|}{\textbf{Description}}\\\hline
    !T! & \mbox{::=} & ... & \grammarcomment{Same as source} \\
    \hline
    !e! & \mbox{::=} & !var!$_{\Gamma,i}$ & \grammarcomment{Variable}\\
    & $\mid$ & !op!$_{\Gamma,n}$ & \grammarcomment{Operations, for $0\leq n\leq 2$}\\
    & $\mid$ & !e;e! & \grammarcomment{Sequential composition}\\
    & $\mid$ & !map!$_{\Gamma,x.e}$ $\mid$ !map2!$_{\Gamma,x,y.e}$ & \grammarcomment{Map and map2}\\
    & $\mid$ & !foldl!$_{\Gamma,x,y.e}$ & \grammarcomment{Fold left}\\
    \hline
    \end{tabular}
    \vspace{-0.2cm}
    \caption{Grammar of the source UNF}
    \label{fig:unf_source_grammar}
    \end{figure*}

There are a few notable things in this syntax. 
Every term is indexed by a well-defined contex $\Gamma$ from the source language.
In addition, a variable has a integer index $1\leq i\leq n$ where $n$ is the size of the context $\Gamma$.
Every constant, unary, or binary operator is summarised as an $n$-ary operator !op!$_n$.
Sequential composition is denoted by !;! and !e1;e2! means that !e1! should be performed, and then !e2!.
Every array operator !map!, !map2!, !foldl! has an extra index which represents a well-formed term in the source language.

The types are the same as in the source language. 
A context $\Gamma=\{$!x1: T1,...,xn:Tn!$\}$ can always be seen as a type !T1!$\times$...$\times$!Tn!.
UNF is very syntactic and does not really use variables. 
In fact it consists only of constants and composition. 
For this reason, we denote contexts as types in the indices of the terms, and $\times$ is used for context extension.
But we do emphasize that we mean contexts and the actual variables are remembered and
this will be key when compiling to the target language after differentiation. 

The typing rules are detailed in Figure~\ref{fig:source_unf_typesystem}.
By $\reals^{\times n}$ we mean the product $\reals\times...\times\reals$ of $n$ factors $\reals$.

\MH{need to explain that return values produce some kind of ANF and introduce new vars!}

\begin{figure*}[tb]
    \centering
    \begin{tabular}{c} 
    \\\hline
    !T(=A1,...,An)! $\vdash$ !var!$_{T;i}$: !A1,...,An,Ai!
    \end{tabular}
    \hspace{0.5cm}
    \begin{tabular}{c}
        \\\hline
        !T(=A1,...,An)!, $\reals^{\times(n)}$,$\vdash$ !op!$_{T;n}$ : !A1,...,An!,$\reals^{\times(n+1)}$
    \end{tabular}

    \begin{tabular}{c}
    !A1,...,An! $\vdash$ !e1!: !A1,...,An,A! $\quad$ !A1,...,An,A! $\vdash$ !e2!: !A1,...,An,A,B! \\\hline
    !A1,...,An! $\vdash$ !e1!$\comp$!e2!: !A1,...,An,A,B!
    \end{tabular}
    \hspace{0.5cm}
    \begin{tabular}{c}
        !A1,...,An! $\vdash$ !e1!: !A1,...,An,A! $\quad$ !A1,...,An! $\vdash$ !e2!: !A1,...,An,B! \\\hline
        !A1,...,An! $\vdash$ !e1!$\pcomp$!e2!: !A1,...,An,A,B!
    \end{tabular}

    \begin{tabular}{c}
        \\\hline
        !T(=A1,...,An),A,B! $\vdash$ !pair!$_{T;A\times B}$ : !A1,...,An,A!$\times$!B!
    \end{tabular}

    \begin{tabular}{c}
        \\\hline
        !T1,A,T2! $\vdash$ !proj!$_{T1;A;T2}$ : !T1,T2!
    \end{tabular}

    \begin{tabular}{c}
        !x1:A1,...,xn:An,x:!$\reals$!,y:!$\reals$ $\vdash$ !e!: $\reals$ in Source Language
        \\\hline  
        !T(=A1,...,An)!, $\Array{\reals},\Array{\reals} \vdash$ !map2!$_{T; x,y.e}$: !A1,...,An,!$\Array{\reals}$,$\Array{\reals}$,$\Array{\reals}$
    \end{tabular}

    \begin{tabular}{c}
        !x:!$\reals$!,y:!$\reals$ $\vdash$ !e!: $\reals$ in Source Language
        \\\hline  
        !T(=A1,...,An)!, $\Array{\reals} \vdash$ !reduce!$_{T; x,y.e1}$: !A1,...,An,!$\Array{\reals}$,$\reals$
    \end{tabular}
    \vspace{-0.2cm}
    \caption{Type system of the Source UNF}
    \vspace{-0.4cm}
    \label{fig:source_unf_typesystem}
\end{figure*}

We give a translation from our source language to source UNF in Figure~\ref{fig:source_to_unf}.

\MH{explain kind of monadic ; need theorem saying this makes sense.}

\begin{figure*}[t]
    \begin{tabular}{r c l}
    $\UNFSymbol$($\Gamma\vdash $!c!) &=& !c!$_{\Gamma,0}$ \text{constant seen as a 0-ary operator}\\
    $\UNFSymbol$($\Gamma\vdash $ !let x = e1 in e2!) &=& $\UNFSymbol$(!e1!)$\widehat{;}$ $\UNFSymbol$(!e2!) \\ 
    $\UNFSymbol$($\Gamma\vdash $ !< e1, e2 >!) &=& $\UNFSymbol$(!e1!)$\widehat{;}$ $\UNFSymbol$(!e2!) \\ 
    $\UNFSymbol$($\Gamma\vdash $ !e1 op2 e2!) &=& $\UNFSymbol$(!e1!)$\widehat{;}$ $\UNFSymbol$(!e2!)$\widehat{;}$ !op!$_{\Gamma,2}$\\
    $\UNFSymbol$($\Gamma\vdash $ !op1 e!) &=& $\UNFSymbol$(!e!) $\widehat{;}$ !op!$_{\Gamma,1}$ \\
    $\UNFSymbol$($\Gamma\vdash $ !map (x.e1) e2!) &=& $\UNFSymbol$(!e2!)$\widehat{;}$ !map!$_{\Gamma,x.e}$\\
    $\UNFSymbol$($\Gamma\vdash $ !map2 (x,y.e1) e2 e3!) &=& $\UNFSymbol$(!e2!)$\widehat{;}$ $\UNFSymbol$(!e3!)$\widehat{;}$ !map2!$_{\Gamma, x,y.e}$ \\ 
    $\UNFSymbol$($\Gamma\vdash $ !foldl (x,y.e1) e2 e3!) &=& $\UNFSymbol$(!e2!)$\widehat{;}$ $\UNFSymbol$(!e3!)$\widehat{;}$ !foldl!$_{\Gamma, x,y.e}$ \\  
    \end{tabular}
    \caption{UNF transformation from Source to Source UNF}
    \label{fig:source_to_unf}
    \end{figure*}

\subsection{Target Unary Form}

\begin{figure*}[t]
    \setlength{\tabcolsep}{0.3em}
    \centering
    \begin{tabular}{|l c l|l|}
    \hline
    \multicolumn{3}{|c|}{\textbf{Core Grammar}} & \multicolumn{1}{c|}{\textbf{Description}}\\\hline
    !T! & \mbox{::=} & ![A$_{1}$,$\ldots$,A$_n$]! & \grammarcomment{Lists of types from target} \\
    \hline
    !e! & \mbox{::=} & $\ldots$ & \grammarcomment{Same as source UNF}\\
    & $\mid$ & !J!$^T$!var!$_{T;i}$ & \grammarcomment{Jacobian for variable}\\
    & $\mid$ & !J!$^T$!op!$_{T;n}$ & \grammarcomment{Jacobian for operation, $0\leq n\leq 2$}\\
    & $\mid$ & !J!$^T$!pair$_{T;A\times B}$! & \grammarcomment{Jacobian for pairing}\\
    & $\mid$ & !J!$^T$!proj$_{T_1;T_2;T_3}$! & \grammarcomment{Jacobian for projection}\\
    & $\mid$ & !J!$^T$!map2!$_{T;x,y.e}$ & \grammarcomment{Jacobian for map2}\\
    & $\mid$ & !J!$^T$!reduce!$_{T;x,y.e;e}$ & \grammarcomment{Jacobian for reduce}\\
    & $\mid$ & !<e, e>! & \grammarcomment{Term pairing}\\
    & $\mid$ & !e! $\icomp$ !e! & \grammarcomment{Internal function composition}\\
    \hline
    \end{tabular} \\ \vspace{0.2cm}
    \begin{tabular}{|c|}
    \hline
    \begin{tabular}{c} 
        \\\hline
        !T,A$_i$! $\vdash$ !J!$^T$!var!$_{T;i}$: !T!
        \end{tabular}~(!T=A$_1$,$\ldots$,A$_n$!)
        \hspace{0.5cm}
        \begin{tabular}{c}
            \\\hline
            !T,!$\reals^{\times(n+1)} \vdash$ !J!$^T$!op!$_{T;n}$ : !T,!$\reals^{\times(n)}$
        \end{tabular}~(!T=A$_1$,$\ldots$,A$_n$!)
\\
        \begin{tabular}{c}
            \\\hline
            !T,A$\times$B! $\vdash$ !J!$^T$!pair!$_{T;A\times B}$ : !T,A,B!
        \end{tabular}~(!T=A$_1$,$\ldots$,A$_n$!)
\\
        \begin{tabular}{c}
            \\\hline
            !T$_1$,T$_3$! $\vdash$ !J!$^T$!proj!$_{T_1;T_2;T_3}$ : !T$_1$,T$_2$,T$_3$!
        \end{tabular}
\\  
        \begin{tabular}{c}
            !x$_1$:A$_1$,$\ldots$,x$_n$:A$_n$,x:!$\reals$!,y:!$\reals$ $\vdash$ !e!: $\reals$ \quad in Source Language
            \\\hline  
            !T,!$\Array{\reals}{n},\Array{\reals}{n},\Array{\reals}{n} \vdash$ !J!$^T$!map2!$_{T; x,y.e}$: !T,!$\Array{\reals}{n}$,$\Array{\reals}{n}$
        \end{tabular}~(!T=A$_1$,$\ldots$,A$_n$!)
\\  
        \begin{tabular}{c}
            !x:!$\reals$!,y:!$\reals$ $\vdash$ !e$_1$!: $\reals$ \quad !x$_1$:A$_1$,$\ldots$,x$_n$:A$_n$! $\vdash$ !e$_2$!:$\reals$ \quad in Source Language
            \\\hline  
            !T,!$\Array{\reals}{n},\reals \vdash$ !J!$^T$!reduce!$_{T; x,y.e_1; e_2}$: !T,!$\Array{\reals}{n}$
        \end{tabular}~(!T=A$_1$,$\ldots$,A$_n$!)
\\
        \begin{tabular}{c}
            !T! $\vdash$ !e$_1$: T$_1$!  \quad !T! $\vdash$ !e$_1$: T$_2$!
            \\ \hline
            !T! $\vdash$ !<e$_1$, e$_2$>: T$_1$,T$_2$!
        \end{tabular}
        \hspace{0.5cm}
        \begin{tabular}{c}
            !T1! $\vdash$ e$_1$: [\underline{T$_3$} -> B]  \quad !T$_2$ $\vdash$ e$_2$: T$_3$! 
            \\ \hline
            !T1! $\vdash$ !e$_1$! $\icomp$ !e$_2$!:[\underline{T$_2$}! -> B!]
        \end{tabular}\\\hline
    \end{tabular}
    \vspace{-0.4cm}
    \caption{Grammar and type system of the target UNF.}
    \label{fig:unf_target_grammar}
    \vspace{-0.4cm}
\end{figure*}

\begin{figure*}[tb]
    \centering
    \begin{tabular}{c} 
    \\\hline
    $\Dtype(\Gamma) \vdash$ !<var!$_{\Gamma,i}$!, J!$^T$!var!$_{\Gamma,i}$!>!: $\Dtype(\Gamma\times$!Ti!$)$ 
    \end{tabular}($\Gamma=$!T1!$\times\ldots\times$!Tn!)
    \hspace{0.5cm}
    \begin{tabular}{c}
        \\\hline
        $\Dtype(\Gamma\times\reals^{\times n}) \vdash$ !op!$_{\Gamma,n}$ : $\Dtype(\Gamma\times\reals^{\times(n+1)})$
    \end{tabular}

    \begin{tabular}{c}
    $\Dtype$(!T0!) $\vdash$ !<e1, e2>!: $\Dtype$(!T1!) $\quad$ $\Dtype$(!T1!) $\vdash$ !<e3, e4>!: $\Dtype$(!T2!) \\\hline
    $\Dtype$(!T0!) $\vdash$ !<e1, e2>; <e3, e4>!: $\Dtype$(!T2!)
    \end{tabular}

    \begin{tabular}{c}
        \\\hline  
        $\Dtype(\Gamma \times \Array{\reals}) \vdash$ !<map!$_{\Gamma,x.e}$!, J!$^T$!map!$_{\Gamma,x.e}$!>!: $\Dtype(\Gamma \times \Array{\reals} \times \Array{\reals})$
    \end{tabular}

    \begin{tabular}{c}
        \\\hline  
        $\Dtype(\Gamma \times \Array{\reals} \times \Array{\reals})\vdash$ !<map2!$_{\Gamma,x,y.e}$!, J!$^T$!map2!$_{\Gamma,x,y.e}$!>!: $\Dtype(\Gamma \times \Array{\reals} \times \Array{\reals} \times \Array{\reals})$
    \end{tabular}

    \begin{tabular}{c}
        \\\hline  
        $\Dtype(\Gamma \times \reals\times\Array{\reals})\vdash$ !<foldl!$_{\Gamma,x,y.e}$!, J!$^T$!foldl!$_{\Gamma,x,y.e}$!>!: $\Dtype(\Gamma \times \reals\times \Array{\reals} \times \reals)$
    \end{tabular}
    \vspace{-0.2cm}
    \caption{Type system of the Target UNF}
    \vspace{-0.4cm}
    \label{fig:target_unf_typesystem}
    \end{figure*}

\subsection{Macro for reverse mode}

\begin{figure*}[t]
\begin{tabular}{|c|}
\hline
    \begin{tabular}{r c l}
    $\Dsynrevsymbol_{\rho}$(!A$_1$,$\ldots$,A$_n$!) &=& !A$_1$,$\ldots$,A$_n$,(A$_1\times\ldots\times$A$_n$)->$\rho$!\\ \\
    $\Dsynrevsymbol_{\rho}$(!var!$_{T;i}$) &=& $F$(!var$_{T;i}$, J$^T$var$_{T;i}$!) \\
    $\Dsynrevsymbol_{\rho}$(!op!$_{\Gamma;n}$) &=& $F$(!op$_{T;n}$,J$^T$op$_{T;n}$!) \\ 
    $\Dsynrevsymbol_{\rho}$(!pair!$_{T;A\times B}$) &=& $F$(!pair$_{T;A\times B}$, J$^T$pair$_{T;A\times B}$!) \\
    $\Dsynrevsymbol_{\rho}$(!proj!$_{T_1;T_2;T_3}$) &=& $F$(!proj$_{T_1;T_2;T_3}$,J$^T$proj$_{T1;T2;T3}$!) \\
    $\Dsynrevsymbol_{\rho}$(!e$_1\comp$e$_2$!) &=& $\Dsynrevsymbol_{\rho}$(!e$_1$!)$\comp$ $\Dsynrevsymbol_{\rho}$(!e$_2$!)\\ 
    $\Dsynrevsymbol_{\rho}$(!map2!$_{T;x,y.e}$) &=& $F$(!map2$_{T;x,y.e}$, J$^T$map2$_{T;x,y.e}$!) \\
    $\Dsynrevsymbol_{\rho}$(!reduce!$_{T;x,y.e_1;e_2}$) &=& $F$(!reduce$_{T;x,y.e_1;e_2}$, J$^T$reduce$_{T;x,y.e_1;e_2}$!) \\
    \end{tabular}\\
    where !$F$(A,B)$\defeq$ <proj$_{T;\underline{T}->\rho;[]} \comp$ A, proj$_{[];T;\underline{T}->\rho} \icomp$ (proj$_{T;\underline{T}->\rho;[]}$ $\comp$ B)>!\\\hline
    \end{tabular}
    \vspace{-0.4cm}
    \caption{Reverse-mode differentiation from Source UNF to Target UNF}
    \label{fig:diff_macro}    
    \vspace{-0.4cm}
\end{figure*}
