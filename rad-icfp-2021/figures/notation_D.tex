Ground types are defined inductively by 
$$G::= \RR \mid G\times \ldots \times G \mid A[G]$$

$(\RR,+,\underline{0})$ forms a monoid and this monoid structure extends canonically 
to a monoid structure $(G,\widehat{+},0_G)$ for every ground type $G$. 
It is defined inductively on $G$ as follows
\begin{itemize}
    \item $0_\RR \defeq \underline{0}$
    \item $0_{G_1\times \ldots \times G_n}\defeq < O_{G_1},\ldots, 0_{G_n} >$
    \item $0_{A[\RR]_n}\defeq$ ZerosLike($n$) TODO: general array? what about index n?
    \item $a\widehat{+}_\RR b\defeq a+b$
    \item $(a_1,\ldots,a_n)\widehat{+}_{G_1\times\ldots\times G_n}(b_1,\ldots,b_n)\defeq (a_1\widehat{+}_{G_1}b_1,\ldots,a_n\widehat{+}_{G_n}b_n)$
    \item $A\widehat{+}_{A[\RR]}B \defeq$ !map! + $A$ $B$
\end{itemize}

A ground context is a context only containing variables of ground type.
The previous monoid structure again extends canonically on ground contexts $\Gamma$ by defining
$0_{x1:G1,\ldots,x_n:G_n}\defeq 0_{G_1},\ldots,0_{G_n}$ and 
$a_1,\ldots,a_n\widehat{+}_{x1:G_1,\ldots, x_n:G_n}b_1,\ldots,b_n\defeq a_1\widehat{+}_{G_1}b_1,\ldots,a_n\widehat{+}_{G_n}b_n$.

\begin{notation}
We introduce more notation in the table below.\\

\begin{tabular}{|l c l|}
    \hline
    !let x$_{1}$=e$_{1}$,$\ldots$,xn=en!  & \multirow{2}{*}{=} & !let x$_{1}$=e$_{1}$ in let x$_{2}$=e$_{2}$ in $\ldots$! \\
    !in e! && !let x$_n$ = e$_n$ in e!\\ \hline
    $Id_\Gamma$ \quad\quad\quad ($\Gamma \, = \, x_1:A_1,\ldots,x_n:A_n)$ & = & !fun! $(y_1:A_1,\ldots,y_n:A_n)$! -> !$(y_1,\ldots,y_n)$ \\ \hline
    $\grad_\Gamma e$ \quad\quad\hspace{0.6em}($\Gamma\vdash e:\RR$) & = & $\pi_2\directD{\Gamma}{\Gamma}{Id_\Gamma}(e)(0_\Gamma,\underline{1})$ \\ \hline
    !pos(x)! \quad(!x!$\in\Gamma=x_:A_1,\ldots,x_n:A_n)$ & = & position $i$ of !x! in $\Gamma$ \\ \hline
    ![i]e! \quad\quad(!e! of ground type $G_i$) & \multirow{2}{*}{=} &  $(0_{G_1},\ldots,0_{G_{i-1}},e,0_{G_{i+1}},\ldots,0_{G_n})$ \\
    && ($G_j$ are ground types) \\ \hline
    $\grad_{\Gamma_1}$(!e!) \quad\quad($\Gamma=\Gamma_1,\Gamma_2$, $|\Gamma_1|=k$)& = & $(e_1,\ldots,e_k)$ \\
    $\grad_{\Gamma_2}$(!e!) & = & $(e_{k+1},\ldots,e_n)$ \\
    && when $\grad_{\Gamma}($!e!$) = (e_1,\ldots,e_n)$ \\ \hline
    !ZerosLike(A)! & = & !map (x.0) A! \\ \hline
    !OnesLike(A)! & = & !map (x.1) A! \\ \hline
    !ZerosLike(n)! & = & !map (x.0) (x:\Array{\reals}{n})! \\ \hline
    !OnesLike(n)! & = & !map (x.1) (x:\Array{\reals}{n})! \\ \hline
\end{tabular}
TODO: I don't have map in the language, super annoying.
\end{notation}

