\section{UNF}
\label{sec:unf}

Following the intuition highlighted in Section~\ref{subsec:insights}, 
we present a new language which we call unary form (UNF). 
There is a Source UNF for the Source language and a Target UNF for the Target language.
Just as Target is an extension of Source, Target UNF is an extension of Source UNF.

One takeaway from this paper is that efficient pure reverse-mode is complicated because of the several things it does.
It goes through the term and keeps a store for the gradients to be updated. 
It keeps track of the new bound variables found while going through the term.
The updates to the gradient are done via pre-composition and not post-composition.
Due to the linearity requirement in the usage of the continuation variable, 
it needs to process like a call-by-value evaluation and evaluate the arguments of operators in a certain order before 
dealing with the operator itself.

By introducing UNF, we are trying to decouple some of the problems.
UNF introduces the call-by-value evaluation of the arguments before evaluating the arguments and a good management of the environment.
Differentiation on UNF deals with the purely functional store and gradients update via pre-composition.
Finally, going back from UNF to Target uses the stored information in UNF to make sure the Jacobians are computed efficiently. 

\subsection{Source UNF} % (fold)
\label{sub:Source UNF}

Intuitively, a term consists of a composition of unary operators. 
To compile our source language to this intermediate representation, 
we need to remember some information about the context of the initial term. 
The grammar of Source UNF is given in Figure~\ref{fig:unf_source_grammar}. 

\begin{figure*}[t]
    \setlength{\tabcolsep}{0.3em}
    \centering
    \begin{tabular}{|l c l|l|}
    \hline
    \multicolumn{3}{|c|}{\textbf{Core Grammar}} & \multicolumn{1}{c|}{\textbf{Description}}\\\hline
    !T! & \mbox{::=} & ... & \grammarcomment{Same as source} \\
    \hline
    !e! & \mbox{::=} & !var!$_{\Gamma,i}$ & \grammarcomment{Variable}\\
    & $\mid$ & !op!$_{\Gamma,n}$ & \grammarcomment{Operations, for $0\leq n\leq 2$}\\
    & $\mid$ & !e;e! & \grammarcomment{Sequential composition}\\
    & $\mid$ & !map!$_{\Gamma,x.e}$ $\mid$ !map2!$_{\Gamma,x,y.e}$ & \grammarcomment{Map and map2}\\
    & $\mid$ & !foldl!$_{\Gamma,x,y.e}$ & \grammarcomment{Fold left}\\
    \hline
    \end{tabular}
    \vspace{-0.2cm}
    \caption{Grammar of the source UNF}
    \label{fig:unf_source_grammar}
    \end{figure*}

There are a few notable things in this syntax. 
Every primitive is indexed by a list of types from the source language, 
which corresponds to a context being carried. 
Every constant, unary, or binary operator from Source has a corresponding $n$-ary operator !op$_{T,n}$! in UNF.
Sequential composition is denoted by !;! and !e$_{1}$;e$_{2}$! means that !e$_{1}$! should be performed, and then !e$_{2}$!.
The array operators !map2! and !reduce! have extra indices that represent well-formed terms in Source.
The language also contains a projection !proj! which forgets some of the elements, 
and a pairing operator !pair! which takes the last pair of elements and returns the pairing of these elements.

The typing rules are detailed in Figure~\ref{fig:source_unf_typesystem}.
$\reals^{\times n}$ is a notation for $\reals,\ldots,\reals$ with $n$ factors $\reals$.

\begin{figure*}[tb]
    \centering
    \begin{tabular}{c} 
    \\\hline
    !T(=A1,...,An)! $\vdash$ !var!$_{T;i}$: !A1,...,An,Ai!
    \end{tabular}
    \hspace{0.5cm}
    \begin{tabular}{c}
        \\\hline
        !T(=A1,...,An)!, $\reals^{\times(n)}$,$\vdash$ !op!$_{T;n}$ : !A1,...,An!,$\reals^{\times(n+1)}$
    \end{tabular}

    \begin{tabular}{c}
    !A1,...,An! $\vdash$ !e1!: !A1,...,An,A! $\quad$ !A1,...,An,A! $\vdash$ !e2!: !A1,...,An,A,B! \\\hline
    !A1,...,An! $\vdash$ !e1!$\comp$!e2!: !A1,...,An,A,B!
    \end{tabular}
    \hspace{0.5cm}
    \begin{tabular}{c}
        !A1,...,An! $\vdash$ !e1!: !A1,...,An,A! $\quad$ !A1,...,An! $\vdash$ !e2!: !A1,...,An,B! \\\hline
        !A1,...,An! $\vdash$ !e1!$\pcomp$!e2!: !A1,...,An,A,B!
    \end{tabular}

    \begin{tabular}{c}
        \\\hline
        !T(=A1,...,An),A,B! $\vdash$ !pair!$_{T;A\times B}$ : !A1,...,An,A!$\times$!B!
    \end{tabular}

    \begin{tabular}{c}
        \\\hline
        !T1,A,T2! $\vdash$ !proj!$_{T1;A;T2}$ : !T1,T2!
    \end{tabular}

    \begin{tabular}{c}
        !x1:A1,...,xn:An,x:!$\reals$!,y:!$\reals$ $\vdash$ !e!: $\reals$ in Source Language
        \\\hline  
        !T(=A1,...,An)!, $\Array{\reals},\Array{\reals} \vdash$ !map2!$_{T; x,y.e}$: !A1,...,An,!$\Array{\reals}$,$\Array{\reals}$,$\Array{\reals}$
    \end{tabular}

    \begin{tabular}{c}
        !x:!$\reals$!,y:!$\reals$ $\vdash$ !e!: $\reals$ in Source Language
        \\\hline  
        !T(=A1,...,An)!, $\Array{\reals} \vdash$ !reduce!$_{T; x,y.e1}$: !A1,...,An,!$\Array{\reals}$,$\reals$
    \end{tabular}
    \vspace{-0.2cm}
    \caption{Type system of the Source UNF}
    \vspace{-0.4cm}
    \label{fig:source_unf_typesystem}
\end{figure*}

\begin{example}
    !cos$_{\reals,\reals}$ $\comp$ pair$_{\reals,\reals; \reals\times\reals}$! 
    represents !x$_1$:$\reals$, x$_2$:$\reals$, x$_3$:$\reals$ $\vdash$ let (x$_1$,x$_2$,x$_3$,x$_4$)=(x$_1$,x$_2$,x$_3$,cos(x$_3$)) in (x$_1$,x$_2$,<x$_3$,x$_4$>)!.
    $\comp$ can more generally be understood as a let binding of a tuple, and primitives only act on the last parts of the tuple, always returning the context.
\end{example}

\subsection{Target UNF} % (fold)
\label{sub:Target UNF}

The Target UNF is an extension of Source UNF. 
Its types are lists of types of Target. 
Higher-order types are needed for the continuation introduced by reverse-mode. 
In addition, it contains several new primitives which represent the transpose Jacobian of the primitives of Source UNF.
Given a type !T$\defeq$ [A$_1$,$\ldots$,A$_n$]!, we write \underline{T} for the type !A$_1$ $\times\ldots\times$ A$_n$! of Target. 
Target UNF has an internal composition $\icomp$. 
!e$_{1}$ $\icomp$ e$_{2}$! represents precomposition of a curried function !e$_{1}$! by a term !e$_{2}$!.
It is used by the reverse-mode transformation to pre-compose the continuation term by a transpose Jacobian.
The grammar of Target UNF is given in Figure~\ref{fig:unf_source_grammar} and its type system in Figure~\ref{fig:target_unf_typesystem}.

\begin{figure*}[t]
    \setlength{\tabcolsep}{0.3em}
    \centering
    \begin{tabular}{|l c l|l|}
    \hline
    \multicolumn{3}{|c|}{\textbf{Core Grammar}} & \multicolumn{1}{c|}{\textbf{Description}}\\\hline
    !T! & \mbox{::=} & ![A$_{1}$,$\ldots$,A$_n$]! & \grammarcomment{Lists of types from target} \\
    \hline
    !e! & \mbox{::=} & $\ldots$ & \grammarcomment{Same as source UNF}\\
    & $\mid$ & !J!$^T$!var!$_{T;i}$ & \grammarcomment{Jacobian for variable}\\
    & $\mid$ & !J!$^T$!op!$_{T;n}$ & \grammarcomment{Jacobian for operation, $0\leq n\leq 2$}\\
    & $\mid$ & !J!$^T$!pair$_{T;A\times B}$! & \grammarcomment{Jacobian for pairing}\\
    & $\mid$ & !J!$^T$!proj$_{T_1;T_2;T_3}$! & \grammarcomment{Jacobian for projection}\\
    & $\mid$ & !J!$^T$!map2!$_{T;x,y.e}$ & \grammarcomment{Jacobian for map2}\\
    & $\mid$ & !J!$^T$!reduce!$_{T;x,y.e;e}$ & \grammarcomment{Jacobian for reduce}\\
    & $\mid$ & !<e, e>! & \grammarcomment{Term pairing}\\
    & $\mid$ & !e! $\icomp$ !e! & \grammarcomment{Internal function composition}\\
    \hline
    \end{tabular} \\ \vspace{0.2cm}
    \begin{tabular}{|c|}
    \hline
    \begin{tabular}{c} 
        \\\hline
        !T,A$_i$! $\vdash$ !J!$^T$!var!$_{T;i}$: !T!
        \end{tabular}~(!T=A$_1$,$\ldots$,A$_n$!)
        \hspace{0.5cm}
        \begin{tabular}{c}
            \\\hline
            !T,!$\reals^{\times(n+1)} \vdash$ !J!$^T$!op!$_{T;n}$ : !T,!$\reals^{\times(n)}$
        \end{tabular}~(!T=A$_1$,$\ldots$,A$_n$!)
\\
        \begin{tabular}{c}
            \\\hline
            !T,A$\times$B! $\vdash$ !J!$^T$!pair!$_{T;A\times B}$ : !T,A,B!
        \end{tabular}~(!T=A$_1$,$\ldots$,A$_n$!)
\\
        \begin{tabular}{c}
            \\\hline
            !T$_1$,T$_3$! $\vdash$ !J!$^T$!proj!$_{T_1;T_2;T_3}$ : !T$_1$,T$_2$,T$_3$!
        \end{tabular}
\\  
        \begin{tabular}{c}
            !x$_1$:A$_1$,$\ldots$,x$_n$:A$_n$,x:!$\reals$!,y:!$\reals$ $\vdash$ !e!: $\reals$ \quad in Source Language
            \\\hline  
            !T,!$\Array{\reals}{n},\Array{\reals}{n},\Array{\reals}{n} \vdash$ !J!$^T$!map2!$_{T; x,y.e}$: !T,!$\Array{\reals}{n}$,$\Array{\reals}{n}$
        \end{tabular}~(!T=A$_1$,$\ldots$,A$_n$!)
\\  
        \begin{tabular}{c}
            !x:!$\reals$!,y:!$\reals$ $\vdash$ !e$_1$!: $\reals$ \quad !x$_1$:A$_1$,$\ldots$,x$_n$:A$_n$! $\vdash$ !e$_2$!:$\reals$ \quad in Source Language
            \\\hline  
            !T,!$\Array{\reals}{n},\reals \vdash$ !J!$^T$!reduce!$_{T; x,y.e_1; e_2}$: !T,!$\Array{\reals}{n}$
        \end{tabular}~(!T=A$_1$,$\ldots$,A$_n$!)
\\
        \begin{tabular}{c}
            !T! $\vdash$ !e$_1$: T$_1$!  \quad !T! $\vdash$ !e$_1$: T$_2$!
            \\ \hline
            !T! $\vdash$ !<e$_1$, e$_2$>: T$_1$,T$_2$!
        \end{tabular}
        \hspace{0.5cm}
        \begin{tabular}{c}
            !T1! $\vdash$ e$_1$: [\underline{T$_3$} -> B]  \quad !T$_2$ $\vdash$ e$_2$: T$_3$! 
            \\ \hline
            !T1! $\vdash$ !e$_1$! $\icomp$ !e$_2$!:[\underline{T$_2$}! -> B!]
        \end{tabular}\\\hline
    \end{tabular}
    \vspace{-0.4cm}
    \caption{Grammar and type system of the target UNF.}
    \label{fig:unf_target_grammar}
    \vspace{-0.4cm}
\end{figure*}

\begin{figure*}[tb]
    \centering
    \begin{tabular}{c} 
    \\\hline
    $\Dtype(\Gamma) \vdash$ !<var!$_{\Gamma,i}$!, J!$^T$!var!$_{\Gamma,i}$!>!: $\Dtype(\Gamma\times$!Ti!$)$ 
    \end{tabular}($\Gamma=$!T1!$\times\ldots\times$!Tn!)
    \hspace{0.5cm}
    \begin{tabular}{c}
        \\\hline
        $\Dtype(\Gamma\times\reals^{\times n}) \vdash$ !op!$_{\Gamma,n}$ : $\Dtype(\Gamma\times\reals^{\times(n+1)})$
    \end{tabular}

    \begin{tabular}{c}
    $\Dtype$(!T0!) $\vdash$ !<e1, e2>!: $\Dtype$(!T1!) $\quad$ $\Dtype$(!T1!) $\vdash$ !<e3, e4>!: $\Dtype$(!T2!) \\\hline
    $\Dtype$(!T0!) $\vdash$ !<e1, e2>; <e3, e4>!: $\Dtype$(!T2!)
    \end{tabular}

    \begin{tabular}{c}
        \\\hline  
        $\Dtype(\Gamma \times \Array{\reals}) \vdash$ !<map!$_{\Gamma,x.e}$!, J!$^T$!map!$_{\Gamma,x.e}$!>!: $\Dtype(\Gamma \times \Array{\reals} \times \Array{\reals})$
    \end{tabular}

    \begin{tabular}{c}
        \\\hline  
        $\Dtype(\Gamma \times \Array{\reals} \times \Array{\reals})\vdash$ !<map2!$_{\Gamma,x,y.e}$!, J!$^T$!map2!$_{\Gamma,x,y.e}$!>!: $\Dtype(\Gamma \times \Array{\reals} \times \Array{\reals} \times \Array{\reals})$
    \end{tabular}

    \begin{tabular}{c}
        \\\hline  
        $\Dtype(\Gamma \times \reals\times\Array{\reals})\vdash$ !<foldl!$_{\Gamma,x,y.e}$!, J!$^T$!foldl!$_{\Gamma,x,y.e}$!>!: $\Dtype(\Gamma \times \reals\times \Array{\reals} \times \reals)$
    \end{tabular}
    \vspace{-0.2cm}
    \caption{Type system of the Target UNF}
    \vspace{-0.4cm}
    \label{fig:target_unf_typesystem}
    \end{figure*}

\subsection{Simple reverse mode transformation} % (fold)
\label{sub:Simple reverse mode transformation}

We are now able to present a simpler transformation for purely functional reverse-mode from Source UNF to Target UNF.
The differentiation transformation is given in Figure~\ref{fig:diff_macro}.
To a list of types, the transformation adds a higher-order type. 
It is to be understood as the continuation for the pure storage of the gradient.
On primitive constants of Source UNF, it returns a pair. 
The first part is the initial term (pre-composed by a projection because it is not using the continuation variable).
The second part returns the continuation term pre-composed by the transpose Jacobian of the primitive.

\begin{figure*}[t]
\begin{tabular}{|c|}
\hline
    \begin{tabular}{r c l}
    $\Dsynrevsymbol_{\rho}$(!A$_1$,$\ldots$,A$_n$!) &=& !A$_1$,$\ldots$,A$_n$,(A$_1\times\ldots\times$A$_n$)->$\rho$!\\ \\
    $\Dsynrevsymbol_{\rho}$(!var!$_{T;i}$) &=& $F$(!var$_{T;i}$, J$^T$var$_{T;i}$!) \\
    $\Dsynrevsymbol_{\rho}$(!op!$_{\Gamma;n}$) &=& $F$(!op$_{T;n}$,J$^T$op$_{T;n}$!) \\ 
    $\Dsynrevsymbol_{\rho}$(!pair!$_{T;A\times B}$) &=& $F$(!pair$_{T;A\times B}$, J$^T$pair$_{T;A\times B}$!) \\
    $\Dsynrevsymbol_{\rho}$(!proj!$_{T_1;T_2;T_3}$) &=& $F$(!proj$_{T_1;T_2;T_3}$,J$^T$proj$_{T1;T2;T3}$!) \\
    $\Dsynrevsymbol_{\rho}$(!e$_1\comp$e$_2$!) &=& $\Dsynrevsymbol_{\rho}$(!e$_1$!)$\comp$ $\Dsynrevsymbol_{\rho}$(!e$_2$!)\\ 
    $\Dsynrevsymbol_{\rho}$(!map2!$_{T;x,y.e}$) &=& $F$(!map2$_{T;x,y.e}$, J$^T$map2$_{T;x,y.e}$!) \\
    $\Dsynrevsymbol_{\rho}$(!reduce!$_{T;x,y.e_1;e_2}$) &=& $F$(!reduce$_{T;x,y.e_1;e_2}$, J$^T$reduce$_{T;x,y.e_1;e_2}$!) \\
    \end{tabular}\\
    where !$F$(A,B)$\defeq$ <proj$_{T;\underline{T}->\rho;[]} \comp$ A, proj$_{[];T;\underline{T}->\rho} \icomp$ (proj$_{T;\underline{T}->\rho;[]}$ $\comp$ B)>!\\\hline
    \end{tabular}
    \vspace{-0.4cm}
    \caption{Reverse-mode differentiation from Source UNF to Target UNF}
    \label{fig:diff_macro}    
    \vspace{-0.4cm}
\end{figure*}

\subsection{Transformations to and from UNF} % (fold)
\label{sub:transformations to and from UNF}

We first give a translation from our Source to Source UNF in Figure~\ref{fig:source_to_unf}, which we call $\UNFSymbol$.
As discussed above, $\UNFSymbol$ sends a constant or an operator to a primitive in Source UNF. 
This term in Source UNF carries the context of the original term in Source.
$\UNFSymbol$ sequentializes a term, mimicking a left-to-right call-by-value evaluation.
Because of this, we we're trying to return a pair, this needs to be witnessed, and this is the role of !pair!.
When we need to pass the result of type !A! of !e$_{1}$! through !e$_{2}$! which does not take it as an input, we can transform !e$_{2}$! to accept and pass it.
It's the equivalent in UNF of a weakening.
This weakening is required when we transform a non-unary operator such as $+$ or !pair!, as we transform the arguments in order and we need to keep all of their results.
We write this new term $(\widetilde{e_{2}})_{A}$ and it defined by induction on !e$_{2}$! as follows. 
We assume given a type !A! that needs to be passed, and drop the index !A!.

\begin{tabular}{r c l}
    $\widetilde{var_{T;i}}$  &=& $var_{T,A;i}$ \\
    $\widetilde{op_{T;n}}$  &=& $op_{T,A;n}$ \\§
    $\widetilde{pair_{T;B\times C}}$  &=& $pair_{T,A;B\times C}$ \\
    $\widetilde{proj_{T_1;T_2;T_3}}$  &=& $proj_{T_1;T_2;T_3,A}$ \\
    $\widetilde{e_1 \comp e_2}$  &=& $\widetilde{e_1}\comp\widetilde{e_2}$ \\
    $\widetilde{map2_{T;x,y.e}}$  &=& $map2_{T,A;x,y.e}$ \\
    $\widetilde{reduce_{T;x,y.e_1;e_2}}$ &=& $reduce_{T,A;x,y.e_1;e_2}$ \\
\end{tabular}

We want to preserve the invariant that $\UNFSymbol$(!e!) represents the term !e! which is also returning its context. 
Because !x! is not free in $\Gamma \vdash$ !let x=e$_1$in e$_2$!, we need to hide !x! after $\UNFSymbol$(!e$_2$!).
This explain the projection !proj! in $\UNFSymbol$ of a !let!. 
More generally, in !e!$_{1} \comp$ !e!$_{2}$ the return value of !e!$_{1}$ acts as a new variable for !e!$_{2}$, 
is not in the context of !e!$_{1}$, but will be returned by !e!$_{2}$.
This means we need to hide all these intermediate values to preserve the invariant,
and this explains the projection !proj! in the $\UNFSymbol$ of other terms. 

\begin{lemma}[Well typedness $\UNFSymbol$]
    If !$\Gamma$=x$_1$:A$_{1}$,$\ldots$,x$_n$:A$_{n}$ $\vdash$ e:B! then
   !A$_{1}$,$\ldots$,A$_{n}$ $\vdash$ $\UNFSymbol$(e): A$_{1}$,$\ldots$,A$_{n}$,B!.
\end{lemma}

% TODO: we could merge everything of the form op;proj into just op by changing the type of op.

\begin{figure*}[t]
    \begin{tabular}{r c l}
    $\UNFSymbol$($\Gamma\vdash $!c!) &=& !c!$_{\Gamma,0}$ \text{constant seen as a 0-ary operator}\\
    $\UNFSymbol$($\Gamma\vdash $ !let x = e1 in e2!) &=& $\UNFSymbol$(!e1!)$\widehat{;}$ $\UNFSymbol$(!e2!) \\ 
    $\UNFSymbol$($\Gamma\vdash $ !< e1, e2 >!) &=& $\UNFSymbol$(!e1!)$\widehat{;}$ $\UNFSymbol$(!e2!) \\ 
    $\UNFSymbol$($\Gamma\vdash $ !e1 op2 e2!) &=& $\UNFSymbol$(!e1!)$\widehat{;}$ $\UNFSymbol$(!e2!)$\widehat{;}$ !op!$_{\Gamma,2}$\\
    $\UNFSymbol$($\Gamma\vdash $ !op1 e!) &=& $\UNFSymbol$(!e!) $\widehat{;}$ !op!$_{\Gamma,1}$ \\
    $\UNFSymbol$($\Gamma\vdash $ !map (x.e1) e2!) &=& $\UNFSymbol$(!e2!)$\widehat{;}$ !map!$_{\Gamma,x.e}$\\
    $\UNFSymbol$($\Gamma\vdash $ !map2 (x,y.e1) e2 e3!) &=& $\UNFSymbol$(!e2!)$\widehat{;}$ $\UNFSymbol$(!e3!)$\widehat{;}$ !map2!$_{\Gamma, x,y.e}$ \\ 
    $\UNFSymbol$($\Gamma\vdash $ !foldl (x,y.e1) e2 e3!) &=& $\UNFSymbol$(!e2!)$\widehat{;}$ $\UNFSymbol$(!e3!)$\widehat{;}$ !foldl!$_{\Gamma, x,y.e}$ \\  
    \end{tabular}
    \caption{UNF transformation from Source to Source UNF}
    \label{fig:source_to_unf}
    \end{figure*}

Next, the transformation from Target UNF to Target is presented in Figure~\ref{fig:unf_to_target}. 
We call this transformation $\UNFSymbol^{-1}$, but it is not a strict inverse of $\UNFSymbol$.
Doing $\UNFSymbol$ followed by $\UNFSymbol^{-1}$ performs some version of ANF transformation. 
$\UNFSymbol^{-1}$ does not preserve a typing property as nicely as $\UNFSymbol$, 
because it treats primitives with and without a $J$ differently.

Target UNF does not have variables. 
A context type !T=[A$_1$,$\ldots$,A$_n$]! is transformed to a context !$\Gamma$=x$_1$:A$_1$,$\ldots$,x$_n$:A$_n$!.
We use this convention when definiting $\UNFSymbol^{-1}$. 
For operators like !op$_{\Gamma;m}$!, we extend the notation above by saying that the canonical variables are 
!x$_1$:A$_1$,$\ldots$,x$_n$:A$_n$,x$_{n+1}$:$\reals$,$\ldots$,x$_{n+m}$:$\reals$!.

\begin{figure*}[t]
    \begin{tabular}{r c l}
    $\invUNFSymbol$(!var!$_{\Gamma;i}$) &=& !x$_i$: Ti! \\
    && where !x$_i$! is the $i$-th element in the context $\Gamma$ \\
    $\invUNFSymbol$(!op!$_{\Gamma;n}$) &=& !op$_n$(x$_k$,$\ldots$,x$_{k+n}$)! \\
    && where !(x$_k$,$\ldots$,x$_{k+n}$)! are the last $n$ variables of $\Gamma$ \\ 
    $\invUNFSymbol$(!e$_1$ $\comp$ e$_2$!) &=& !let x=$\invUNFSymbol$(e$_1$) in $\invUNFSymbol$(e$_2$)! where !$\Gamma$,x:A $\vdash$ e$_2$:$\Gamma$,A,B! \\ 
    $\invUNFSymbol$(!map2!$_{\Gamma;x,y.e}$) &=& !map2 (x,y.e) X$_1$ X$_2$! where !$\Gamma$,X$_1$,X$_2$ $\vdash$ map2$_{\Gamma;x,y.e}$! \\
    $\invUNFSymbol$(!reduce!$_{\Gamma;x,y.e_1;e_2}$) &=& !reduce (x,y.e$_1$) $e_2$ X! where !$\Gamma$,X $\vdash$ reduce$_{\Gamma;x,y.e_1;e_2}$! \\ 
    $\invUNFSymbol$(!< e$_1$, e$_2$>!) &=& !<$\invUNFSymbol$(e$_1$), $\invUNFSymbol$(e$_2$)>! \\
    $\invUNFSymbol$(!e$_1$!$\pcomp$!e$_2$!) &=& !let x=$\invUNFSymbol$(e$_1$) in $\invUNFSymbol$(e$_2$)! TODO: not sure \\
    $\invUNFSymbol$(!e1!$\icomp$!e2!) &=& TODO: \\
    $\invUNFSymbol$(!proj!) &=& TODO \\
    $\invUNFSymbol$(!pair!) &=& TODO \\
    $\invUNFSymbol$(!J!$^T$!var!$_{\Gamma;i}$) &=& !(x1,$\ldots$,x(i-i),xi+z,x(i+1),$\ldots$,xm)! \\
    $\invUNFSymbol$(!J!$^T$!op!$_{\Gamma;n}$) &=& $(x_1',\ldots,'x_{m-n},x_{m-n+1}+\partial_1op_n*z,\ldots,xm+\partial_nop_n*z)$ \\
    && where $\partial_iop_n$ is the $i$-th partial derivative of $op_n$ \\
    $\invUNFSymbol$(!J!$^T$!proj!) &=& TODO \\
    $\invUNFSymbol$(!J!$^T$!pair!) &=& TODO \\
    $\invUNFSymbol$(!J!$^T$!map2!$_{\Gamma;x,y.e}$) &=&  Y((x1,$\ldots$,xm)+$\nabla_{\Gamma}e *$(!reduce! + 0 Z),\\
    && !map2 (fun (a,b) -> a+!$\nabla_{\{x1\}}e *$!b) X1' Z'!, \\
    && !map2 (fun (a,b) -> a+!$\nabla_{\{x2\}}e *$!b) X2' Z'!) \\
    && where $X1'$, $X2'$ are the tangent arrays for the two arrays arguments \\
    && of !map2! and $Z$ the tangent part for the return array of !map2! \\
    $\invUNFSymbol$(!J!$^T$!reduce!$_{\Gamma;x,y.e1;e2}$) &=& !let A = scanl (fun (x,y) -> e1) e2 e3 in! \\
    && !let B = map2 (fun (x,y) ->! $\nabla_{\{x1\}}$!e1(x,y)) A e3 in!\\
    && !let C = map2 (fun (x,y) ->! $\nabla_{\{x2\}}$!e1(x,y)) A e3) in!\\
    && !let D = scanr (fun (x,y) -> x*y) 1 B in!\\
    && Y(x1,$\ldots$,xm, !map3 (fun (a,b,c) -> a+b*c*z) X' D C!)
    \end{tabular}
    \caption{UNF transformation from Target UNF to Target}
    \label{fig:unf_to_target}
    \end{figure*}
