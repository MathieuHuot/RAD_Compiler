\section{Introduction}
\label{sec:intro}

% % (Source-to-source) Automatic Differentiation (AD) is a technique for transforming code that implements 
% % a function $f$ into code that computes $f$'s derivative, essentially by using the chain rule for derivatives. 
% % Due to its efficiency and numerical stability, AD is the technique of choice whenever derivatives need to be computed 
% % of functions that are implemented as programs, particularly in high dimensional settings. 
% % Optimizations and Monte Carlo integration algorithms, such as Gradient Descent and Hamiltonian Monte-Carlo methods, 
% % rely crucially on the calculation of derivatives. These algorithms are used in virtually everywhere in machine learning 
% % and computational statistics, and the calculation of derivatives is usually the computational bottle-neck. 
% % AD, roughly speaking, comes in two modes; forward-mode and reverse-mode. 
% % When differentiating a function $\RR^n\to\RR^m$, forward-mode tends to be more efficient if $m \gg n$, while reverse-mode 
% % is generally more performant if $n \gg m$. As most applications reduce to optimization of Monte-Carlo 
% % integration of an objective function $\RR^n\to\RR$ with $n$ very large ($10^4-10^7$), reverse-mode AD is 
% % in many ways the more interesting algorithm.

% % %define diff prog beyond AD, be specific with references/different define-then-run vs Pytorch/Swift etc.
% % %Diff Curry has OK intro

% \subsection{Motivation}
Deep learning is moving towards increasingly sophisticated optimization objectives that employ tensors and operations on tensors.
Reverse-mode Automatic Differentiation (AD) is a technique to automatically compute the gradient of objective functions of form $\RR^n\to\RR$.
Such functions appear a lot in practice: for instance, as loss functions in machine learning.

In order to reach the efficiency of the usual imperative version of reverse-mode, the transformations even in functional languages~\cite{pearlmutter2008reverse} resort to references.
The lack of purity in reverse-mode is known to make it significantly harder to optimize and parallelize. 
None of the current implementations of reverse-mode in a functional setting~\cite{lantern_icfp,pearlmutter2008reverse,baydin2016diffsharp} are pure, and often complicated heuristics with no guarantees are used, e.g.~\cite{xla}.
As a result, to optimize for efficiency, a hand-crafted reverse-derivative must be given, for every important non-trivial operation.
In other words, abstracting away from imperative code is still a hurdle that functional implementations need to overcome.

% An efficient purely functional reverse-mode transformation would be more easily explainable, optimizable, and reusable.

% \subsection{Problem}

% Following \cite{pearlmutter2008reverse}, it is relatively easy to define a purely functional reverse-mode on a first-order language using continuations. 
% Yet, as remarked by the authors, this is highly inefficient. 
% The problem we tackle in this paper is how to go from a simple and general setting, where we can relatively easily prove correctness of reverse-mode, 
% to a representation that leads to more efficient implementations.

In this paper, we define a purely functional (without references or control mechanisms such as state monads), denotationally correct, and provably efficient reverse-mode AD. 
To do so, we define the Unary Normal Form (UNF) representation inspired by PROPs \cite{hackney2015category} and compilation to categories \cite{elliott2017compiling}.
We can easily define and prove correctness of reverse-mode on this representation. 
The whole reverse-mode transformation is obtained by compiling the language to this intermediate representation, applying the simpler reverse-mode transformation, and compiling again to the original language.
After standard optimizations, the output program looks like SSA~\cite{cytron1989efficient} or ANF~\cite{sabry1993reasoning}, which leads to more efficient implementations.

\begin{table}
 \label{fig:comparison-table}
 % \begin{tabular}{|l|c|c|c|c|c|c|c|c|}
 % \hline
 %  & \rot{Reverse Mode}  & \rot{Complexity} & \rot{Pure Derivatives} & \rot{Correctness Proofs} & \rot{Tensor Support}  & \rot{Recursion} & \rot{Conditional} \\
 % \hline
 % \system{} (This Paper) &
 % \supfull & \supfull & \supfull & \supfull  & \supfull  & \suphalf & \supfull \\ 
 % \hline
 % Lantern~\cite{lantern_icfp} & 
 % \supfull & \suphalf & \supnone & \supnone & \supfull & \supfull & \supfull \\ 
 % \hline
 % \dfsmooth{}~\cite{shaikhha2019efficient} 
 % & 
 % \supnone & \supnone & \supfull & \supnone & \supfull  & \suphalf  & \supfull \\ 
 % \hline
 % \cite{huot2020correctness} &
 % \supfull & \supnone & \supfull & \supfull & \supnone & \suphalf & \supfull \\ 
 % \hline
 % \cite{brunel2019backpropagation} &
 % \supfull & \suphalf & \supfull & \supfull & \supnone & \supnone & \supnone \\ 
 % \hline
 % \cite{abadi2019simple} &
 % \supfull & \suphalf & \supfull & \supfull & \supnone & \supfull & \supfull \\ 
 % \hline
 % \cite{barthe2020versatility} &
 % \supnone & \supnone & \supfull & \supfull  & \supnone & \supnone & \supfull \\ 
 % \hline
 % \cite{pearlmutter2008reverse} &
 % \supfull & \supfull  & \supnone & \supnone & \supnone & \supfull & \supfull \\ 
 % \hline
 % \cite{Elliott:2018:SEA:3243631.3236765} &
 % \supfull & \supnone & \supfull & \supfull & \supnone & \supnone & \supnone \\ 
 % \hline
 % \cite{sherman2021} & 
 % \supnone & \supnone & \supfull & \supfull & \supnone & \suphalf & \supfull \\ 
 % \hline
 % \cite{vytiniotis2019differentiable} &
 % \supfull & \suphalf & \supfull & \supnone & \supfull & \supnone & \supnone \\ 
 % \hline
 % \cite{mak2020differential} & 
 % \supfull & \supnone & \supfull & \supfull & \supnone & \supnone & \supnone \\ 
 % \hline
 % \cite{vakar2020reverse} & 
 % \supfull & \supnone & \supfull & \supfull & \supfull & \supnone & \supnone \\ 
 % \hline
 % \cite{Manzyuk2012} & 
 % \supnone & \supnone & \supfull & \supfull & \supnone & \supnone & \supnone \\ 
 % \hline 
 % \cite{cockett2019reverse} &
 % \supfull & \supnone & \supfull & \supfull & \supnone & \supnone & \supnone  \\ 
 % \hline
 % \cite{gallagher-sdg}  & 
 % \supnone & \supnone & \supfull & \supfull & \supnone & \supnone & \supnone  \\ 
 % \hline
 \begin{tabular}{|l|c|c|c|c|c|c|c|c|c|c|c|c|c|c|c|c|}
 \hline
  & \rot{This Paper} & \rot{\cite{lantern_icfp}}  & \rot{\cite{shaikhha2019efficient}}  & \rot{\cite{huot2020correctness}}  & \rot{\cite{brunel2019backpropagation}}  & \rot{\cite{abadi2019simple}}  & \rot{\cite{barthe2020versatility}}  & \rot{\cite{pearlmutter2008reverse}}  & \rot{\cite{Elliott:2018:SEA:3243631.3236765}} & \rot{\cite{sherman2021}}  & \rot{\cite{vytiniotis2019differentiable}}  & \rot{\cite{mak2020differential}}  & \rot{\cite{vakar2020reverse}}  & \rot{\cite{Manzyuk2012}}  & \rot{\cite{cockett2019reverse}}  & \rot{\cite{gallagher-sdg}} \\ \hline
Reverse Mode &
\supfull & \supfull & \supnone & \supfull & \supfull & \supfull & \supnone & \supfull & \supfull & \supnone & \supfull & \supfull & \supfull & \supnone & \supfull & \supnone \\ \hline
Complexity &
\supfull & \suphalf & \supnone & \supnone & \suphalf & \suphalf & \supnone & \supfull & \supnone & \supnone & \suphalf & \supnone & \supnone & \supnone & \supnone & \supnone \\ \hline
Pure Derivatives &
\supfull & \supnone & \supfull & \supfull & \supfull & \supfull & \supfull & \supnone & \supfull & \supfull & \supfull & \supfull & \supfull & \supfull & \supfull & \supfull \\ \hline
Correctness Proofs &
\supfull & \supnone & \supnone & \supfull & \supfull & \supfull & \supfull & \supnone & \supfull & \supfull & \supnone & \supfull & \supfull & \supfull & \supfull & \supfull \\ \hline
Tensor Support &
\supfull & \supfull & \supfull & \supnone & \supnone & \supnone & \supnone & \supnone & \supnone & \supnone & \supfull & \supnone & \supfull & \supnone & \supnone & \supnone \\ \hline
Higher-order Functions &
\supnone & \supfull & \supfull & \supfull & \supfull & \supnone & \supfull & \supfull & \supnone & \supfull & \supfull & \supfull & \supfull & \supfull & \supnone & \supfull \\ \hline
Recursion &
\suphalf & \supfull & \suphalf & \suphalf & \supnone & \supfull & \supnone & \supfull & \supnone & \suphalf & \supnone & \supnone & \supnone & \supnone & \supnone & \supnone \\ \hline
Conditional &
\supfull & \supfull & \supfull & \supfull & \supnone & \supfull & \supfull & \supfull & \supnone & \supfull & \supnone & \supnone & \supnone & \supnone & \supnone & \supnone \\ \hline
 \end{tabular}
 \caption{Comparison of different functional differentiable programming frameworks.
 $\supfull$ means that the property is verified, and $\supnone$ means that it is absent in the work.
 $\suphalf$ for complexity means that the proof is not fully covered, and for recursion, that it does not support general recursion.}
 \label{tbl:relwork}
 \end{table}

\subsection{Examples}

We introduce the general idea of efficient reverse-mode in a functional setting the next examples.
% We recall some basics of reverse-mode differentiation in Section~\ref{sec:background}.

\begin{example}[First-order term]
% On simple first-order terms, the output code is reminiscent of SSA code~\cite{cytron1989efficient}.
Let us consider the term !let w$_1$ = x$_1$ * x$_2$ in let w$_2$ = w$_1$ * x$_1$ in w$_2$! in the context $\Gamma:=\{x_1,x_2, x_3:\RR\}$.\\
After an (inefficient) reverse-mode transformation, we obtain:
\begin{center}
    \begin{tabular}{l}
        !let w$_1$,w$_1'$! = !< x$_1$ * x$_2$, fun (y$_1$,$\ldots$, y$_4$) -> (y$_1$+x$_2$*y$_4$, y$_2$+x$_1$*y$_4$, y$_3$)> in!\\
        !let w$_2$,w$_2'$! = !< w$_1$*x$_1$, fun (y$_1$,$\ldots$, y$_5$) -> w$_1'$!!(y$_1$+w$_1$*y$_5$, y$_2$, y$_3$, y$_4$+x$_1$*y$_5$)> in!\\
        !w$_2'$(0,0,0,0,1)!
    \end{tabular}
\end{center}
The part !(0,0,0,0,1)! comes as an equivalent to initialing the tangent variables in the imperative reverse-mode algorithm. 
After some general partial evaluation techniques that will be detailed further in the paper, we obtain:    

    \begin{center}
            \begin{tabular}{l}
                !let w$_1$ = x$_1$ * x$_2$ in!\\ 
                !let w$_2$ = w$_1$ * x$_1$ in!\\
                !let y$_1$,y$_2$,y$_3$,y$_4$,y$_5$ = 0,0,0,0,1 in!\\
                !let y$_1'$! != y$_1$+w$_1$*y$_5$ in!\\
                !let y$_4'$! != y$_4$+x$_1$*y$_5$ in!\\
                !(y$_1'$+x$_2$*y$_4'$!!, y$_2$+x$_1$*y$_4'$!!, y$_3$)!
            \end{tabular}
    \end{center}   
This is very close to the SSA form \cite{cytron1989efficient} of what the imperative reverse-mode differentiation of our initial term would be.
This term can be further optimized via constant propagation and algebraic simplifications to give
        \begin{center}
            \begin{tabular}{{c}}
                !let w$_1$ = x$_1$ * x$_2$ in!\\ 
                !let w$_2$ = w$_1$ * x$_1$ in!\\
                !(w$_1$+x$_2$*x$_1$, x$_1$*x$_1$, 0)!
            \end{tabular}
        \end{center}
    \end{example}

 \begin{example}[Simple operations on arrays]
    On arrays, three simple operations of interest are the dot-product of two vectors, and the product or sum of the elements of a vector.
    In a functional setting, these can be defined as follows:
\begin{center}
    \begin{tabular}{{r c l}}
        !prod(A)! &:=& !reduce * 1 A! \\
        !sum(A)! &:=& !reduce + 0 A! \\
        !dot(A,B)! &:=& !reduce + 0 (map2 * A B)!     
    \end{tabular}
\end{center}
where !reduce! is a known fold-left operator for which the function argument is associative. 
It is notably faster to execute than a fold-left as it is parallel friendly.

The gradient of each of these expressions w.r.t. !A! will be:
\begin{center}
    \begin{tabular}{{r c l}}
        $\nabla_A$!prod(A)! &:=& !map2 * (scanr * 1 (shift1L A)) (shift1R (scanl * 1 A))! \\
        $\nabla_A$!sum(A)! &:=& !map (x -> 1) A!\\
        $\nabla_A$!dot(A,B)! &:=& !B! 
    \end{tabular}
\end{center}
where:
\begin{itemize}
\item !scanl! is the scan-left operator that returns all the intermediate results of fold-left,
\item !scanr! is the scan-right operator that returns all the intermediate results of fold-right,
\item !shift1L [v$_1$,$\ldots$,v$_n$]! is the shift-left operator and returns ![v$_2$,$\ldots$,v$_{n}$]!, and 
\item !shift1R [v$_1$,$\ldots$,v$_n$]! is the shift-right operator and returns ![v$_1$,$\ldots$,v$_{n-1}$]!.
\end{itemize}
These gradients are some examples among numerous ones which are usually derived by hand, and are here obtained automatically as special cases of our work.
\end{example}   

\[
\begin{tikzcd}
    \Source \ar[rrrr,"\text{efficient }\Dsynrevsymbol (Fig.~\ref{fig:direct_diff_macro})"] \ar[dr,"(Fig.~\ref{fig:source_to_unf})"'] &&&& \Target \arrow[r,loop right,"\text{optim }(Fig.~\ref{fig:optim})"] \\
    & Source\UNFSymbol \ar[rr,"\Dsynrevsymbol (Fig.~\ref{fig:diff_macro})"] && Target\UNFSymbol \ar[ur,"(Fig.~\ref{fig:unf_to_target})"'] & 
\end{tikzcd}
\]

\subsection{Contributions}
We propose a source-code transformation on a simple purely functional language for purely functional reverse-mode differentiation.
Our transformation is comprised of a compilation scheme that is outlined in Figure~\ref{fig:outline}.
We make the following contributions:

\begin{itemize}[leftmargin=*]
\item We present our work with a simple yet expressive array-based language (with constructs such as !map2! and !reduce!) in Section~\ref{sec:simplediff}. 
We show how to directly compute an efficient reverse-mode AD for the expressions of this program (top of Figure~\ref{fig:outline}).  
Furthermore, we show how to extend our work to a richer language in Section~\ref{sec:generalization}.
\item One of the key insights behind efficient reverse-mode AD is to only consider unary operators. 
Inspired by this insight and following IRs such as SSA and ANF, we introduce a novel IR, UNF (Section~\ref{sec:unf})).
We introduce an alternative and easier to follow compilation pipeline for efficient reverse-mode AD (bottom of Figure~\ref{fig:outline}).
\item We prove the complexity guarantees for the reverse ADed programs. Furthermore, we show a list of optimizations that can further improve the constant factors (Section~\ref{sec:complexity}).
\item We proof the correctness of our transformations (top/bottom parts of Figure~\ref{fig:outline}) by defining the denotational semantics of our languages using multicategories and concategories (Section~\ref{sec:correctness}).
\end{itemize}

Next, we recall rudiments of automatic differentiation, forward and reverse-mode differentiation.

% \subsection{Summary of contributions}

% We propose a source-code transformation on a simple purely functional language for purely functional reverse-mode differentiation.
% Our transformation is comprised of a compilation scheme through a novel intermediate representation (IR): unary form (UNF).
% To emphasize the key ideas and for expository purposes, we present our work with a simple yet expressive language in Section~\ref{sec:simplediff}, 
% and show how to extend our work to a richer language in Section~\ref{sec:generalization}.

% We summarize the key ideas of our work. Reverse-mode can be made efficient
% \begin{itemize}
%    \item easily if there are only unary operators
%    \item in a purely functional way
%    \item on expressive constructs such as second-order array operators (e.g. !map2, reduce!) 
%    \item via standard functional optimization techniques
% \end{itemize}

% Our contribution makes these claims precise. In more details, we introduce
% \begin{itemize}
%     \item an expressive source language with array operations, yet restrictive enough for efficient differentiation (\S\ref{sub:sourcelang})
%     \item a method for extending our work and modular approaches to add and prove correct new primitives as needed in practice (\S\ref{sec:generalization})
%     \item a new intermediate representation (IR), UNF, essentially turning every operation into a unary one which inputs and outputs a tuple (\S\ref{sec:unf})
%     \item a transformation between the IR and the main language (\S\ref{sub:transformations to and from UNF})
%     \item a simple reverse-mode transformation on the IR (\S\ref{sub:Simple reverse mode transformation})
%     \item the direct reverse-mode transformation obtained from the source to the target language (\S\ref{sub:Macro for pure reverse mode transformation})
%     \item a denotational semantics of our languages and transformations using multicategories and concategories (\S\ref{sec:correctness})
%     \item correctness of the transformations (\S\ref{sec:correctness})
%     \item complexity guarantees (\S\ref{sec:complexity})
%     \item how efficiency is ensured via standard optimization techniques (\S\ref{sub:Optimizations})
%     \item several additions to the source language (\S\ref{sec:generalization})
% \end{itemize}

% \subsection{Outline of the paper}

% Our main transformation is present on the top of the diagram and presented in Section~\ref{sec:simplediff}. 
% It it decomposed 3 steps which are presented in Section~\ref{sec:unf}.

% \[
\begin{tikzcd}
    \Source \ar[rrrr,"\text{efficient }\Dsynrevsymbol (Fig.~\ref{fig:direct_diff_macro})"] \ar[dr,"(Fig.~\ref{fig:source_to_unf})"'] &&&& \Target \arrow[r,loop right,"\text{optim }(Fig.~\ref{fig:optim})"] \\
    & Source\UNFSymbol \ar[rr,"\Dsynrevsymbol (Fig.~\ref{fig:diff_macro})"] && Target\UNFSymbol \ar[ur,"(Fig.~\ref{fig:unf_to_target})"'] & 
\end{tikzcd}
\]

% In Section~\ref{sec:background} we recall rudiments of automatic differentiation, forward and reverse-mode differentiation.
% Next, in Section~\ref{sec:simplediff} we introduce the source and target languages for differentiation, and our reverse mode transformation. 
% In Section~\ref{sec:unf} we introduce a new intermediate representation UNF, and a simple reverse-mode macro on this representation.
% We introduce transformations from the source language to UNF and from UNF to the target language.
% Next, in Section~\ref{sec:correctness} we give denotational semantics to our languages and show that our decomposition respects the diagram above and that our reverse mode transformation is correct.
% In Section~\ref{sec:complexity}, we show that our transformation verifies the cheap gradient principle and extra general optimizations.
% In Section~\ref{sec:generalization} we show how to adapt our work for a richer source language. 
% Finally, related work, limitations to our approach and future work are presented in Section~\ref{sec:conclusion}.