\section{Simple pure reverse-mode differentiation}
\label{sec:simplediff}

\subsection{Source Language} 

We consider a standard call-by-value language. 
It consist in a first-order functional language enriched with arrays and a few typical second-order array operations. 
The types !T1,T2! and terms !e1,e2! are given in Figure~\ref{fig:source_grammar}.

\begin{figure*}[t]
\setlength{\tabcolsep}{0.3em}
\centering
\begin{tabular}{|l c l|l|}
\hline
\multicolumn{3}{|c|}{\textbf{Core Grammar}} & \multicolumn{1}{c|}{\textbf{Description}}\\\hline
!e! & \mbox{::=} & !sum(x in e) e! & \grammarcomment{Collection Aggregation} \\
% & $\mid$ & !build(x <- e) e! \\
& $\mid$ & !{ e -> e, ... }! & \grammarcomment{Dictionary Construction}\\
% & $\mid$ & ![[ e ]]! & \grammarcomment{Set Construction}\\
% & $\mid$ & !dom(e)! & \grammarcomment{Key Set of Dictionary}\\
& $\mid$ & !e(e)! & \grammarcomment{Dictionary Lookup}\\
& $\mid$ & !< a = e, ... >! & \grammarcomment{Record Construction}\\
& $\mid$ & !e.a! & \grammarcomment{Field Access}\\
& $\mid$ & !let x = e in e! & \grammarcomment{Variable Binding}\\
& $\mid$ & !if(e) then e else e! & \grammarcomment{Conditional}\\
& $\mid$ & !e + e! $\mid$ !e * e! & \grammarcomment{Addition \& Multiplication}\\
& $\mid$ & !n! $\mid$ !r! $\mid$ !false! $\mid$ !true! & \grammarcomment{Numeric \& Boolean Values}\\
& $\mid$ & !dn! & \grammarcomment{Dense Integer Value}\\
% & $\mid$ & !"some_text"! & \grammarcomment{String Literal}\\ 
\hline
!T! & \mbox{::=} & !{ T -> T }! & \grammarcomment{Dictionary Type} \\
& $\mid$ & !< a:T, ... >! & \grammarcomment{Record Type}\\
& $\mid$ & !S! & \grammarcomment{Scalar Type}\\
!S! & \mbox{::=} & !int! $\mid$ !real! $\mid$ !bool! & \grammarcomment{Numeric \& Boolean Types}\\
& $\mid$ & !denseint! & \grammarcomment{Dense Integer Type}\\
% & $\mid$ & !string! & \grammarcomment{String Type}\\
\hline
\end{tabular}
\vspace{0.2cm}
\caption{Grammar of the core part of the language.}
% \vspace{-0.4cm}
\label{fig:lang}
\end{figure*}


The typing rules are in Figure~\ref{fig:source_typesystem}. We have included a minimal set of array operations for the sake of illustration, 
but it is not difficult to add further operations. 
For scalar operations, we assume given a set of operations, including $+$ and $*$. 
!op1! and !op2! denote respectively a unary and a binary smooth operation on reals. 
Typical examples include !cos, exp, +, *!. 
We use infix notation for binary operators.
For the sake of simplicity in the presentation, a bound function in an array operation !map, map2, foldl! is restricted to having no free variables.
For the same reason, we restrict arrays of reals.
The general cases are presented in Section~\ref{sec:generalisation}.

% \begin{figure*}[t]
%     \centering
%     \begin{tabular}{c}
%     $\Gamma \vdash$ !e1!: !{ T1 -> T2 }! $\quad$ $\Gamma$, !x!: !<key:T1,val:T2>! $\vdash$ !e2!: !T3! \\\hline
%     !sum(x <- e1) e2!: !T3!
%     \end{tabular}
    
%     \begin{tabular}{c}
%     !k_1!: !T1! $\quad$ !v_1!: !T2! $\quad$ ... $\quad$ !k_n!: !T1! $\quad$ !v_n!: !T2!\\\hline
%     !{ k_1 -> v_1, ..., k_n -> v_n }!: !{ T1 -> T2 }!
%     \end{tabular}
%     \hspace{0.5cm}
%     \begin{tabular}{c}
%     !e1!: !{ T1 -> T2 }! $\quad$ !e2!: !T1!\\\hline
%     !e1(e2)!: !T2!
%     \end{tabular}
%     \hspace{0.5cm}
%     \begin{tabular}{c}
%     !e_1!: !T_1! $\quad$ ... $\quad$ !e_n!: !T_k!\\\hline
%     !<a_1=e_1,...,a_k=e_k>!: !<a_1:T_1,...,a_n:T_k>!
%     \end{tabular}
%     \hspace{0.5cm}
%     \begin{tabular}{c}
%     !e!: !<a_1:T_1,...,a_k:T_k>!\\\hline
%     !e.a_i!: !T_i!
%     \end{tabular}
%     \hspace{0.5cm}
%     \begin{tabular}{c}
%     $\Gamma \vdash$ !e1!: !T1! $\quad$ $\Gamma$, !x!: !T1! $\vdash$ !e2!: !T2! \\\hline
%     !let x = e1 in e2!: !T2!
%     \end{tabular}
    
%     \begin{tabular}{c}
%     !e1!: !bool! $\quad$ !e2!: !T! $\quad$ !e3!: !T! \\\hline
%     !if(e1) then e2 else e3!: !T!
%     \end{tabular}
%     \hspace{0.5cm}
%     \begin{tabular}{c}
%     !e1!: !T! $\quad$ !e2!: !T!\\\hline
%     !e1 + e2!: !T!
%     \end{tabular}
%     \hspace{0.5cm}
%     \begin{tabular}{c}
%     !e1!: !T1! $\quad$ !e2!: !T2!\\\hline
%     !e1 * e2!: !T1$\otimes$T2!
%     \end{tabular}
    
    
%     \begin{tabular}{c}
%     !e!: !S1$\otimes$S2!\\\hline
%     !e!: !S1$\vee$S2!
%     \end{tabular}
%     \hspace{0.5cm}
%     \begin{tabular}{c}
%     !e!: !S$\otimes${T1->T2}!\\\hline
%     !e!: !{T1->S$\otimes$T2}!
%     \end{tabular}
%     \hspace{0.5cm}
%     \begin{tabular}{c}
%     !e!: !S$\otimes$<a1:T1, ...>!\\\hline
%     !e!: !<a1:S$\otimes$T1, ...>!
%     \end{tabular}
    
%     \begin{tabular}{c}
%     !e!: !{T1->T2}$\otimes$T0!\\\hline
%     !e!: !{T1->T2$\otimes$T0}!
%     \end{tabular}
%     \hspace{0.5cm}
%     \begin{tabular}{c}
%     !e!: !<a1:T1, ...>$\otimes$T0!\\\hline
%     !e!: !<a1:T1$\otimes$T0, ...>!
%     \end{tabular}
%     \vspace{-0.2cm}
%     \caption{Type system of \lang.}
%     \vspace{-0.4cm}
%     \label{fig:typesystem}
%     \end{figure*}

    \begin{figure}[tb]
        \fbox{
          \parbox{\textwidth}{
      
            \[
                \begin{array}{ll}
                  \inferrule{
                    (\var : \ty) \in \Gamma
                  }{
                    \Gamma \vdash \var\ty
                  }
                &
                 \inferrule{
                    (\var : \ty) \in \Gamma
                  }{
                    \Gamma \vdash \var\ty
                  }
                \end{array} 
                \]
        \caption{Type system}
        \label{fig:type_system}}
        \end{figure}

\subsection{Target Language}

The target language for differentiation is an extension to the source language, 
which is higher-order as needed to carry the continuation used to have a pure reverse-mode transformation, 
and contains more array primitives. In addition, we require the set of operations to be closed under partial differentiation. 

The types and terms are presented in Figure~\ref{fig:target_grammar}.
We are not interested here with partial applications and our language is pure, so functions take $n$ arguments.
Lambda abstractions and applications will be removed during partial evaluation and this notation makes reading slightly easier.

\begin{figure*}[t]
    \setlength{\tabcolsep}{0.3em}
    \centering
    \begin{tabular}{|l c l|l|}
    \hline
    \multicolumn{3}{|c|}{\textbf{Core Grammar}} & \multicolumn{1}{c|}{\textbf{Description}}\\\hline
    !T! & \mbox{::=} & ... & \grammarcomment{Same as Source} \\
    & $\mid$ & !T->T! & \grammarcomment{Function Type}\\ 
    \hline
    !e! & \mbox{::=} & ... & \grammarcomment{Same as Source}\\
    & $\mid$ & !fun (x$_1$,...,x$_n$) -> e! & \grammarcomment{Lambda Abstraction}\\
    & $\mid$ & !e(e$_1$...e$_n$)! & \grammarcomment{Function Application}\\
    & $\mid$ & !scanl (x,y.e) e e! $\mid$ !scanr (x,y.e) e e! & \grammarcomment{Array scan left and right}\\
    & $\mid$ & !shift1L e! & \grammarcomment{Array left shifting}\\
    & $\mid$ & !shift1R e! & \grammarcomment{Array right shifting}\\
    \hline
    \end{tabular}
    \vspace{-0.2cm}
    \caption{Grammar of the target language.}
    \label{fig:target_grammar}
    \end{figure*}
    

In more detail, for every unary scalar operation !op1!, 
we assumed given an operator $\partial_1$!op1! whose semantics should be the derivative of !op1!.
For every binary operator !op2!, we assume given operators $\partial_1$!op2!, $\partial_2$!op2! 
representing respectively the first and second partial derivative of !op2!.
Similarly, some additional operations on arrays are needed. 
!scanl! is a !foldl! that stores all intermediate results. 
!scanr! is similar, but reads the array from right to left, and also stores from right to left.
!reduce! is a !foldl! for which the function is associative. 
This means in practice that it can be computed faster.
We also include a !map3! operator. 
Though not necessary as it can be simulated with two !map2!, it is a convenient notation and can sometimes run faster in practice.
Finally, we add a somewhat strange operator which we call !shift1!. 
It takes an array of size $n$,  shifts all the element of the array by one, 
forgetting the last element, and puts a one in the first place. 
This operator naturally shows up when differentiating !foldl!.

The type system for the extended grammar of target is presented in Figure~\ref{fig:target_typesystem}. 

\begin{figure*}[tb]
    \centering

    \begin{tabular}{c}
        $\Gamma$, !x!: $\reals$, !y!: $\reals$ $\vdash$ !e1!: $\reals$ 
        $\quad$ $\Gamma$ $\vdash$ !e2!: $\reals$
        $\quad$ $\Gamma$ $\vdash$ !e3!: $\Array{\reals}$
        \\\hline  
        $\Gamma \vdash$ !reduce (x,y.e1) e2 e3!: $\reals$
    \end{tabular}

    \begin{tabular}{c}
        $\Gamma$, !x!: $\reals$, !y!: $\reals$ $\vdash$ !e1!: $\reals$ 
        $\quad$ $\Gamma$ $\vdash$ !e2!: $\reals$
        $\quad$ $\Gamma$ $\vdash$ !e3!: $\Array{\reals}$
        \\\hline  
        $\Gamma \vdash$ !scanl (x,y.e1) e2 e3!: $\Array{\reals}$
    \end{tabular}

    \begin{tabular}{c}
        $\Gamma$, !x!: $\reals$, !y!: $\reals$ $\vdash$ !e1!: $\reals$ 
        $\quad$ $\Gamma$ $\vdash$ !e2!: $\reals$
        $\quad$ $\Gamma$ $\vdash$ !e3!: $\Array{\reals}$
        \\\hline  
        $\Gamma \vdash$ !scanr (x,y.e1) e2 e3!: $\Array{\reals}$
    \end{tabular}

    \begin{tabular}{c}
        $\Gamma$, !x!: $\reals$, !y!: $\reals$ $\vdash$ !e1!: $\reals$ 
        $\quad$ $\Gamma$ $\vdash$ !e2!: $\Array{\reals}$
        $\quad$ $\Gamma$ $\vdash$ !e3!: $\Array{\reals}$
        $\quad$ $\Gamma$ $\vdash$ !e4!: $\Array{\reals}$
        \\\hline  
        $\Gamma \vdash$ !map3 (x,y.e1) e2 e3 e4!: $\Array{\reals}$
    \end{tabular}

    \begin{tabular}{c}
        $\Gamma$, !x1!: !G1!, $\ldots$, !xn!: !Gn! $\vdash$ !e!: !T! 
        \\\hline  
        $\Gamma \vdash$ !fun (x1,...,xn) -> e!: !G1x...xGn->T!
    \end{tabular}

    \begin{tabular}{c}
        $\Gamma$ $\vdash$ !e!: !G1x...xGn -> T!
        $\quad$ $\Gamma$ $\vdash$ !ei!: !Gi! for all $1\leq i\leq n$
        \\\hline  
        $\Gamma \vdash$ !ee1...en!: !T!
    \end{tabular}

    \vspace{-0.2cm}
    \caption{Type system of the target language}
    \vspace{-0.4cm}
    \label{fig:target_typesystem}
    \end{figure*}    

A standard semantics of this language is given in Section~\ref{sec:correctness}.

\subsection{Macro for pure reverse mode transformation} % (fold)
\label{sub:Macro for pure reverse mode transformation}

In Figure~\ref{fig:direct_diff_macro} we present the direct transformation for pure reverse mode differentiation.
We explain in the next Section (\ref{sub:Partial evaluation and optimisation}) how to optimise it further and compute the gradient.
This transformation may be hard to read and prove correct. 
In Section~\ref{sec:unf} we decompose this transformation into several easier 
and more intuitive steps via a new intermediate representation which we call UNF.

TODO: on the examples

\begin{figure*}[t]
    \begin{tabular}{r c l}
        $\directD{\rho}{\Gamma}{Y}$($\Gamma\vdash $ !c!) &=& 
            !<c, fun (x1,...,xn,z) ->! \\
            && !Y(x1,...,xn) >!\\
        $\directD{\rho}{\Gamma}{Y}$($\Gamma\vdash $ !x!) &=& 
            !<x, fun (x1,...,xn,z) ->! \\
            && !Y((x1,...,xn)!$\widehat{+}$![pos(x)]z) >!\\
        $\directD{\rho}{\Gamma}{Y}$($\Gamma\vdash $ !let x:A = e1 in e2!) &=& 
            !let x,Y1 = !$\directD{\rho}{\Gamma}{Y}$!(e1) in! \\
            &&!let y,Y2 = !$\directD{\rho}{\Gamma,x:A}{Y1}$!(e2) in!\\ 
            &&!<y, fun (x1,...,xn,z) -> Y2(x1,...,xn,!$0_{A}$!,z)>!\\
        $\directD{\rho}{\Gamma}{Y}$($\Gamma\vdash $ !< e1, e2 >!) &=&
            !let x1,Y1 = !$\directD{\rho}{\Gamma}{Y}$!(e1) in! \\
            &&!let x2,Y2 = !$\directD{\rho}{\Gamma}{Y1}$!(e2) in!\\
            &&!< <x,y>, fun (x1,...,xn,z) -> !\\
            &&!Y(x1,...,xn,!$\pi_1$!(z),!$\pi_2$!(z)) >!\\ 
        $\directD{\rho}{\Gamma}{Y}$($\Gamma\vdash \pi_1$(!e!:AxB)) &=&
            !let (x,y),Y1 = !$\directD{\rho}{\Gamma}{Y}$!(e) in! \\
            && !<x, fun (x1,...,xn,z) -> Y(x1,...,xn,(z,!$0_B$!))>! \\
        $\directD{\rho}{\Gamma}{Y}$($\Gamma\vdash \pi_2$(!e!:AxB)) &=&
            !let (x,y),Y1 = !$\directD{\rho}{\Gamma}{Y}$!(e) in! \\
            && !<x, fun (x1,...,xn,z) -> Y(x1,...,xn,(!$0_A$,z!))>! \\
        $\directD{\rho}{\Gamma}{Y}$($\Gamma\vdash $ !op1 e!) &=&  
            !let x,Y1 = !$\directD{\rho}{\Gamma}{Y}$!(e) in! \\
            && !<op1 x, fun (x1,...,xn,z) -> ! \\
            && !Y(x1,...,xn,!$\partial$!op1(x)*z) >! \\
        $\directD{\rho}{\Gamma}{Y}$($\Gamma\vdash $ !e1 op2 e2!) &=& 
            !let x1,Y1 = !$\directD{\rho}{\Gamma}{Y}$!(e1) in! \\
            && !let x2,Y2 = !$\directD{\rho}{\Gamma}{Y1}$!(e2) in! \\
            && !<x1 op2 x2, fun (x1,...,xn,z) ->! \\
            && !Y2(x1,...,xn,!$\partial_1$!op2(x1,x2)*z,!$\partial_2$!op2(x1,x2)*z>! \\
        $\directD{\rho}{\Gamma}{Y}$($\Gamma\vdash $ !map2 (x,y.e1) e2 e3!) &=&  
            !let A,Y1 = !$\directD{\rho}{\Gamma}{Y}$!(e2) in! \\
            && !let B,Y2 = !$\directD{\rho}{\Gamma}{Y1}$!(e3) in! \\
            && !let C =!$\grad_{\Gamma}$!e1 in!\\
            && !<map2 (x,y.e1) A B, fun (x1,...,xn,Z) -> !\\
            && !Y2( (x1,...,xn)!$\widehat{+}$!(C*(reduce + 0 Z)),!\\
            && \quad\quad! map2 (a,b.(!$\grad_{\{x\}}$!e1)[a/x]*b) A Z,!\\
            && \quad\quad! map2 (a,b.(!$\grad_{\{y\}}$!e1)[a/x]*b) B Z )>!\\
        $\directD{\rho}{\Gamma}{Y}$($\Gamma\vdash $ !reduce (x,y.e1) e2 e3!) &=&
            !let A,Y1 = !$\directD{\rho}{\Gamma}{Y}$!(e2) in! \\
            && !let B,Y2 = !$\directD{\rho}{\Gamma}{Y1}$!(e3) in! \\
            && !let C = scanl (x,y.e1) A B in! \\
            && !let D = map2 (a,b.(!$\grad_{\{x\}}$!e1)[a/x,b/y]) C B! \\
            && !let E = shift1 (map2 (a,b.(!$\grad_{\{y\}}$!)e1[a/x,b/y]) C B)! \\
            && !let F = scanr * 1 D! \\
            && !<reduce (x,y.e1) A B, fun (x1,...,xn,z) ->! \\
            && !Y2(x1,...,xn, map2 (x,y. x*y*z) E F>! \\
            && TODO: reduce recomputed after scanl, wrong on the zero element \\
        \end{tabular}
    \caption{Reverse-mode transformation from source to target language}
    \label{fig:direct_diff_macro}    
\end{figure*}

\subsection{Partial evaluation and optimisation} % (fold)
\label{sub:Partial evaluation and optimisation}

TODO: how to get gradient or subpart of the gradient from the previosu term.

TODO: maybe intuition around here

TODO: on the examples

TODO: partial evaluation to get a tuple (i.e no lambdas)

TODO: use the optimisations to be efficient, just on example?

The full list of optimisations can be found on Figure~\ref{fig:optim}.