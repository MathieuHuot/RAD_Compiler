\section{Complexity}
\label{sec:complexity}

\subsection{Cost model}

We follow a simple model similar to the one in \cite{griewank2008evaluating}.
We assume the cost is divided into 4 elementary measures, being the number of MOVES, ADDS, MULTS, and NLOPS.
MOVES assumes a flat memory and represents moving a fixed size information  (e.g. 64 bits). 
ADDS represents the number of additions, 
MULTS the number of multiplications, 
and NLOPS the number of elementary non linear operations like !cos!, !exp!.

The gives a complexity function $\cost$ valued in $\RR^4$. 
For instance, we have 

\begin{itemize}
    \item $\cost(*)=(3,0,1,0)$
    \item $\cost(+)=(3,1,0,0)$
    \item $\cost(c)=(1,0,0,0)$
    \item $\cost($!sin!$)=(2,0,0,1)$
\end{itemize}

\MH{need cost array operations}
\MH{need to extend the cost function in a compositional way}


\subsection{Cheap gradient principle}

\begin{theorem}
    The reverse mode transformation satisfies the cheap gradient principle.
\end{theorem}

\subsection{Optimisations} % (fold)
\label{sub:Optimisations}

