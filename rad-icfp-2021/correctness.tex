\section{Correctness}
\label{sec:correctness}

This section provides a categorical analysis of our language and transformations.
The end result is the correctness of the reverse mode transformation from Section~\ref{sub:Macro for pure reverse mode transformation}:

 \begin{theorem}
     For every $\Gamma \vdash$ !e:!$\reals$, we have 
     $\sem{\grad_\Gamma e}= \grad_\Gamma\sem{e}$.
 \end{theorem}
where $\sem{-}$ is ab appropriate denotational semantics of smooth functions.

\subsection{Denotational semantics source and Target} % (fold)
\label{sub:Denotational semantics source and Target}

It is standard to give a denotational semantics of a first-order language in a Cartesian category.
An alternative approach consists in giving a semantics in a multicategory.
A multicategory generalises a category by allowing multimorphims, that is morphisms from a list of objects to an object.
Most categorical structures from category theory can be phrases similarly in multicategories.

In terms of semantics, a term $x_1:A_1,\ldots,x_n:A_n\vdash e:A$ is interpreted as a morphism $\sem{e}:[\sem{A_1},\ldots,\sem{A_n}]\to \sem{A}$ in a multicategory.
Substitution is still interpreted as composition. 

One can consider a syntactic model for a language, which consists in a free multicategory.
Our source and target languages induce syntactic multicategories as follows.

\begin{definition}{Syntactic multicategory for Source}
    Let $\SynSource$ be the multicategory whose objects are types of Source, and where a morphism 
    $[A_1,\ldots,A_n]\to A$ is a term $x_1:A_1,\ldots,x_n:A_n\vdash e:A$ of Source modulo the $\eta\beta$-laws.
    Composition is by substitution.
\end{definition}

We similarly define $\SynTarget$, the syntactic multicategory for Target.

As is standard, $\SynSource$ satisfies the following universal property: 
for every Cartesian multicategory $\catC$ with arrays map and reduce,
and every object $F(\reals)\in\catC$ and morphisms $F(\underline{c})\in\catC(1;F(\reals))$, 
$F(op1)\in\catC(F(\reals);F(\reals))$, $F(op2)\in\catC(F(\reals),F(\reals);F(\reals))$, there is a unique
multifunctor $F:\SynSource\to\catC$ respecting the interpretation and preserving all the structure.

We can for instance use this to give a simple semantics of Source 
in the multicategory of Cartesian spaces and smooth maps between them. 

\begin{definition}{CartSp}
    TODO, its definition and semantics of Source in it.
\end{definition}

TODO: what about semantics Target?

\subsection{Semantics for UNF using concategories} % (fold)
\label{sub:Semantics for UNF using concategories}

\begin{definition}{concategory}
    TODO
\end{definition}

\begin{definition}{free concategory}
    TODO
\end{definition}

\begin{example}{Concat1, Concat2}
    TODO
\end{example}

TODO: semantics of UN1, UNF2 inside Concat1, Concat2

\subsection{Semantics for UNF transformations} % (fold)
\label{sub:Semantics for UNF transformations}

\begin{definition}{Mutlicat from concat}
    TODO
\end{definition}

\begin{proposition}{construction above gives UNF}
    TODO
\end{proposition}

\begin{definition}{lax functor between concat}
    TODO
\end{definition}

\begin{example}{D as a lax functor}
    TODO
\end{example}

\begin{proposition}{semantic of D syntactic matches D lax functor}
    TODO
\end{proposition}

\begin{definition}{Concat from multicat}
    TODO
\end{definition}

\begin{proposition}{Initiality of concat2 gives $UNF^-1$}
    TODO
\end{proposition}

\subsection{Correctness theorem} % (fold)
\label{sub:Correctness theorem}

\begin{proposition}{The transpose Jacobian of !map2! and !reduce! are correct.}
    TODO
\end{proposition}

\begin{theorem}{UNF;D;$UNF^-1$ correct}
    TODO
\end{theorem}

\begin{proposition}{UNF;D;$UNF^-1$=D}
    TODO
\end{proposition}

\begin{theorem}{D is correct}
    TODO
\end{theorem}