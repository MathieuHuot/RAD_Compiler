\section{Correctness}
\label{sec:correctness}

This section provides a categorical analysis of our language and transformations.
The end goal is to prove the correctness of the reverse mode transformation from Section~\ref{sub:Macro for pure reverse mode transformation}.

 \begin{theorem}
     For every $\Gamma \vdash$ !e:!$\reals$, we have 
     $\sem{\grad_\Gamma e}= \grad_\Gamma\sem{e}$.
 \end{theorem}
where $\sem{-}$ is ab appropriate denotational semantics of smooth functions.

\subsection{Denotational semantics source and Target} % (fold)
\label{sub:Denotational semantics source and Target}

\begin{definition}{multicategory}
    TODO
\end{definition}

\begin{definition}{free multicategory}
    TODO
\end{definition}

\begin{example}{Source and Target as syntactic multicategories}
    TODO
\end{example}

\begin{definition}{CartSp}
    TODO
\end{definition}

TODO: semantics of Source in CartSp, what about semantics Target?

\subsection{Semantics for UNF using concategories} % (fold)
\label{sub:Semantics for UNF using concategories}

\begin{definition}{concategory}
    TODO
\end{definition}

\begin{definition}{free concategory}
    TODO
\end{definition}

\begin{example}{Concat1, Concat2}
    TODO
\end{example}

TODO: semantics of UN1, UNF2 inside Concat1, Concat2

\subsection{Semantics for UNF transformations} % (fold)
\label{sub:Semantics for UNF transformations}

\begin{definition}{Mutlicat from concat}
    TODO
\end{definition}

\begin{proposition}{construction above gives UNF}
    TODO
\end{proposition}

\begin{definition}{lax functor between concat}
    TODO
\end{definition}

\begin{example}{D as a lax functor}
    TODO
\end{example}

\begin{proposition}{semantic of D syntactic matches D lax functor}
    TODO
\end{proposition}

\begin{definition}{Concat from multicat}
    TODO
\end{definition}

\begin{proposition}{Initiality of concat2 gives $UNF^-1$}
    TODO
\end{proposition}

\subsection{Correctness theorem} % (fold)
\label{sub:Correctness theorem}

\begin{proposition}{The transpose Jacobian of !map2! and !reduce! are correct.}
    TODO
\end{proposition}

\begin{theorem}{UNF;D;$UNF^-1$ correct}
    TODO
\end{theorem}

\begin{proposition}{UNF;D;$UNF^-1$=D}
    TODO
\end{proposition}

\begin{theorem}{D is correct}
    TODO
\end{theorem}